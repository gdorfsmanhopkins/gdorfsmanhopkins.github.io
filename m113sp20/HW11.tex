\documentclass[11pt]{article}
\usepackage[top = 1in, bottom = 1in, left =1in, right = 1in]{geometry}
\usepackage{graphicx}
\usepackage{amsmath}
\usepackage{tabu}
\usepackage{amssymb}
\usepackage{etoolbox}
\usepackage{xcolor}
\usepackage{amsthm}
\usepackage{tikz-cd}
\usepackage{tikz}
\AtBeginEnvironment{proof}{\color{blue}}
\newtheorem{theorem}{Theorem}
\newtheorem{proposition}{Proposition}
\newtheorem{lemma}{Lemma}
\newtheorem*{remark}{Remark}
\newtheorem{corollary}{Corollary}
\newtheorem{definition}{Definition}
\usepackage{enumerate}
\usepackage{hyperref}
\usepackage{fancyhdr}\pagestyle{fancy}
\newcommand{\la}{\langle}
\newcommand{\ra}{\rangle}
\newcommand{\tors}{\mathrm{tors}}
\newcommand{\ab}{\mathrm{ab}}
\newcommand{\Aut}{\operatorname{Aut}}
\newcommand{\Inn}{\operatorname{Inn}}
\newcommand{\im}{\operatorname{im}}
\newcommand{\lcm}{\operatorname{lcm}}
\newcommand{\ch}{\operatorname{char}}

%Math blackboard:
\newcommand{\bC}{\mathbb{C}}
\newcommand{\bF}{\mathbb{F}}
\newcommand{\bN}{\mathbb{N}}
\newcommand{\bQ}{\mathbb{Q}}
\newcommand{\bR}{\mathbb{R}}
\newcommand{\bS}{\mathbb{S}}
\newcommand{\bZ}{\mathbb{Z}}

%Greek blackboard font:
\newcommand{\bmu}{\mbox{$\raisebox{-0.59ex}
  {$l$}\hspace{-0.18em}\mu\hspace{-0.88em}\raisebox{-0.98ex}{\scalebox{2}
  {$\color{white}.$}}\hspace{-0.416em}\raisebox{+0.88ex}
  {$\color{white}.$}\hspace{0.46em}$}{}}

\lhead{University of California, Berkeley}
\rhead{Math 113 Section 6, Spring 2020}

\begin{document}
\begin{center}
\Large {Homework 11}\\
\small {Due Monday, May 4th}
\end{center}
In this assignment we fill the proofs of a few crucial lemmas from lecture and takehome 3.  Studying semidirect products reduces to the study of automorphism groups, so our first goal is to get a good way to decompose them.  Here is the goal:
\begin{lemma}\label{autDecomp}
  Let $H$ and $K$ be finite groups whose orders are coprime.  Then \[\Aut(H\times K)\cong\Aut H\times \Aut K.\]
\end{lemma}
The following definition will be useful.
\begin{definition}
  Let $\varphi:G\to G'$ be a homomorphism, and let $H\le G$.  The \textit{restriction of} $\varphi$ \textit{to }$H$ is the map $\varphi|_H:H\to G'$ given by evaluating $\varphi$ on elements of $H$.
\end{definition}
Let's consider it obvious that $\varphi|_H$ is a homomorphism (why?), and so you may use this fact without proof.
\begin{enumerate}
  \item{
  Let's prove Lemma \ref{autDecomp}.
  \begin{enumerate}
    \item{
    Let $G$ be a group and let $H\ch G$ be a \textit{characteristic subgroup} (recall the definition from HW9 Problem 1).  Fix any automorphism $\varphi\in\Aut G$. Show that $\varphi|_H$ is an automorphism of $H$.  (Hint: you must first show its image lands in $H$ so you can consider it as a map from $H$ to itself).
    }
    \item{
    With $H$ and $G$ as in part (a), show that the rule $\varphi\mapsto\varphi|_H$ is a homomorphism $\Aut G\to\Aut H$.
    }
    \item{
    Let $H,K$ be finite groups of coprime orders.  Show that $H$ and $K$ are characteristic in $H\times K$.
    }
    \item{
    With $H,K$ as in (c), construct an isomorphism $\Aut(H\times K)\to\Aut H\times\Aut K$.
    }
  \end{enumerate}
  }
\end{enumerate}
Recall that any homomorphism $\varphi:K\to\Aut H$ allows us to build a semidirect product $H\rtimes_\varphi K$.  An interesting question is when different maps give us isomorphic semidirect products.  In class we stated and used the following lemma.
\begin{lemma}\label{leftSemidirect}
  Let $\varphi,\psi:K\to\Aut H$ be two homomorphisms, and suppose they differ by an automorphism of $K$.  That is, suppose there is some $\gamma\in\Aut(K)$ such that $\psi\circ\gamma = \varphi$:
  \[
  \begin{tikzcd}
    K\ar[dr,"\varphi"]\ar[dd,swap,"\gamma"] & \\
     & \Aut H\\
    K\ar[ur,swap, "\psi"] &
  \end{tikzcd}
  \]
  Then $H\rtimes_\varphi K\cong H\rtimes_\psi K$.
\end{lemma}
One could ask if this is the only thing that could allow different $\varphi$ to give different semidirect products.  The answer would be no, as the following lemma shows.
\begin{lemma}\label{rightSemidirect}
  Let $\varphi,\psi:K\to\Aut H$ be two homomorphisms, and suppose they are conjugate in $\Aut H$.  Explicitely, suppose there is some $\alpha\in\Aut H$, corresponding to the inner automorphism $\sigma_\alpha:\beta\mapsto \alpha\beta\alpha^{-1}$, and suppose that $\psi = \sigma_\alpha\circ\varphi$:
  \[
  \begin{tikzcd}
    &\Aut H\ar[dd,"\sigma_\alpha"]\\
    K\ar[ur,"\varphi"]\ar[dr,swap,"\psi"]\\
    &\Aut H
  \end{tikzcd}
  \]
  Then $H\rtimes_\varphi K\cong H\rtimes_\psi K$.
\end{lemma}
\begin{enumerate}
  \setcounter{enumi}{1}
  \item{The lemmas say that if we alter $\varphi$ by an automorphism of $K$, or an inner automorphism of $\Aut H$, (or both), we don't change the semidirect products.  Let's prove this.
  \begin{enumerate}
    \item{
    Consider the setup of Lemma \ref{leftSemidirect}.  Show that the map:
    \begin{eqnarray*}
      H\rtimes_\varphi K&\longrightarrow& H\rtimes_\psi K\\
      (h,k)&\mapsto&(h,\gamma(k))
    \end{eqnarray*}
    is an isomorphism, thereby proving the lemma.
    }
    \item{
    Consider the setup of Lemma \ref{rightSemidirect}.  Show that the map:
    \begin{eqnarray*}
      H\rtimes_\varphi K&\longrightarrow&H\rtimes_\psi K\\
      (h,k) &\mapsto& (\alpha(h),k)
    \end{eqnarray*}
    is an isomorphism, thereby proving the lemma.  (Notice that $\alpha\in\Aut H$ is an automorphism of $H$, wheras $\sigma_\alpha$ is an automorphism of $\Aut H$, given by conjugation by $\alpha$.  In unweildy notation, this says $\sigma_\alpha\in\Aut(\Aut H)$.)
    }
    \item{
    Now suppose $\varphi,\psi:K\to\Aut H$ are two homomorphisms, and suppose there is an automorphism $\gamma\in\Aut K$ and an inner automorphism $\sigma\in\Inn(\Aut(H))$ such that the following diagram commutes:
    \[
    \begin{tikzcd}
      K\ar[r,"\varphi"]\ar[d,swap,"\gamma"]&\Aut H\ar[d,"\sigma"]\\
      K\ar[r,"\psi"]&\Aut H.
    \end{tikzcd}
    \]
    That is, $\sigma\circ\varphi = \psi\circ\gamma$.  Then $H\rtimes_\varphi K\cong H\rtimes_\psi K$.  (Hint: This should follow formally from Lemmas \ref{leftSemidirect} and \ref{rightSemidirect}, so you shouldn't have to do any lengthy computations).
    }
  \end{enumerate}
  }
\end{enumerate}
To prove the uniqueness part of the fundamental theorem of finite abelain groups in Takehome 3, we made use of the following lemma.
\begin{lemma}\label{cancellation}
  Let $M,M',N,N'$ groups, and suppose $M\times N\cong M'\times N'$.  If $M$ and $M'$ are finite and $M\cong M'$ then $N\cong N'$.
\end{lemma}
\begin{remark}
  This is a slightly more general restatement of the lemma we used in the takehome.  In particular, before we identified $M$ and $M'$ as \textit{equal} rather than \textit{isomorphic}, and we assumed that $N,N'$ were finite as well.  We will see that this greater generality makes it a bit easier to prove.
\end{remark}
\begin{enumerate}
  \setcounter{enumi}{2}
  \item{
  Let's explore and prove Lemma \ref{cancellation}, and thereby fill the remaining hole in the fundamental theorem of finite abelian groups.  It is actually more subtle then you might think.
  \begin{enumerate}
    \item{
    You will need to make use of the following fact, so we prove it first.  If $G_1,G_2$ are groups and $H_i\unlhd G_i$ for $i=1,2$.  Then under the usual identifications, $H_1\times H_2\unlhd G_1\times G_2$ and:
    \[(G_1\times G_2)/(H_1\times H_2)\cong(G_1/H_1)\times(G_2/H_2).\]
    }
    \item{
    Give an example to show that Lemma \ref{cancellation} is not true without the finiteness assumption.  (Hint: Let $G$ a nontrivial group and $M = G\times G\times G\times\cdots$ an infinite product of copies of $G$).
    }
    \item{
    Identify $M\times N$ and $M'\times N'$ as the same group $G$.  Show that if either  $M'\cap N = 1$, or if $M\cap N'=1$ then Lemma \ref{cancellation} holds.  (Hint: 2nd isomorphism theorem).
    }
    \item{
    Prove Lemma \ref{cancellation} by induction on $|M|$.  (Hint: The base case is easy (why?).  For the general case, notice that if $H = M\cap N'$ or $K = M'\cap N$ are trivial, we are done by part (b).  Otherwise, try manipulating $G/(H\times K)$ to apply induction).
    }
  \end{enumerate}
  }
  \item{
  Let's finish with a classification problem.  Classify all groups of order 20 up to isomorphism.  How many are there total?  (You may use that if $p$ is prime, then $\Aut(Z_p)\cong Z_{p-1}$).
  }
\end{enumerate}
\end{document}
