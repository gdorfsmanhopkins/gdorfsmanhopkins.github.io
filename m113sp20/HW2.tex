\documentclass[11pt]{article}
\usepackage[top = 1in, bottom = 1in, left =1in, right = 1in]{geometry}
\usepackage{graphicx}
\usepackage{amsmath}
\usepackage{amssymb}
\usepackage{etoolbox}
\usepackage{xcolor}
\usepackage{amsthm}
\AtBeginEnvironment{proof}{\color{blue}}
\newtheorem{lemma}{Lemma}
\usepackage{enumerate}
\usepackage{hyperref}
\usepackage{fancyhdr}\pagestyle{fancy}

\lhead{University of California, Berkeley}
\rhead{Math 113 Section 6, Spring 2020}

\begin{document}
\begin{center}
\Large {Homework Assignment 2}\\
\small {Due Friday, February 7}
\end{center}
\begin{enumerate}
\item Fix $x\in\mathbb{Z}/m\mathbb{Z}$.  Recall that a \textit{multiplicative inverse} of $x$ is an element $y\in\mathbb{Z}/m\mathbb{Z}$ so that $xy=yx=\overline 1$.
\begin{enumerate}[(a)]
\item Show that $\overline a\in\mathbb{Z}/m\mathbb{Z}$ has a multiplicative inverse if and only if $\operatorname{gcd}(a,m)=1$.
\item Suppose $\overline a$ has a multiplicative inverse in $\mathbb{Z}/m\mathbb{Z}$  .  Show that this means we can solve equations of the form $\overline a x = \overline b$ for a congruence class $x$.
\item By part (a) we know that $\overline 3$ has a multiplicative inverse in $\mathbb{Z}/7\mathbb{Z}$.  What is it?  Use it to solve the equation $\overline 3x = \overline 4$ for $x$.
\end{enumerate}
\item Let $*$ denote multiplication modulo 15, and consider the set $\{3,6,9,12\}$.  Fill in the following multiplication table.
\begin{center}
\begin{tabular}{c|c|c|c|c}
* & 3 & 6 & 9 & 12\\
\hline
3 &&&&\\
\hline
6 &&&&\\
\hline
9 &&&&\\
\hline
12&&&&
\end{tabular}
\end{center}
Use the table to prove that $\left(\{3,6,9,12\},*\right)$ is a group.  What is the identity element?
\item Let $S$ be a set, and define $\operatorname{Aut}(S):=\{f:S\to S$ $|$ $f$ is bijective$\}$.  Define a binary operation by composition $f*g := g\circ f$.  Show that $\operatorname{Aut}(S)$ is a group.  We will call this the \textit{automorphism group of }$S$.
\item Prove the generalized associative law for groups.  Explicitly, for $G$ a group, and $b_1,b_2,\cdots,b_k$, then the product $b_1\times b_2\times\cdots\times b_k$ does not depend on the the bracketing.  (Hint: Use induction on $k$, with base cases 1, 2,and 3).
\item Compute the order of every element of $(\mathbb{Z}/7\mathbb{Z})^\times$.
\item Fix an element $x$ of a group $G$ and suppose $|x| = n$.  
\begin{enumerate}[(a)]
\item Show that $x^{-1}$ is a power of $x$.
\item Show that the all of $1,x,x^2,\cdots,x^{n-1}$ are distinct.  Conclude that $|x|\le|G|$.  (We will later show that if $|G|$ is finite then $|x|$ \textit{divides} $|G|$.)
\end{enumerate}
\item Fix elements $x,y$ of a group $G$, and suppose $xy=e$.  Show that $yx = e$.
\item Consider the presentation of the Dihedral group $D_{2n} = \langle r,s$ $|$ $r^n=s^2=1,rs=sr^{-1}\rangle$.  Use this presentation to show that every element which is not a power of $r$ has order 2.
\end{enumerate}
\end{document}