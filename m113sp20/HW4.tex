\documentclass[11pt]{article}
\usepackage[top = 1in, bottom = 1in, left =1in, right = 1in]{geometry}
\usepackage{graphicx}
\usepackage{amsmath}
\usepackage{amssymb}
\usepackage{etoolbox}
\usepackage{xcolor}
\usepackage{amsthm}
\AtBeginEnvironment{proof}{\color{blue}}
\newtheorem{lemma}{Lemma}
\usepackage{enumerate}
\usepackage{hyperref}
\usepackage{fancyhdr}\pagestyle{fancy}
\newcommand{\la}{\langle}
\newcommand{\ra}{\rangle}
\newcommand{\tors}{\mathrm{tors}}

\lhead{University of California, Berkeley}
\rhead{Math 113 Section 6, Spring 2020}

\begin{document}
\begin{center}
\Large {Homework Assignment 4}\\
\small {Due Friday, February 21}
\end{center}
\begin{enumerate}
	\item Let $G$ be a group.  Let $H,K\le G$ be two subgroups.
  \begin{enumerate}
    \item Show that the intersection $H\cap K$ is a subgroup of $G$.
    \item Give an example to show that the union $H\cup K$ need not be a subgroup of $G$.
    \item Show that $H\cup K$ is a subgroup of $G$ if and only if $H\subset K$ or $K\subset H$.
  \end{enumerate}

  \item Let $A$ be an \textit{abelian} group.
  \begin{enumerate}
    \item Let $A^n = \{a^n|a\in A\}$ be the collection of $n$th powers of elements in $A$.  Show that this is a subgroup of $A$.
    \item Let $A[n] = \{a\in A|a^n=1\}$.  Show that $A[n]$ is a subgroup of $A$.  This is often called the $n$\textit{-torsion} subgroup of $A$.
    \item Let $A^\tors = \{a\in A |\text{ }|a|<\infty\}$.  Show that $A^\tors$ is a subgroup of $A$.  This is often called the \textit{torsion} subgroup of $A$.
    \item Give an example of a nonabelian group $G$ where $G^\tors$ is not a subgroup of $G$.  (Note that $G$ must be infinite, as if $G$ were finite every element would have finite order so that we would have $G^\tors = G$).
  \end{enumerate}
  \item Compute the center of the dihedral group.  Explicitly, let $n$ be an integer $\ge3$.  Compute $Z(D_{2n})$.  (Note: you will need to split into the two cases, where $n$ is even or $n$ is odd).
  \item Let $G$ be a group.
  \begin{enumerate}
    \item Show that if $H$ is a subgroup of $G$, then $H\le N_G(H)$.
    \item Give an example where $A\subset G$ is a a subset (not necessarily a subgroup), and $A\not\subset N_G(A)$.
    \item Show that $H\le C_G(H)$ if and only if $H$ is abelian.
  \end{enumerate}
  \item In class we classified all finite cyclic groups and their generators.  In this exercise you take care of the infinite case.  Let $H = \la x\ra$ be a cyclic group of infinite order.
  \begin{enumerate}
    \item Show that the map $\varphi:\mathbb{Z}\to H$ defined by the rule $\varphi(a) = x^a$ is an isomorphism.
    \item Since $H$ is cyclic every element of $H$ is of the form $x^a$ for some $a$.  Show that $x^a$ generates $H$ if and only if $a=\pm1$.
  \end{enumerate}
  \item In this exercise we study products of finite cyclic groups.  Recall that we denote by $Z_n$ the cyclic group of order $n$ (written multiplicatively).
  \begin{enumerate}
    \item Prove that $Z_2\times Z_2$ is not a cyclic group.
    \item Prove that $Z_2\times Z_3\cong Z_6$.  Conclude that $Z_2\times Z_3$ is a cyclic group.
  \end{enumerate}
  Those two examples really cover all the bases.  Use the intuition you gained from them to prove the following classification result.
  \begin{enumerate}
    \setcounter{enumii}{2}
    \item Show that $Z_n\times Z_m$ is cyclic if and only if $\gcd(n,m)=1$.  (Hint: recall that up to isomorphism there is only one cyclic group of order $N$ for every positive integer $N$).
  \end{enumerate}
  \item Let $G = S_n$ be the symmetric group equipped with it's natural action on $\Omega_n = \{1,2,\cdots,n\}$ by permutations.  For $i\in\Omega_n$, let $G_i = \{\sigma\in G|\sigma(i)=i\}$ be the stabilizer of $i$.  What is $|G_i|$?
\end{enumerate}
\end{document}
