\documentclass[11pt]{article}
\usepackage[top = 1in, bottom = 1in, left =1in, right = 1in]{geometry}
\usepackage{graphicx}
\usepackage{amsmath}
\usepackage{amssymb}
\usepackage{etoolbox}
\usepackage{xcolor}
\usepackage{amsthm}
\usepackage{tikz-cd}
\usepackage{tikz}
\AtBeginEnvironment{proof}{\color{blue}}
\newtheorem{theorem}{Theorem}
\newtheorem{lemma}{Lemma}
\newtheorem{corollary}{Corollary}
\newtheorem{definition}{Definition}
\usepackage{enumerate}
\usepackage{hyperref}
\usepackage{fancyhdr}\pagestyle{fancy}
\newcommand{\la}{\langle}
\newcommand{\ra}{\rangle}
\newcommand{\tors}{\mathrm{tors}}
\newcommand{\ab}{\mathrm{ab}}
\newcommand{\Aut}{\operatorname{Aut}}
\newcommand{\im}{\operatorname{im}}
\newcommand{\lcm}{\operatorname{lcm}}

%Math blackboard:
\newcommand{\bC}{\mathbb{C}}
\newcommand{\bF}{\mathbb{F}}
\newcommand{\bN}{\mathbb{N}}
\newcommand{\bQ}{\mathbb{Q}}
\newcommand{\bR}{\mathbb{R}}
\newcommand{\bS}{\mathbb{S}}
\newcommand{\bZ}{\mathbb{Z}}

%Greek blackboard font:
\newcommand{\bmu}{\mbox{$\raisebox{-0.59ex}
  {$l$}\hspace{-0.18em}\mu\hspace{-0.88em}\raisebox{-0.98ex}{\scalebox{2}
  {$\color{white}.$}}\hspace{-0.416em}\raisebox{+0.88ex}
  {$\color{white}.$}\hspace{0.46em}$}{}}

\lhead{University of California, Berkeley}
\rhead{Math 113 Section 6, Spring 2020}

\begin{document}
\begin{center}
\Large {Homework Assignment 7}\\
\small {Due Friday, March 13}
\end{center}
There are two parts to this homework.  The first part outlines a proof of the Jordan-H\"older theorem, while the second introduces a new class of examples of finite groups.
\section{Jordan-H\"older}
Recall the following defintion from class.
\begin{definition}
  Let $G$ be a group.  A sequence of subgroups:
  \[1 = N_0\le N_1\le N_2\le\cdots\le N_{k-1}\le N_k = G,\]
  is called a \textit{composition series} if for each $i$ $N_i\unlhd N_{i+1}$ and the quotient $N_{i+1}/N_i$ is normal.
\end{definition}
The important point is that composition series exist, and are in some sense unique.
\begin{theorem}[Jordan-H\"older]
  Let $G$ be a finite group with $G\not=1$.  Then,
  \begin{enumerate}[(1)]
    \item $G$ has a composition series.
    \item The composition factors of the composition series are unique.  Specifically, this means that if
    \[1 = N_0\le N_1\le\cdots\le N_k = G,\]
    \[1 = M_0\le M_1\le\cdots\le M_s = G,\]
    are two composition series', then $s = k$ and there is a permutation $\pi$ of $\{1,\cdots,k\}$ such that
    \[M_{\pi(i)}/M_{\pi(i)-1}\cong N_i/N_{i-1},\]
    for each i.
  \end{enumerate}
\end{theorem}
\begin{enumerate}
  \item This first exercise proves the Jordan-H\"older theorem.
  \begin{enumerate}
    \item Prove part (1) of the Jordan-H\"older theorem by induction on $|G|$.
    \item Prove part (2) if the Jordan-H\"older theorem in the case that $s = 2$.  (Hint: Show if H,K are normal subgroups, then so is HK, then use the second isomorphism theorem with $M_1$ and $N_{k-1}$).
    \item Prove part (2) of the Jordan-H\"older theorem by induction on the minimum of $r$ and $s$.  (Apply the inductive hypothesis to $H = N_{r-1}\cap M_{s-1}$).
  \end{enumerate}
\end{enumerate}
\section{Matrix Groups}
The rest of the homework introduces a new family of finite groups.  So far we've only studies a few examples of finite groups: $D_{2n},S_n$ and direct products of cyclic groups.  As we start defining more exotic properties of groups we will need to expand our library of finite groups to exhibit some of these interesting properties.  In this homework we will introduce a new example: finite matrix groups.  We will need a definition.
\begin{definition}
  A \textit{field} is a set $F$ together with two commutative binary operations, $+$ and $\cdot$ (addition and multiplication), such that $(F,+)$ and $(F\setminus\{0\},\cdot)$ are abelian groups, and such that the distributive law holds.  That is, for all $a,b,c\in F$ we have:
  \[a\cdot(b+c) = a\cdot b + a\cdot c.\]
  For any field we let $F^\times = F\setminus\{0\}$ be its \textit{mutliplicative group}.  A field $F$ is called a finite field if $|F|<\infty$.
\end{definition}
It turns out that vector space theory over $F$ is pretty much identical to vector space theory over $R$.  We can define the first matrix group we hope to study.
\begin{definition}
  Let $F$ be a field.  If $M,N$ are matrices with entries in $F$, we can compute their product $MN$ and the determinant $\det(M)\in F$ using the same formulas as if $F=\bR$.  Then the \textit{general linear group of degree} $n$ over $F$ is,
  \[GL_n(F) = \{A\text{ }|\text{ } A\text{ is an }n\times n\text{ matrix with entries in }F\text{ and }\det(A)\not=0\}.\]
\end{definition}
\begin{enumerate}
  \setcounter{enumi}{1}
  \item It turns out that we have seen exampless of finite fields already.
  \begin{enumerate}
    \item Let $p$ be a prime number.  Show that $\bZ/p\bZ$ with the operations $+$ and $\times$ is a field.  This is the \textit{finite field of order} $p$ and will be denoted by $\bF_p$.
    \item Show that if $n$ is not prime, $\bZ/n\bZ$ is not a field.
  \end{enumerate}
  \item Let's study the simplest example of general linear groups: $GL_2(F)$.
  \begin{enumerate}
    \item Let $A,B\in GL_2(F)$.  Show that $\det(AB) = \det(A)\det(B)$.
    \item Show that $\det(A)=0$ if and only if one row is a multiple of the other.
    \item Show that $A^{-1} = \frac{1}{\det(A)}\tilde A$ where $\tilde A$ is defined by the rule:
    \[\widetilde{
    \begin{pmatrix}
      a & b\\
      c & d
    \end{pmatrix}} =
    \begin{pmatrix}
      d & -b\\
      -c & a
    \end{pmatrix}
    \]
    \item Conclude that $GL_2(F)$ is a group, and that $\det:GL_2(F)\to F^\times$ is a homomorphism.
    \item Prove that $GL_2(F)$ is isomorphic to the group $G$ of linear ismorphisms from $F^2\to F^2$.  (Hint, use matrix multiplication to get a map $GL_2(F)\to G$.)
  \end{enumerate}
  As you may have noticed, this proof went through the same way it does for $F=\bR$ in linear algebra.  The proofs can get more computationally intense for $GL_n(F)$ as $n$ increases, so for now lets take on faith that $GL_n(F)$ is a group and $\det:GL_n(F)\to F^\times$ is a homomorphism, and that $GL_n(F)$ parametrizes linear automorphisms of $F^n$.
  \item Now let's study $GL_2(\bF_p)$.
  \begin{enumerate}
    \item Prove that $|GL_2(\bF_2)| = 6$.
    \item Write all the elements of $GL_2(\bF_2)$ and compute the order of each element.
    \item Show that $GL_2(\bF_2)$ is not abelian.  (We will later see that it is isomorphic to $S_3$).
    \item Generalizing part (a), show that if $p$ is prime then
    \[|GL_2(\bF_p)| = p^4-p^3-p^2+p.\]
    Use exercise 3(b).
  \end{enumerate}
  \item The general linear group has lots of interesting subgroups and quotients.
  \begin{enumerate}
    \item Show that the constant diagonal matrices are a normal subgroup of $GL_n(F)$ isomorphic to $F^\times$
  \end{enumerate}
  We will often abuse notation and denote this by $F^\times\unlhd GL_n(F)$.  The quotient group $GL_n(F)/F^\times$ is called the \textit{projective general linear group} and denoted $PGL_n(F)$.
  \begin{enumerate}
    \setcounter{enumii}{1}
    \item The \textit{special linear group} $SL_n(F)$ is defined
    \[SL_n(F) = \{A\in GL_n(F)\text{ }|\text{ }\det(A) = 1.\}\]
    Show that $SL_n(F)$ is a normal subgroup of $GL_n(F)$.
    \item Prove the following isomorphism.
    \[GL_n(F)/SL_n(F)\cong F^\times.\]
    \item List all the elements of $SL_2(\bF_2)$
    \item Use problem 3(d) to compute $|SL_2(\bF_p)|$.
    \item Let $I$ be the identity matrix.  Show that $\{\pm I\}\le SL_n(F)$ if and only if $n$ is even.
    \item Use the second isomorphism theorem to construct an isomorphism:
    \[PGL_2(F)\cong SL_2(F)/\{\pm I\}.\]
  \end{enumerate}
\end{enumerate}
\end{document}
