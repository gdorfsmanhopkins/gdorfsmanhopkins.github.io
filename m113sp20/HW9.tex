\documentclass[11pt]{article}
\usepackage[top = 1in, bottom = 1in, left =1in, right = 1in]{geometry}
\usepackage{graphicx}
\usepackage{amsmath}
\usepackage{tabu}
\usepackage{amssymb}
\usepackage{etoolbox}
\usepackage{xcolor}
\usepackage{amsthm}
\usepackage{tikz-cd}
\usepackage{tikz}
\AtBeginEnvironment{proof}{\color{blue}}
\newtheorem{theorem}{Theorem}
\newtheorem{lemma}{Lemma}
\newtheorem*{remark}{Remark}
\newtheorem{corollary}{Corollary}
\newtheorem{definition}{Definition}
\usepackage{enumerate}
\usepackage{hyperref}
\usepackage{fancyhdr}\pagestyle{fancy}
\newcommand{\la}{\langle}
\newcommand{\ra}{\rangle}
\newcommand{\tors}{\mathrm{tors}}
\newcommand{\ab}{\mathrm{ab}}
\newcommand{\Aut}{\operatorname{Aut}}
\newcommand{\Inn}{\operatorname{Inn}}
\newcommand{\im}{\operatorname{im}}
\newcommand{\lcm}{\operatorname{lcm}}
\newcommand{\ch}{\operatorname{char}}

%Math blackboard:
\newcommand{\bC}{\mathbb{C}}
\newcommand{\bF}{\mathbb{F}}
\newcommand{\bN}{\mathbb{N}}
\newcommand{\bQ}{\mathbb{Q}}
\newcommand{\bR}{\mathbb{R}}
\newcommand{\bS}{\mathbb{S}}
\newcommand{\bZ}{\mathbb{Z}}

%Greek blackboard font:
\newcommand{\bmu}{\mbox{$\raisebox{-0.59ex}
  {$l$}\hspace{-0.18em}\mu\hspace{-0.88em}\raisebox{-0.98ex}{\scalebox{2}
  {$\color{white}.$}}\hspace{-0.416em}\raisebox{+0.88ex}
  {$\color{white}.$}\hspace{0.46em}$}{}}

\lhead{University of California, Berkeley}
\rhead{Math 113 Section 6, Spring 2020}

\begin{document}
\begin{center}
\Large {Homework 9}\\
\small {Due Monday, April 20th}
\end{center}
Recall the following important Lemma from the April 8th lecture.
\begin{lemma}\label{LemmaA}
  Let $G$ be a finite group, and $H\unlhd G$ a normal subgroup.  Let $P\le H$ be a Sylow $p$ subgroup of $H$.  If $P\unlhd H$ then $P\unlhd G$.
\end{lemma}
We noted in class that this feels like a normal Sylow subgroup is somehow \textit{strongly} normal, in such a way that we get transitivity of normal subgroups.  The following definition makes this precise.
\begin{definition}[Characteristic Subgroups]
  A subgroup $H\le G$ is called \textit{characteristic} in $G$ if for every automorphism $\varphi\in\Aut G$, we have $\varphi(H) = H$.  This is denoted by $H\ch G$.
\end{definition}
\begin{enumerate}
  \item{
  Let's prove some basic facts about characteristic subgroups and use them to prove Lemma \ref{LemmaA}.
  \begin{enumerate}
    \item{
    Show that characteristic subgroups are normal.  That is, if $H\ch G$ then $H\unlhd G$.
    }
    \item{
    Let $H\le G$ be the unique subgroup of $G$ of a given order.  Then $H\ch G$.
    }
    \item{
    Let $K\ch H$ and $H\unlhd G$, then $K\unlhd G$.  (This is the transitivity statement alluded to, and justifies the feeling that a characteristic subgroup is somehow \textit{strongly normal}).
    }
    \item{
    Let $G$ be a finite group and $P$ a Sylow $p$-subgroup of $G$.  Show that $P\unlhd G$ if and only if $P\ch G$.
    }
    \item{
    Put all this together to deduce Lemma \ref{LemmaA}.
    }
  \end{enumerate}
  }
  \item{
  Recall from HW7 exercise 5 the definition of the subgroup $SL_n(F)\le GL_n(F)$, which consists of matrices whose determinant is 1.  Let's use the tools we've developted to study $SL_2(\bF_3)$.
  \begin{enumerate}
    \item{
    Compute the order of $SL_2(\bF_3)$ (\textit{cf.} HW7 problem 5e).
    }
    \item{
    Show that the matrices:
    \[\begin{pmatrix}0 & -1\\1 & 0\end{pmatrix}\hspace{20pt}\begin{pmatrix}1&1\\1&-1\end{pmatrix}\]
    generate a subgroup $H\le SL_2(\bF_3)$ which is isomorphic to $Q_8$.
    }
    \item{
    Conclude (\textit{cf.} takehome 2 problem 3) that $SL_2(\bF_3)$ and $S_4$ are 2 nonisomorphic groups of the same order.  (We point out that this is in contrast to $GL_2(\bF_2)$ being isomorphic to $S_3$.)
    }
    \item{
    Compute the number of Sylow 3-subgroups of $SL_2(\bF_3)$.
    }
    \item{
    Show that the subgroup defined in part (b) is the unique Sylow 2-subgroup of $SL_2(\bF_3)$.  (Hint, use a counting argument together with part (d)).
    }
    \item{
    Show that $Z(SL_2(\bF_3)) = \{\pm I\}$ where $I$ is the identity matrix.  (You will need to use what you learned in parts (d) and (e) together with the computation of $Z(Q_8)$ from the takehome).
    }
    \item{
    Prove that $SL_2(\bF_3)/Z(SL_2(\bF_3))\cong A_4$.  (Hint: Use what we know about groups of order 12).
    }
  \end{enumerate}
  }
  \newpage
  \item{
  Next lets poke and prod $GL_2(\bF_p)$.
  \begin{enumerate}
    \item{
    Recall the order of $GL_2(\bF_p)$ from HW7 problem 4(d).  What is the maximal $p$ divisor of $|GL_2(\bF_p)|$?
    }
    \item{
    The subset of \textit{upper triangular matrices} of $GL_2(\bF_p)$ is:
    \[T = \left\{\begin{pmatrix}a & b\\0 & d\end{pmatrix}\in GL_2(\bF_p)\right\}.\]
    The subset of \textit{strictly upper triangular matrices} is:
    \[\overline T = \left\{\begin{pmatrix}1 & b\\0 & 1\end{pmatrix}\in GL_2(\bF_p)\right\}.\]
    Show that $T$ and $\overline T$ are subgroups of $GL_w(\bF_p)$.  We will see that they are not normal.
    }
    \item{
    Show that $\overline T$ is a Sylow $p$-subgroup of $GL_2(\bF_p)$ and of $T$.
    }
    \item{
    Show that $GL_2(\bF_p)$ has $p+1$ Sylow $p$-subgroups (Hint: you only need to exhibit one more than you already have).  Conclude that $\overline T$ is not normal in $GL_2(\bF_p)$.
    }
    \item{
    Show that $\overline T\unlhd T$.
    }
    \item{
    Conclude that $T$ is not normal in $GL_2(\bF_p)$.  (Hint: use Lemma 1).
    }
  \end{enumerate}
  }
\end{enumerate}
\color{red} Skip question 4.  We haven't talked about finite fields of order 4, so it was silly of me to assign.
\begin{enumerate}
  \setcounter{enumi}{3}
  \item{
  Let's study $SL_2(\bF_4)$.
  \begin{enumerate}
    \item{
    Compute the order of $SL_2(\bF_4)$.
    }
    \item{
    Give 2 subgroups of order 5 in $SL_2(\bF_4)$
    }
    \item{
    Conclude that $SL_2(\bF_4)$ is simple and isomorphic to $A_5$.
    }
  \end{enumerate}
  }
\color{black}
  \item{
  Next let's study the dihedral group.
  \begin{enumerate}
    \item{
    Let $P$ be a Sylow 2-subgroup of $D_{2n}$.  Show that $N_{D_{2n}}(P) = P$.
    }
    \item{
    Suppose that $2n = 2^ak$ for some odd $k$.  Show that the number of Sylow 2-subgroups is $k$.
    }
    \item{
    List all the Sylow 2-subgroups of $D_{2n}$ if $n$ is odd.
    }
    \item{
    Give an example of a Sylow 2-subgroup of $D_{12}$.
    }
  \end{enumerate}
  }
\end{enumerate}
\end{document}
