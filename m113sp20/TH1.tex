\documentclass[11pt]{article}
\usepackage[top = 1in, bottom = 1in, left =1in, right = 1in]{geometry}
\usepackage{graphicx}
\usepackage{amsmath}
\usepackage{amssymb}
\usepackage{etoolbox}
\usepackage{xcolor}
\usepackage{amsthm}
\AtBeginEnvironment{proof}{\color{blue}}
\newtheorem{lemma}{Lemma}
\newtheorem{theorem}{Theorem}
\newtheorem{definition}{Definition}
\usepackage{enumerate}
\usepackage{hyperref}
\usepackage{fancyhdr}\pagestyle{fancy}
\newcommand{\la}{\langle}
\newcommand{\ra}{\rangle}
\newcommand{\tors}{\mathrm{tors}}
\newcommand{\bZ}{\mathbb{Z}}

\lhead{University of California, Berkeley}
\rhead{Math 113 Section 6, Spring 2020}

\begin{document}
\begin{center}
\Large {Take Home Assignment 1}\\
\small {Due Monday, February 24}
\end{center}
In this assignment, we will prove an important result called \textit{Lagrange's Theorem}.  It goes as follows.
\begin{theorem}[Lagrange's Theorem]~\\
  If $G$ is a finite group and $H$ is a subgroup of $G$ then $|H|$ divides $|G|$.
\end{theorem}
With this result in hand, we will be able to deduce a celebrated result of Fermat, which is central to number theory.
\begin{theorem}[Fermat's Little Theorem]~\\
  Let $p$ be a prime number and $a$ an integer.  Then $a^p\equiv a\mod p$.
\end{theorem}
To do all this, we will need the following definition.
\begin{definition}~\\
  Let $H$ be a group acting on a set $A$ and fix $a\in A$.  The \textit{orbit} of $a$ under $H$ is the set
  \[H\cdot a = \{b\in A\text{ }|\text{ }b=h\cdot a\text{ for some }h\in H\}.\]
\end{definition}
Lets begin!
\begin{enumerate}
  \item Let $H$ be a group acting on a set $A$.
  \begin{enumerate}
    \item Show that the relation
    \begin{center}
      $a\sim b$ if and only if $a = h\cdot b$ for some $h\in H$
    \end{center}
    is an equivalence relation on the set $A$.
    \item Show that the equivalence classes of this equivalence relation are precisely the orbits of the elements of $A$ under the action of $H$.
    \item Conclude that the orbits of $A$ under the action of $H$ form a partition of $A$.
  \end{enumerate}
  \item Let $H$ be a subgroup of a group $G$, and let $H$ act on $G$ by left mulptilication.
  \begin{eqnarray*}
    H\times G &\to& G\\
    (h,g) &\mapsto& hg
  \end{eqnarray*}
  \begin{enumerate}
    \item Fix $x\in G$, and consider its orbit $H\cdot x$. Show that $H$ and $H\cdot x$ have the same cardinality.  (Hint: build a bijective map $H\to H\cdot x$).  Deduce that all the orbits of $G$ under the action of $H$ have the same cardinality.
    \item Now suppose further that $G$ is a finite group.  Use part (a) and the exercise 1 to deduce Lagrange's theorem.
  \end{enumerate}
  \item We can use Lagrange's theorem and what we know about cyclic groups to prove Fermat's little theorem.
  \begin{enumerate}
    \item Let $|G|=n<\infty$.  Fix some $x\in G$.  Use Lagrange's theorem to show that $x^n = 1$.
    \item Let $p$ be a prime number.  Compute the order of $(\bZ/p\bZ)^\times$.  Fully justify your answer.
    \item Combine parts (a) and (b) to prove Fermat's little theorem.
  \end{enumerate}
\end{enumerate}
\end{document}
