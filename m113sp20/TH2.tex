\documentclass[11pt]{article}
\usepackage[top = 1in, bottom = 1in, left =1in, right = 1in]{geometry}
\usepackage{graphicx}
\usepackage{amsmath}
\usepackage{amssymb}
\usepackage{etoolbox}
\usepackage{xcolor}
\usepackage{amsthm}
\usepackage{tikz-cd}
\usepackage{tikz}
\AtBeginEnvironment{proof}{\color{blue}}
\newtheorem{theorem}{Theorem}
\newtheorem{lemma}{Lemma}
\newtheorem{corollary}{Corollary}
\newtheorem{definition}{Definition}
\usepackage{enumerate}
\usepackage{hyperref}
\usepackage{fancyhdr}\pagestyle{fancy}
\newcommand{\la}{\langle}
\newcommand{\ra}{\rangle}
\newcommand{\tors}{\mathrm{tors}}
\newcommand{\ab}{\mathrm{ab}}
\newcommand{\Aut}{\operatorname{Aut}}
\newcommand{\Inn}{\operatorname{Inn}}
\newcommand{\im}{\operatorname{im}}
\newcommand{\lcm}{\operatorname{lcm}}

%Math blackboard:
\newcommand{\bC}{\mathbb{C}}
\newcommand{\bF}{\mathbb{F}}
\newcommand{\bN}{\mathbb{N}}
\newcommand{\bQ}{\mathbb{Q}}
\newcommand{\bR}{\mathbb{R}}
\newcommand{\bS}{\mathbb{S}}
\newcommand{\bZ}{\mathbb{Z}}

%Greek blackboard font:
\newcommand{\bmu}{\mbox{$\raisebox{-0.59ex}
  {$l$}\hspace{-0.18em}\mu\hspace{-0.88em}\raisebox{-0.98ex}{\scalebox{2}
  {$\color{white}.$}}\hspace{-0.416em}\raisebox{+0.88ex}
  {$\color{white}.$}\hspace{0.46em}$}{}}

\lhead{University of California, Berkeley}
\rhead{Math 113 Section 6, Spring 2020}

\begin{document}
\begin{center}
\Large {Take Home Assignment 2}\\
\small {Due Monday, March 23}
\end{center}
With everthing going on right now the Monday deadline is flexible.  That being said, if you are going to need extra time \textit{please let me know}, I will be granting extensions no questions asked but I need to know when to expect your assignment so that nothing falls between the cracks.  Good luck and stay safe.

In this set of problems we will study the quaternion group $Q_8$.  It is a nonabelian group with very interesting properties.
\begin{definition}
  The \textit{quaternion group of order 8}, denoted $Q_8$ is the group of the following 8 elements:
  \[Q_8 = \{\pm1,\pm i, \pm j, \pm k\}\]
  subject to the relations:
  \[i^2 = j^2 = k^2 = -1,\]
  \begin{eqnarray*}
    ij = k, & \hspace{20pt} & ji = -k,\\
    jk = i, & \hspace{20pt} & kj = -i,\\
    ki = j, & \hspace{20pt} & ik = -j.
  \end{eqnarray*}
\end{definition}
\begin{enumerate}
  \item Let's start with a few simple facts.  Much of this is worked out in the book.
  \begin{enumerate}
    \item Write the entire multiplication table for $Q_8$.
    \item Find a presentation for $Q_8$ with 2 generators and 3 relations.
    \item Prove that $Q_8$ is not isomorphic to $D_8$.
    \item Find all the subgroups of $Q_8$, and draw its lattice.  (Hint: there are 6 total subgroups).
    \item Prove that every subgroup of $Q_8$ is normal.
    \item Prove that every subgroup and quotient group of $Q_8$ is abelian (Hint: use the classification of groups of order 4 and 2, as well as Lagrange's theorem).
    \item Compute $Z(Q_8)$ and $Q_8/Z(Q_8)$ (Hint for the second part: you can do this by hand, but it might be slicker to apply Homework 6 problem 5(b)).
    \item Write a composition series for $Q_8$.
  \end{enumerate}
  \item Now let's follow the proof of Cayley's theorem to exhibit $Q_8$ as a subgroup of $S_8$.
  \begin{enumerate}
    \item Label $\{1,-1,i,-i,j,-j,k,-k\}$ as the numbers $\{1,2,\cdots,8\}$. Then the action of $Q_8$ on itself by left multiplication gives an injective map $Q_8\to S_8$.  Write the permutation representations for $-1$ and $i$ as elements $\sigma_{-1},\sigma_i\in S_8$, and verify that $\sigma_i^2 = \sigma_{-1}$.  (Using the multiplication table from question 1 will make this easier).
    \item Use the generators from question 1(b) to give two elements of $S_8$ which generate a subgroup $H\le S_8$ isomorphic to $Q_8$.
    \item Is $\sigma_i$ even or odd?
    \item $A_8\cap H$ is isomorphic to a subgroup of $Q_8$.  Which one?
  \end{enumerate}
  \item Cayley's theorem says that if $|G|=n$ then $G$ embeds at $S_n$.  But might not be the smallest symmetric group that $G$ embeds in.  For example, $D_8$ embeds in $S_4$ (thinking about symmetries of the square as permutations of the vertices).  Nevertheless, for $Q_8$ the symmetric group given by Cayley's theorem is the smallest.
  \begin{enumerate}
    \item Let $Q_8$ act an a set $A$ with $|A|\le 7$.  Let $a\in A$.  Show that the stabilizer of $a$,  $(Q_8)_a\le Q_8$ must contain the subgroup $\{\pm1\}$.
    \item Deduce that the kernel of the action of $Q_8$ on $A$ contains $\{\pm1\}$.
    \item Conclude that $Q_8$ cannot embed into $S_n$ for $n\le7$.
  \end{enumerate}
  \item Finally let's say a few things about the automorphism group of $Q_8$.
  \begin{enumerate}
    \item By counting possible places where the generators may go, show that $|\Aut(Q_8)|\le 24$.
    \item What is $\Inn(Q_8)$? (Hint: You already did this in question 1(g)!)
    \item Use parts (a) and (b) to conclude that $|\Aut(Q_8)|$ must be one of $\{4,8,12,16,20,24\}$.  (Note: it will turn out that it is 24, but the proof of this fact is more involved).
  \end{enumerate}
\end{enumerate}
\end{document}
