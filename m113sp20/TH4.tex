\documentclass[11pt]{article}
\usepackage[top = 1in, bottom = 1in, left =1in, right = 1in]{geometry}
\usepackage{graphicx}
\usepackage{amsmath}
\usepackage{tabu}
\usepackage{amssymb}
\usepackage{etoolbox}
\usepackage{xcolor}
\usepackage{amsthm}
\usepackage{tikz-cd}
\usepackage{tikz}
\AtBeginEnvironment{proof}{\color{blue}}
\newtheorem{theorem}{Theorem}
\newtheorem{proposition}{Proposition}
\newtheorem{lemma}{Lemma}
\newtheorem*{facts}{Fact}
\newtheorem*{remark}{Remark}
\newtheorem{corollary}{Corollary}
\newtheorem{definition}{Definition}
\usepackage{enumerate}
\usepackage{hyperref}
\usepackage{fancyhdr}\pagestyle{fancy}
\newcommand{\la}{\langle}
\newcommand{\ra}{\rangle}
\newcommand{\tors}{\mathrm{tors}}
\newcommand{\ab}{\mathrm{ab}}
\newcommand{\Aut}{\operatorname{Aut}}
\newcommand{\Inn}{\operatorname{Inn}}
\newcommand{\im}{\operatorname{im}}
\newcommand{\lcm}{\operatorname{lcm}}
\newcommand{\ch}{\operatorname{char}}

%Math blackboard:
\newcommand{\bC}{\mathbb{C}}
\newcommand{\bF}{\mathbb{F}}
\newcommand{\bN}{\mathbb{N}}
\newcommand{\bQ}{\mathbb{Q}}
\newcommand{\bR}{\mathbb{R}}
\newcommand{\bS}{\mathbb{S}}
\newcommand{\bZ}{\mathbb{Z}}

%Greek blackboard font:
\newcommand{\bmu}{\mbox{$\raisebox{-0.59ex}
  {$l$}\hspace{-0.18em}\mu\hspace{-0.88em}\raisebox{-0.98ex}{\scalebox{2}
  {$\color{white}.$}}\hspace{-0.416em}\raisebox{+0.88ex}
  {$\color{white}.$}\hspace{0.46em}$}{}}

\lhead{University of California, Berkeley}
\rhead{Math 113 Section 6, Spring 2020}

\begin{document}
\begin{center}
\Large {Takehome 4}\\
\small {Due Friday, May 15th}
\end{center}
First of all, thank you everybody for your patience and perseverance this spring.  It's been a difficult couple of months to say the least, and I must say I am impressed and grateful for all of you.

To finish off the course, we're going to do some classifications of groups of order $p^2q$ for $p,q$ prime.  \textbf{Important note:} when classifying groups you must show your work and \textbf{justify your steps}.  A lot of folks on HW 11 Problem 4 (classifying groups of order 20), just listed the groups with no justification.  As in HW11, this will not recieve credit.

You are welcome to use all course notes, but make sure everything is cited!  This includes the ``table of stuff," we generated while studying Sylow's theorems, which I think you will find especially useful.  This table is available in its entirety on the course website, as the second page of the \href{http://www.gabrieldorfsmanhopkins.com/m113sp20/April29.pdf}{April 29 lecture notes}.  \textbf{All work must be your own and outside resources are not allowed}.  So break out this ``table of stuff" and let's get started.  You will need the following three facts, which you may use freely without proving yourself.
\begin{facts}[Automorphisms of abelian groups of order $p$ and $p^2$]
  Let $p$ a prime number.  Then:
  \begin{itemize}
    \item{
    $\Aut Z_p\cong Z_{p-1}$
    }
    \item{
    $\Aut Z_{p^2}\cong Z_{p(p-1)}.$
    }
    \item{
    $\Aut\left(Z_p\times Z_p\right)\cong GL_2(\bF_p).$
    }
  \end{itemize}
\end{facts}
Good luck!
\begin{enumerate}
  \item{
  A lot of studying semidirect products comes down to enumerating and classifying homomorphisms.  So let's begin by doing that.  For the rest of this exercise we fix a group $G$.
  \begin{enumerate}
    \item{
    Show that giving a homomorphism $Z_n\to G$ is the same as selecting an element $g\in G$ with $|g|$ dividing $n$.  That is, give a bijection between the following sets:
    \[\left\{
    \begin{array}{c}
      \text{Homomorphisms}\\
      Z_n\to G
    \end{array}\right\}
    \Longleftrightarrow
    \left\{
    \begin{array}{c}
      \text{Elements }g\in G\\
      \text{where }|g|\text{ divide }n
    \end{array}
    \right\}
    \]
    }
    \item{
    If $p$ is prime show that giving a \textit{nontrivial} map $Z_p\to G$ is the same as choosing an element of order $p$ in $G$. (Note: the trivial map is the one that sends every element to the identity of $G$).
    }
    \item{
    Show that giving a homomorphism $Z_{n_1}\times\cdots\times Z_{n_r}\to G$ is the same as chosing elements $g_1,\cdots,g_r\in G$ such that all the $g_i$ commute with eachother and each $|g_i|$ divides $n_i$.
    }
    \item{
    Suppose $G$ is abelian and $p$ is prime.  Describe the set of homomorphisms $Z_p\times Z_p\to G$ as a subset of $G\times G$.
    }
  \end{enumerate}
  }
  \item{With this in hand let's do some general work.  Let $|G| = p^2q$ for $p\not=q$ prime numbers.  Let $P$ be a Sylow $p$-subgroup and $Q$ a Sylow $q$-subgroup.
  \begin{enumerate}
    \item{
    First suppose that $q>p$.
    \begin{enumerate}
      \item{
      Show if $|G|\not=12$ then $G\cong Q\rtimes P$
      }
      \item{
      Show that if $p\not|q-1$ then $G$ is abelian.  List all possible values of $G$.
      }
      \item{
      $p|q-1$.  Show that $G$ can be nonabelian.  (You may have to deal with the case wehre $|G|=12$ separately).
      }
    \end{enumerate}
    }
    \item{
    Now suppose $p>q$
    \begin{enumerate}
      \item{
      Show that $G\cong P\rtimes Q$.
      }
      \item{
      Suppose $q|p-1$  Show that $G$ can be nonabelian.
      }
      \item{
      Suppse $q\not|p-1$.  Show that there is a nonabelian group of order $p^2q$ if and only if $q|p+1$.
      }
    \end{enumerate}
    }
  \end{enumerate}
  }
  Now we have a general framework.  Let's do a couple of concrete examples.
  \item{
  You may find HW11 Problem 4 useful for this problem (note: you are welcome to refer to the \href{http://www.gabrieldorfsmanhopkins.com/m113sp20/HW11Solutions.pdf}{HW11 Solutions} on the course website).
  \begin{enumerate}
    \item{
    Last week we studied the case where $|G| = 20$.  Which of the cases (from Problem 2) does this fall into?  You can answer this like (a)(ii) or (b)(iii), but make sure to fully justify your answer.
    }
    \item{
    The case where $|G| = 28$ should be similar.  Classify all such groups being sure to fully justify your answer.  There are fewer than the case for $|G|=20$.  Explain exactly why this happens (where is/are the missing group(s)?).
    }
    \item{
    Give presentations (i.e., generators and relations) for each group of order 28.  As usual, fully justify your answer (just listing the presentations without explanation will not recieve credit).
    }
  \end{enumerate}
  }
  \item{
  Now let's classify all groups of order $75$.
  \begin{enumerate}
    \item{
    Which of the cases (from Problem 2) does this fall under?
    }
    \item{
    List all the abelian groups of order 75.
    }
    \item{
    Show that if a group of order 75 has a cyclic Sylow 5-subgroup then it is abelian.
    }
    \item{
    Show that there is a unique nonabelian group of order 75.  (Hint: show that $3$ is the maximal $3$-divisor of $|GL_2(\bF_5)|$.  Then use Sylow's theorems and HW11 Problem 2(c)).
    }
  \end{enumerate}
  }
\end{enumerate}
\end{document}
