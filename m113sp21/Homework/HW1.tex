\documentclass[11pt]{article}
\usepackage[top = 1in, bottom = 1in, left =1in, right = 1in]{geometry}
\usepackage{graphicx}
\usepackage{amsmath}
\usepackage{tabu}
\usepackage{amssymb}
\usepackage{etoolbox}
\usepackage{xcolor}
\usepackage{amsthm}
\usepackage{tikz-cd}
\usepackage{tikz}
\usepackage{seqsplit}
\usepackage{ulem}
\usepackage{tabularx}
\AtBeginEnvironment{proof}{\color{blue}}
\newtheorem{theorem}{Theorem}
\newtheorem{proposition}{Proposition}
\newtheorem{lemma}{Lemma}
\newtheorem*{facts}{Fact}
\newtheorem*{remark}{Remark}
\newtheorem{corollary}{Corollary}
\newtheorem{definition}{Definition}
\usepackage{enumerate}
\usepackage{hyperref}
\usepackage{fancyhdr}\pagestyle{fancy}
\newcommand{\la}{\langle}
\newcommand{\ra}{\rangle}
\newcommand{\tors}{\mathrm{tors}}
\newcommand{\ab}{\mathrm{ab}}
\newcommand{\Aut}{\operatorname{Aut}}
\newcommand{\Inn}{\operatorname{Inn}}
\newcommand{\im}{\operatorname{im}}
\newcommand{\lcm}{\operatorname{lcm}}
\newcommand{\ch}{\operatorname{char}}

%Math blackboard:
\newcommand{\bC}{\mathbb{C}}
\newcommand{\bF}{\mathbb{F}}
\newcommand{\bN}{\mathbb{N}}
\newcommand{\bQ}{\mathbb{Q}}
\newcommand{\bR}{\mathbb{R}}
\newcommand{\bS}{\mathbb{S}}
\newcommand{\bZ}{\mathbb{Z}}

%Math caligraphy
\newcommand{\cC}{\mathcal{C}}
\newcommand{\cK}{\mathcal{K}}
\newcommand{\cM}{\mathcal{M}}
\newcommand{\cO}{\mathcal{O}}

%Greek blackboard font:
\newcommand{\bmu}{\mbox{$\raisebox{-0.59ex}
  {$l$}\hspace{-0.18em}\mu\hspace{-0.88em}\raisebox{-0.98ex}{\scalebox{2}
  {$\color{white}.$}}\hspace{-0.416em}\raisebox{+0.88ex}
  {$\color{white}.$}\hspace{0.46em}$}{}}

\lhead{University of California, Berkeley}
\rhead{Math 113, Spring 2021}

\begin{document}
\begin{center}
\Large {Homework Assignment 1}\\
\small {Due: Friday, January 29}
\end{center}
\begin{enumerate}
\item{
  Let $S$ and $T$ be sets, and suppose that $T\subseteq S$.  Describe the following sets, proving the correctness of your answers.
  \begin{enumerate}
    \item{
    $T\cap S$.
    }
    \item{
    $T\cup S$.
    }
    \item{
    $T\cap(S\setminus T)$
    }
    \item{
    $T\cup(S\setminus T)$.
    }
  \end{enumerate}
}
\item Let $S$ be a set with 3 elements (say \{0,1,2\}) and $T$ be a set with 5 elements (say \{a,b,c,d,e\}).
\begin{enumerate}[(a)]
\item Give an example of an injection $f:S\to T$.
\item Give an example of a surjection $g:T\to S$.
\item Can there be a bijection between $S$ and $T$? Why or why not?
\end{enumerate}
\item A subset $T\subset S$ is called a \textit{proper subset} if $T\not= S$.  This is often denoted $T\subsetneq S$.  Give an example of a set $S$ and a bijection between $S$ and a \textit{proper} subset of $S$.
\item Let $S$ and $T$ be two sets, and $f:S\to T$ a function between them.
\begin{enumerate}
  \item{Show that $f$ is injective if and only if it has a left inverse.}
  \item{Show that $f$ is surjective if and only if it has a right inverse}
  \item{Show that $f$ is bijective if and only if it has an inverse.}
  \item{Show that if $f$ has a (two-sided) inverse, that inverse is unique.}
\end{enumerate}
\begin{remark}
  Because of part (c) and (d) of the question 4, we see that if $f$ is bijective, then $f$ has a unique inverse, which we call \textit{the inverse of $f$} and denote by $f^{-1}$.
\end{remark}
\item{
Let $S$ and $T$ be finite sets and suppose that $|S| = |T|$.  Let $f:S\to T$ be a function.  Prove that
\[f\text{ is injective }\Leftrightarrow f\text{ is surjective }\Leftrightarrow f\text{ is bijective.}\]
}
\item Show that equivalence relations are partitions are equivalent.  Explicitly, let $S$ be a set, construct a natural bijection between the partitions on $S$ and the equivalence relations on $S$ in the following way.
\begin{enumerate}[(a)]
\item Let $\sim$ be an equivalence relation.  Show that the equivalence classes of $\sim$ form a partition of $S$.
\item Conversely, let $\{X_i\}$ be a partition of $S$.  Show that the relation $\sim$ given by the rule
\[x\sim y\text{ if }x,y\in X_i\text{ for the same }i\]
is an equivalence relation for $S$.
\item{Show that parts (a) and (b) give a bijection between the sets:
\[\{\text{Equivalence relations on }S\}\longleftrightarrow\{\text{Partitions of }S\}.\]
(Hint: Part (a) gives a function from the left to the right.  Part (b) gives a function from the right to the left.  Show that these are inverses to eachother).
}
\end{enumerate}
\item{Let $a,b,c\in\bZ$.  Prove the following divisibility facts.
\begin{enumerate}
  \item{If $a|b$ and $a|c$ then $a|(b+c)$}
  \item{If $a|b$ then $a|bc$.}
\end{enumerate}
}
\item{
In this exercise we prove the existence and uniqueness of division with remainder.  Let $a,b\in\bZ$, and suppose that $b\not=0$.  We start with existence.
\begin{enumerate}
  \item{
  We begin by considering the set of numbers $a-bq$ as $q$ varies over the integers.  Prove that the set
  \[S = \{a-bq : q\in\bZ\},\]
  has at least one nonnegative element.
  }
  \item{
  Let $r$ be the minimal nonnegative element of $S$.  Show that $0\le r< |b|$.
  }
  \item{
  Use (b) to conclude that $a = bq+r$ for some $q,r\in\bZ$ with $0\le r<|b|$.  This proves existence.
  }
  \item{
  Show that the division with remainder from part (c) is unique.  That is, suppose there are $q_1,q_2,r_1,r_2\in\bZ$ such that
  \begin{eqnarray*}
    a = bq_1+r_1 &\text{and}&a= bq_2+r_2.
  \end{eqnarray*}
  Suppose further that $0\le r_i< |b|$ for $i=1,2$.  Then show $q_1=q_2$ and $r_1=r_2$.
  }
\end{enumerate}
}
\item{
In this exercise we prove the Euclidean algorithm works.
\begin{enumerate}
  \item{
  Suppose $a,b\in\bN$ are two positive integers, and let $a=bq+r$ for $0\le r<b$ (as in the previous exercise).  Show that:
  \[\gcd(a,b) = \gcd(b,r).\]
  }
  \item{
  Let $a\not=0$ be an integer.  What is $\gcd(a,0)$?  Justify your answer.
  }
  \item{
  Prove the correctness of the Euclidean algorithm.  That is, suppose $a,b\in\bN$ are two positive integers, and suppose you iterate the division algorithm as follows:
  \begin{eqnarray*}
    a &=& bq_0 + r_0\hspace{30pt}0\le r_0<b\\
    b &=& r_0q_1 + r_1\hspace{30pt}0\le r_1<r_0\\
    r_0&=& r_1q_2 + r_2\hspace{30pt}0\le r_2<r_1\\
    &\vdots&\\
    r_{n-2}&=& r_{n-1}q_n + r_n \hspace{30pt}0\le r_n<r_{n-1}\\
    r_{n-1} &=& r_nq_{n+1}.
  \end{eqnarray*}
  Show that $\gcd(a,b) = r_n$.
  }
\end{enumerate}
}
\item Let $d$ be the greatest common divisor of $792$ and $275$.  Using Euclid's algorithm, find $d$ and write $d=792x + 275y$ for some $x$ and $y$.
\end{enumerate}
\end{document}
