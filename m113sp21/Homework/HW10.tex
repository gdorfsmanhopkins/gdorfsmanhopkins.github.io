\documentclass[11pt]{article}
\usepackage[top = 1in, bottom = 1in, left =1in, right = 1in]{geometry}
\usepackage{graphicx}
\usepackage{amsmath}
\usepackage{tabu}
\usepackage{amssymb}
\usepackage{amsmath}
\usepackage{mathrsfs}
\usepackage{etoolbox}
\usepackage{xcolor}
\usepackage{amsthm}
\usepackage{tikz-cd}
\usepackage{tikz}
\usepackage{tkz-graph}
\usepackage{seqsplit}
\usepackage{ulem}
\usepackage{tabularx}
\AtBeginEnvironment{proof}{\color{blue}}
\newtheorem{theorem}{Theorem}
\newtheorem{proposition}[theorem]{Proposition}
\newtheorem{lemma}[theorem]{Lemma}
\newtheorem*{facts}{Facts}
\newtheorem*{remark}{Remark}
\newtheorem{corollary}[theorem]{Corollary}
\newtheorem{definition}[theorem]{Definition}
\newtheorem{defProp}[theorem]{Definition/Proposition}

\newtheorem*{hint}{Hint}
\usepackage{enumerate}
\usepackage{hyperref}
\usepackage{fancyhdr}\pagestyle{fancy}
\newcommand{\la}{\langle}
\newcommand{\ra}{\rangle}
\newcommand{\tors}{\mathrm{tors}}
\newcommand{\ab}{\mathrm{ab}}
\newcommand{\Aut}{\operatorname{Aut}}
\newcommand{\Inn}{\operatorname{Inn}}
\newcommand{\Out}{\operatorname{Out}}
\newcommand{\im}{\operatorname{im}}
\newcommand{\lcm}{\operatorname{lcm}}
\newcommand{\ch}{\operatorname{char}}
\newcommand{\maps}{\operatorname{Maps}}

%Math blackboard:
\newcommand{\bC}{\mathbb{C}}
\newcommand{\bF}{\mathbb{F}}
\newcommand{\bH}{\mathbb{H}}
\newcommand{\bN}{\mathbb{N}}
\newcommand{\bQ}{\mathbb{Q}}
\newcommand{\bR}{\mathbb{R}}
\newcommand{\bS}{\mathbb{S}}
\newcommand{\bZ}{\mathbb{Z}}

%Math caligraphy
\newcommand{\cA}{\mathcal{A}}
\newcommand{\cC}{\mathcal{C}}
\newcommand{\cK}{\mathcal{K}}
\newcommand{\cM}{\mathcal{M}}
\newcommand{\cO}{\mathcal{O}}

%Math scripts:
\newcommand{\sC}{\mathscr{C}}
\newcommand{\sP}{\mathscr{P}}

%Greek blackboard font:
\newcommand{\bmu}{\mbox{$\raisebox{-0.59ex}
  {$l$}\hspace{-0.18em}\mu\hspace{-0.88em}\raisebox{-0.98ex}{\scalebox{2}
  {$\color{white}.$}}\hspace{-0.416em}\raisebox{+0.88ex}
  {$\color{white}.$}\hspace{0.46em}$}{}}

\lhead{University of California, Berkeley}
\rhead{Math 113, Spring 2021}

\begin{document}
\begin{center}
  \Large {Homework Assignment 10}\\
  \small {Due Friday, April 16}
\end{center}
\begin{enumerate}
  \item{
  Let $R$ be a ring.  Recall that for $a\in R$ we denote the \textit{additive} inverse of $a$ by $-a$.  Establish the following identities.
  \begin{enumerate}
    \item{$(-a)b = a(-b) = -ab$}
    \item{$(-a)(-b) = ab$}
    \item{If $1\in R$ then $(-1)a = -a$.}
    \item{Suppose $R$ is an integral domain.  Show that if $a^2=1$ then $a=\pm1$.}
  \end{enumerate}
  }
  \item{
  Let $R$ be a ring with $1\not=0$.
  \begin{enumerate}
    \item{
    Let $R^\times\subseteq R$ be the set of units of $R$.  Show that $R^\times$ is a group under the multiplication operation of $R$.
    }
    \item{
    Suppose that $a\in R$ is a zero divisor.  Show that $a\notin R^\times$.
    }
    \item{
    Suppose $R$ is a subring of some ring $S$.  Show that if $a\in R^\times$ then $a\in S^\times$.  Give an example to show the converse is false.
    }
  \end{enumerate}
  }
  \item{
  Let $R$ be a commutative ring.  An element $r\in R$ is called \textit{nilpotent} if there exists a positive $n$ such that $r^n=0$.  A commutative ring is called \textit{reduced} if it has no nonzero nilpotent elements.
  \begin{enumerate}
    \item{
    Show that a nilpotent element of a ring is either 0 or a zero divisor.
    }
    \item{
    Give an example of a ring with a nonzero nilpotent element.
    }
    \item{
    Show that the sum of nilpotent elements is nilpotent.
    }
    \item{
    Suppose $r$ is nilpotent.  Show that $rx$ is nilpotent for all $x\in R$.  (\textit{Note}, in future terminology, (c) and (d) prove that the set of nilpotent elements is an \textit{ideal} of $R$, which we will call the \textit{nilradical}).
    }
    \item{
    Suppose $R$ is a commutative ring with $1\not=0$, and suppose $r\in R$ is nilpotent.  Show that $1+r\in R^\times$.
    }
  \end{enumerate}
  }
  \item{
  \begin{enumerate}
    \item{
    Let $\{S_i\subseteq R\}$ be a nonempty collection of subrings of $R$.  Show that $\bigcap_i S_i$ is a subring of $R$.
    }
    \item{
    Suppose $S$ is a subring of $R$, and $R$ is a subring of $T$.  Show that $S$ is a subring of $T$.
    }
  \end{enumerate}
  }
  \item{
  For a ring $R$, define the \textit{center} of $R$ to be:
  \[Z(R) = \{r\in R\text{ }|\text{ }ra = ar\text{ for all }a\in R\}.\]
  \begin{enumerate}
    \item{
    Show that $Z(R)$ is a subring of $R$.
    }
    \item{
    Suppose $R$ has $1\not=0$.  Show that $R^\times\cap Z(R)\subseteq Z(R^\times)$.  (The converse is \textit{not true} in general, but I don't consider this to be obvious.  Perhaps we will see an example later).
    }
    \item{
    Show that the center of a division ring is a field.
    }
    \item{
    Let $\bH$ be Hamilton's quaternions (defined in Lecture 21 or [DF] Example 5 on Page 224).  Compute $Z(\bH)$.  (Notice that $\bH$ contains a copy of $\bC$, is this the center?)
    }
  \end{enumerate}
  }
  \item{
  Let $R$ be ring, and $X$ any set.  Define
  \[\maps(X,R) = \{f:X\to R\text{ }|\text{ }f\text{ is a function}\}.\]
  Define binary operations $+$ and $\times$ as follows.
  \[(f+g)(x) = f(x) + g(x)\hspace{30pt}(f\times g)(x) = f(x)g(x).\]
  \begin{enumerate}
    \item{
    Show that $\maps(X,R)$ is a ring.
    }
    \item{
    Suppose $R$ is commutative, show that $\maps(X,R)$ is too.
    }
    \item{
    Suppose $R$ is unital, show that $\maps(X,R)$ is too.
    }
    \item{
    Suppose $R$ is reduced (defined in Problem 3), show that $\maps(X,R)$ is too.
    }
    \item{
    Give an example to show that even if $R$ is a field, $\maps(X,R)$ need not be.
    }
    \item{
    Give an example to show that even if $R$ is an integral domain, $\maps(X,R)$ need not be.
    }
  \end{enumerate}
  }
  \item{
  We now develop an example of rings that appear along the intersection of the algebraic and analytic theory (for example in \textit{functional analysis}).  You may use without proof the following facts from elementary calculus: \textbf{(1)} If $f,g$ are continuous so are their sum and product.  \textbf{(2)} If $f,g$ are differentiable then they are continuous and:
  \[(f+g)' = f'+g'\hspace{40pt}(fg)' = f'g+fg'\]
  \begin{enumerate}
    \item{
    Let $\sP$ be a property of maps from $X\to R$, and let
    \[\maps_\sP(X,R) = \{f:X\to R\text{ }|\text{ }f\text{ has property }\sP\}.\]
    Suppose that the 0 map has property $\sP$.  Suppose also that if $f$ and $g$ have property $\sP$, then so do $f-g$ and $f\times g$.  Show that $\maps_\sP(X,R)$ is a subring of $\maps(X,R)$.
    }
    \item{
    Let $X = R = \bR$.  Let $f:\bR\to\bR$ have property $\sC^0$ if $f$ is continuous, and define $C^0(\bR) = \maps_{\sC^0}(\bR,\bR)$ to be the set of continuous functions from $\bR$ to $\bR$.  Use part (a) to show that $C^0(\bR)$ is a subring of $\maps(\bR,\bR)$.
    }
    \item{
    For each $n>0$ let $f:\bR\to\bR$ have property $\sC^n$ if $f$ has a derivative everywhere, and $df/dx$ has propery $\sC^{n-1}$.  (So for example, $f$ is $\sC^1$ if it is differentiable and its derivative is continuous). Show by induction on $n$ that $C^n(\bR) = \maps_{\sC^n}(\bR,\bR)$ is a subring of $C^{n-1}(\bR)$.
    }
    \item{
    A funcion $f:\bR\to\bR$ is has property $\sC^\infty$ if for each positive $n$ the $n$'th derivative of $f$ exists and is continuous.  (Such a function is also often called \textit{smooth}).  Show that $C^\infty(\bR) = \maps_{\sC^{\infty}}(\bR,\bR)$ is a subring of $C^{n}(\bR)$ for each $n$.  (Hint: rather than prove this directly, you could use (4)).
    }
  \end{enumerate}
  }
  \item{
  Let $A$ be an abelian group (written additively).  Define the \textit{endomorphism ring} of $A$ as follows:
  \[\operatorname{End}(A) = \{f:A\to A\text{ }|\text{ }f\text{ is a homomorpism}\}.\]
  Give $\operatorname{End}(A)$ 2 binary operations $+$ and $\times$ as follows:
  \[(f+g)(a) = f(a)+g(a)\hspace{20pt}(f\times g)(a) = f(g(a)).\]
  \begin{enumerate}
    \item{
    Prove that $\operatorname{End}(A)$ is a ring.
    }
    \item{
    Prove that $(\operatorname{End}(A))^\times\cong\Aut(A)$.
    }
    \item{
    Let $E$ be an elementary abelian $p$-group of order $p^n$.  Show that $\operatorname{End}(E)\cong M_n(\bF_p)$ (You may use that $n\times n$ matrices over a field $F$ correspond to linear maps $F^n\to F^n$.  Compare to HW7 Problem 5).
    }
  \end{enumerate}
  }
\end{enumerate}
\end{document}
