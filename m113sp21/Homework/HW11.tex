\documentclass[11pt]{article}
\usepackage[top = 1in, bottom = 1in, left =1in, right = 1in]{geometry}
\usepackage{graphicx}
\usepackage{amsmath}
\usepackage{tabu}
\usepackage{amssymb}
\usepackage{amsmath}
\usepackage{mathrsfs}
\usepackage{etoolbox}
\usepackage{xcolor}
\usepackage{amsthm}
\usepackage{tikz-cd}
\usepackage{tikz}
\usepackage{tkz-graph}
\usepackage{seqsplit}
\usepackage{ulem}
\usepackage{tabularx}
\AtBeginEnvironment{proof}{\color{blue}}
\newtheorem{theorem}{Theorem}
\newtheorem{proposition}[theorem]{Proposition}
\newtheorem{lemma}[theorem]{Lemma}
\newtheorem*{facts}{Facts}
\newtheorem*{remark}{Remark}
\newtheorem{corollary}[theorem]{Corollary}
\newtheorem{definition}[theorem]{Definition}
\newtheorem{defProp}[theorem]{Definition/Proposition}

\newtheorem*{hint}{Hint}
\usepackage{enumerate}
\usepackage{hyperref}
\usepackage{fancyhdr}\pagestyle{fancy}
\newcommand{\la}{\langle}
\newcommand{\ra}{\rangle}
\newcommand{\tors}{\mathrm{tors}}
\newcommand{\ab}{\mathrm{ab}}
\newcommand{\Aut}{\operatorname{Aut}}
\newcommand{\Inn}{\operatorname{Inn}}
\newcommand{\Out}{\operatorname{Out}}
\newcommand{\im}{\operatorname{im}}
\newcommand{\lcm}{\operatorname{lcm}}
\newcommand{\ch}{\operatorname{char}}
\newcommand{\maps}{\operatorname{Maps}}

%Math blackboard:
\newcommand{\bC}{\mathbb{C}}
\newcommand{\bF}{\mathbb{F}}
\newcommand{\bH}{\mathbb{H}}
\newcommand{\bN}{\mathbb{N}}
\newcommand{\bQ}{\mathbb{Q}}
\newcommand{\bR}{\mathbb{R}}
\newcommand{\bS}{\mathbb{S}}
\newcommand{\bZ}{\mathbb{Z}}

%Math caligraphy
\newcommand{\cA}{\mathcal{A}}
\newcommand{\cC}{\mathcal{C}}
\newcommand{\cK}{\mathcal{K}}
\newcommand{\cM}{\mathcal{M}}
\newcommand{\cO}{\mathcal{O}}

%Math scripts:
\newcommand{\sC}{\mathscr{C}}
\newcommand{\sP}{\mathscr{P}}

%Mathfrak:
\newcommand{\fN}{\mathfrak{N}}

%Greek blackboard font:
\newcommand{\bmu}{\mbox{$\raisebox{-0.59ex}
  {$l$}\hspace{-0.18em}\mu\hspace{-0.88em}\raisebox{-0.98ex}{\scalebox{2}
  {$\color{white}.$}}\hspace{-0.416em}\raisebox{+0.88ex}
  {$\color{white}.$}\hspace{0.46em}$}{}}

\lhead{University of California, Berkeley}
\rhead{Math 113, Spring 2021}

\begin{document}
\begin{center}
  \Large {Homework Assignment 11}\\
  \small {Due Friday, April 23}
\end{center}
\begin{enumerate}
  \item{
  Let $R$ and $S$ be rings and $\varphi:R\to S$ a ring homomorphism.
  \begin{enumerate}
    \item{Show that $\im\varphi$ is a subring of $S$.}
    \item{Show that $\ker\varphi$ is a (two-sided) ideal of $R$.}
    \item{Suppose $J\subseteq S$ is an ideal.  Show that $\varphi^{-1}(J)$ is an ideal of $R$.}
    \item{Suppose $R$ and $S$ are unital rings with \textit{nonzero} identities $1_R$ and $1_S$ respectively.  Prove that if $\varphi(1_R)\not=1_S$ then $\varphi(1_R)$ is either zero, or a zero divisor in $S$.}
    \item{Deduce that if $S$ is an integral domain and $\varphi$ is nonzero then $\varphi(1_R)=1_S$.  (\textit{Remark:} many authors require rings to be unital, and also require ring homomorphisms to take the identity to the identity.)}
  \end{enumerate}
  }
  \item{
  In this exercise we prove the third and fourth isomorphism theorems for rings.
  \begin{enumerate}
    \item{We start with the fourth isomorphism theorem.  Let $R$ be a ring and $I\subseteq R$ an ideal.  In particular (since $R$ is abelian), $I$ is a normal subgroup.  Therefore, applying the fourth isomorphism theorem for groups (HW5 Problem 1), there is a bijection:
    \[\left\{
    \begin{array}{c}
      \text{Subgroups }A\le R\\
      \text{such that }I\le A
    \end{array}\right\}
    \Longleftrightarrow
    \left\{
    \begin{array}{c}
      \text{Subgroups}\\
      \overline{A}\le R/I
    \end{array}
    \right\}
    \]
    Prove the following ring theoretic enhancements hold:
    \begin{enumerate}
      \item{$A$ is a subring of $R$ if and only if $\overline A$ is a subring of $R/I$.}
      \item{If $A$ is a subring of $R$, then $I$ is an ideal of $A$ and that $A/I\cong\overline A$.}
      \item{$A$ is a left ideal of $R$ if and only if $\overline A$ is a left ideal of $R/I$.}
      \item{$A$ is a right ideal of $R$ if and only if $\overline A$ is a right ideal of $R/I$.}
      \item{$A$ is an ideal of $R$ if and only if $\overline A$ is an ideal of $R/I$.}
    \end{enumerate}
    }
    \item{We now prove the third isomorphism theorem for rings.  Let $J\subseteq I\subseteq R$, with $J,I$ ideals of a ring $R$.  By part (a) we know that $I/J$ is an ideal of $R/J$.  Prove that:
    \[\frac{R/J}{I/J}\cong\frac{R}{I}.\]
    }
    \item{We finish with a ring theoretic analog of \textit{passing to the quotient}.  Suppose $\varphi:R\to S$ is a ring map, and suppose that $I\subseteq\ker\varphi$.  Prove that there is a unique map $\overline\varphi:R/I\to S$ such that the following diagram commutes:
    \[
    \begin{tikzcd}
      R\ar[rr,"\varphi"]\ar[d,"\pi"]&&S\\
      R/I\ar[urr,dotted,swap,"\overline\varphi"]&&
    \end{tikzcd}
    \]
    That is, $\overline\varphi$ is the unique map so that $\overline\varphi\circ\pi=\varphi$.  (\textit{Hint}: We already know from group theory that there is a unique such map on the level of group homomorphisms.  What remains is to confirm that map is a ring homomorphism.)
    }
  \end{enumerate}
  }
  \item{
  Let $R$ be a ring.
  \begin{enumerate}
    \item{
    Suppose $\{I_j\}$ is a collection of left ideals of $R$.  Show that the intersection $\cap I_j$ is a left ideal of $R$.
    }
    \item{
    Show that part (a) also holds for right ideals and two-sided ideals.
    }
    \item{
    Let $R$ be a ring with $1\not=0$.  Show that:
    \[RA = \bigcap_{A\subset I\text{ left ideal}}I.\]
    }
    \item{
    State the analog for part (c) for right ideals.  (The proof will be identical, so I won't make you repeat yourself.)
    }
  \end{enumerate}
  }
  \item{
  Let $I$ and $J$ be ideals of a ring $R$.
  \begin{enumerate}
    \item{Prove that $I+J$ is the smallest ideal of $R$ containing both $I$ and $J$.}
    \item{Show that $IJ$ is an ideal contained in $I\cap J$}
    \item{Give an example where $IJ\not=I\cap J$}
    \item{Suppose $R$ is commutative and unital, and that $I+J=R$.  Show $IJ=I\cap J$.}
  \end{enumerate}
  }
  \item{
  Let $R$ be a commutative ring with $1\not=0$.
  \begin{enumerate}
    \item{
    Fix $a\in R$.  Show that $(a)=R$ if and only if $a\in R^\times$.
    }
    \item{
    Fix $a,b\in R$, and suppose that $a$ is not a zero divisor.  Show that $(a)=(b)$ if and only if $a = ub$ for some unit $u\in R^\times$.
    }
    \item{
    Let $I$ be any ideal.  Show that $I=R$ if and only if $I$ contains a unit $u\in R^\times$.
    }
    \item{
    Prove that $R$ is a field if and only if the only ideals in $R$ are $(0)$ and $R$ itself.
    }
    \item{
    Now suppose $S$ is a (not necessarily commutative) ring with $1\not=0$.  Show that $S$ is a division ring if and only if the only all left, right, and 2-sided ideals are one of $S$ or $(0)$.  (\textit{Hint}: Start by proving a version of part (c) for noncommutative rings.)
    }
  \end{enumerate}
  }
  \item{
  Let $R$ be any ring.  We define the \textit{$n$ by $n$ matrix ring} of $R$: $M_n(R)$, to be the set of $n$ by $n$ matrices whose entries are elements of $R$.  We often denote an element of $M$ as a $n^2$-tuple of entries indexed by $i$ and $j$ between 1 and $n$:
  \[M =
  \begin{pmatrix}
    a_{11} & a_{12} & \cdots & a_{1n}\\
    a_{21} & a_{22} & \cdots & a_{2n}\\
    \vdots & \vdots & \ddots & \vdots\\
    a_{n1} & a_{n2} & \cdots & a_{nn}
  \end{pmatrix}
  = (a_{ij}).
  \]
  We make $M_n(R)$ into a ring under usual matrix multiplication and addition.  That is, given $M = (a_{ij})$ and $N=(b_{ij})$ then $M+N=(a_{ij}+b_{ij})$, and the $ij$th entry of $MN$ is $\sum_{k=1}^na_{ik}b_{kj}$.
  \begin{enumerate}
    \item{
    Prove that $M_n(R)$ is a ring.
    }
    \item{
    Suppose $R$ is a ring with $1\not=0$, and that $n\ge2$.  Show that $M_n(R)$ always has a left ideal that is not a right ideal, and vice versa.
    }
    \item{
    Let $I$ be a left (respectively right) ideal of $R$.  Show that $M_n(I)$ is a left (respectively right) ideal of $M_n(R)$.
    }
    \item{
    Suppose $R$ is unital.  Show that the 2-sided ideals of $M_n(R)$ are precisely $M_n(J)$ for two sided ideals $J\subseteq R$.  (\textit{Hint}: Think about mutliplication by the matrices $E_{ij}$ which have a 1 in the $ij$ entry and are are 0 everywhere else).
    }
    \item{
    The determinant $\det:M_n(R)\to R$ is a function.  Is it always a ring homomorphism?  If yes, prove it.  If no, give a counterexample?
    }
  \end{enumerate}
  }
  \item{
  Recall that a group was called \textit{simple} if it had no normal subgroups, or equivalently, if it has no nontrivial quotients.  There is a similar notion for rings.  A ring $R$ is called \textit{simple} if the only quotients of $R$ are $R$ itself and the the zero ring.
  \begin{enumerate}
    \item{Give an equivalent formulation of simplicity in terms of ideals.}
    \item{Show that a commutative unital ring is simple if and only if it is a field.}
    \item{Give an example to show that a noncommutative unital ring may be simple even but not a division ring.}
  \end{enumerate}
  }
  \item{
  Let $R$ be a ring.  The \textit{nilradical} of $R$ is $\fN(R)=\{r\in R:r$ is nilpotent$\}$.  By HW10 Problem 3 we know that $\fN(R)$ is an ideal of $R$.
  \begin{enumerate}
    \item{Show that $R/\fN(R)$ is reduced.  This is often called the \textit{reduction of $R$,} and is denoted $R_{red}$.}
    \item{Let $\varphi:R\to S$ be any ring homomorphism.  Show that $\varphi(\fN(R))\subseteq\fN(S)$.  Deduce that if $S$ is reduced then $\fN(R)$ is contained in the kernel of $\varphi$.}
    \item{Let $S$ be a reduced ring.  Show that there is a bijection:
    \[\{\text{Ring homomorphisms }\varphi:R\to S\}\Longleftrightarrow\{\text{ Ring homomorphisms }\tilde\varphi:R_{red}\to S\}.\]
    \textit{Hint:} Use passing to the quotient!  \textit{Remark: }This should feel reminicient of the \textit{abelianization} from HW6 Problem 4.  In fact, both are examples of something more general, called a \textit{universal property.}  Keep your eyes open for things like this, they appear all over mathematics!}
  \end{enumerate}
  }
\end{enumerate}
\end{document}
