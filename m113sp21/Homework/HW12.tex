\documentclass[11pt]{article}
\usepackage[top = 1in, bottom = 1in, left =1in, right = 1in]{geometry}
\usepackage{graphicx}
\usepackage{amsmath}
\usepackage{tabu}
\usepackage{amssymb}
\usepackage{amsmath}
\usepackage{mathrsfs}
\usepackage{etoolbox}
\usepackage{xcolor}
\usepackage{amsthm}
\usepackage{tikz-cd}
\usepackage{tikz}
\usepackage{tkz-graph}
\usepackage{seqsplit}
\usepackage{ulem}
\usepackage{tabularx}
\AtBeginEnvironment{proof}{\color{blue}}
\newtheorem{theorem}{Theorem}
\newtheorem{proposition}[theorem]{Proposition}
\newtheorem{lemma}[theorem]{Lemma}
\newtheorem*{facts}{Facts}
\newtheorem*{remark}{Remark}
\newtheorem{corollary}[theorem]{Corollary}
\newtheorem{definition}[theorem]{Definition}
\newtheorem{defProp}[theorem]{Definition/Proposition}

\newtheorem*{hint}{Hint}
\usepackage{enumerate}
\usepackage{hyperref}
\usepackage{fancyhdr}\pagestyle{fancy}
\newcommand{\la}{\langle}
\newcommand{\ra}{\rangle}
\newcommand{\tors}{\mathrm{tors}}
\newcommand{\ab}{\mathrm{ab}}
\newcommand{\Aut}{\operatorname{Aut}}
\newcommand{\Inn}{\operatorname{Inn}}
\newcommand{\Out}{\operatorname{Out}}
\newcommand{\im}{\operatorname{im}}
\newcommand{\lcm}{\operatorname{lcm}}
\newcommand{\ch}{\operatorname{char}}
\newcommand{\maps}{\operatorname{Maps}}

%Math blackboard:
\newcommand{\bC}{\mathbb{C}}
\newcommand{\bF}{\mathbb{F}}
\newcommand{\bH}{\mathbb{H}}
\newcommand{\bN}{\mathbb{N}}
\newcommand{\bQ}{\mathbb{Q}}
\newcommand{\bR}{\mathbb{R}}
\newcommand{\bS}{\mathbb{S}}
\newcommand{\bZ}{\mathbb{Z}}

%Math caligraphy
\newcommand{\cA}{\mathcal{A}}
\newcommand{\cC}{\mathcal{C}}
\newcommand{\cK}{\mathcal{K}}
\newcommand{\cM}{\mathcal{M}}
\newcommand{\cO}{\mathcal{O}}

%Math scripts:
\newcommand{\sC}{\mathscr{C}}
\newcommand{\sP}{\mathscr{P}}

%Mathfrak:
\newcommand{\fJ}{\mathfrak{J}}
\newcommand{\fN}{\mathfrak{N}}
\newcommand{\fm}{\mathfrak{m}}
\newcommand{\fp}{\mathfrak{p}}
\newcommand{\fq}{\mathfrak{q}}

%Greek blackboard font:
\newcommand{\bmu}{\mbox{$\raisebox{-0.59ex}
  {$l$}\hspace{-0.18em}\mu\hspace{-0.88em}\raisebox{-0.98ex}{\scalebox{2}
  {$\color{white}.$}}\hspace{-0.416em}\raisebox{+0.88ex}
  {$\color{white}.$}\hspace{0.46em}$}{}}

\lhead{University of California, Berkeley}
\rhead{Math 113, Spring 2021}

\begin{document}
\begin{center}
  \Large {Homework Assigment 12}\\
  \small {Due Friday, April 30}
\end{center}
This is a shorter assignment.  We will use Sun Tzu's theorem to study the Euler totient function, and we will see an example where the nilradical and Jacobson radical are distinct.  First we will need a definition.
\begin{definition}
  Let $n\in\bN$ be a natural number.  Then \textit{Euler's totient function} of $n$ is:
  \[\varphi(n):=\#\{1\le a\le n : \gcd(a,n)=1.\}\]
  This is often also called \textit{Euler's $\varphi$ function}.
\end{definition}
\begin{enumerate}
  \item{
  In this problem $\varphi$ denotes Euler's totient function.
  \begin{enumerate}
    \item{
    Let $R,S$ be two unital rings.  Show that $(R\times S)^\times\cong R^\times\times S^\times$.
    }
    \item{
    Let $\varphi$ be Euler's totient function, and $n\in\bN$.  Explain why $\varphi(n) = |(\bZ/n\bZ)^\times|$.
    }
    \item{
    Let $m,n$ be comprime natural numbers.  Use Sun-Tzu's theorem as well as part (a) and (b) to prove that $\varphi(mn) = \varphi(m)\varphi(n)$.
    }
    \item{
    Let $p$ be a prime number and $j$ a positive integer.  Give a formula for $\varphi(p^j)$, and fully justify your answer.
    }
    \item{
    Use parts (c) and (d) to establish the following general formula for $\varphi$:
    \[\varphi(N) = N\cdot\left(\prod_{\substack{\text{primes }p\\ \text{with }p|N}}\left(1-\frac{1}{p}\right)\right).\]
    }
  \end{enumerate}
  }
\end{enumerate}
In Takehome 3 we introduced two interesting ideals of a commutative unital ring: the Jacobson radical and the nilradical.  These were clearly related, and are often the same, but sometimes they are different.  Let's investigate!
\begin{enumerate}
  \setcounter{enumi}{1}
  \item{
  Prove that $\fJ(R) = \fN(R)$ in each of the following cases.
  \begin{enumerate}
    \item{$R = \bZ$}
    \item{$R = K[x]$ where $K$ is any field.}
    \item{$R = \bZ/n\bZ$.}
  \end{enumerate}
  }
\end{enumerate}
To find a ring where they differ, we will use the following definition:
\begin{definition}
  A commutative unital ring is called a \textit{local ring} if it has a unique maximal ideal.
\end{definition}
\begin{remark}
  This terminology is related to the analogy between points of $X$ and maximal ideals of $\maps(X,\bR)$, so that only having one maximal ideal is related to having only one point (which is certainly ``local").
\end{remark}
\begin{enumerate}
  \setcounter{enumi}{2}
  \item{
  Let's study some properties of local rings.  Throughout, $R$ will denote a commutative unital ring.
  \begin{enumerate}
    \item{
    Let $R$ be a local ring.  Show that every element not contained in the maximal ideal is a unit.
    }
    \item{
    Conversely, show that if the set $\fm = R\setminus R^\times$ of nonunits of $R$ form an ideal, then $R$ is a local ring with maximal ideal $\fm$.
    }
    \item{
    Let $p$ be prime and $j$ a positive integer.  Prove that $\bZ/p^j\bZ$ is a local ring.  What is the maximal ideal?
    }
    \item{
    Let $R$ be a local ring which is an integral domain and not a field.  Prove that $\fJ(R)\not=\fN(R)$.
    }
  \end{enumerate}
  }
\end{enumerate}
Therefore, to construct an example where $\fJ(R)\not=\fN(R)$, we must construct a ring satisfying 3(d).
\begin{definition}
  Let $R$ be a commutative unital ring.  The \textit{ring of formal power series} $R[[x]]$ is the set of power series:
  \[\left\{\sum_{i=0}^\infty a_ix^i\text{ such that }a_i\in R\right\}.\]
  The binary operations are:
  \[\left(\sum_{i=0}^\infty a_ix^i\right) + \left(\sum_{i=0}^\infty b_ix^i\right) = \sum_{i=0}^\infty (a_i+b_i)x^i.\]
  \[\left(\sum_{i=0}^\infty a_ix^i\right) \times \left(\sum_{i=0}^\infty b_ix^i\right) = \sum_{i=0}^\infty\left(\sum_{k=0}^ia_kb_{i-k}\right)x^i.\]
\end{definition}
\begin{enumerate}
  \setcounter{enumi}{3}
  \item{
  Let $R$ be a commutative unital ring.
  \begin{enumerate}
    \item{
    Prove that $R[[x]]$ is a commutative unital ring.
    }
    \item{
    If $R$ is an integral domain, prove that $R[[x]]$ is.
    }
    \item{
    Prove that $1-x$ is a unit in $R[[x]]$.  (Hint: remember the geometric series?).
    }
    \item{
    Prove that $\sum a_ix^i$ is a unit in $R[[x]]$ if and only if $a_0$ is.
    }
    \item{
    Let $K$ be a field.  Prove that $K[[x]]$ is a local ring with maximal ideal $(x)$.  Conclude that $\fN(K[[x]])\not=\fJ(K[[x]])$.
    }
  \end{enumerate}
  }
\end{enumerate}
\end{document}
