\documentclass[11pt]{article}
\usepackage[top = 1in, bottom = 1in, left =1in, right = 1in]{geometry}
\usepackage{graphicx}
\usepackage{amsmath}
\usepackage{tabu}
\usepackage{amssymb}
\usepackage{amsmath}
\usepackage{mathrsfs}
\usepackage{etoolbox}
\usepackage{xcolor}
\usepackage{amsthm}
\usepackage{tikz-cd}
\usepackage{tikz}
\usepackage{tkz-graph}
\usepackage{seqsplit}
\usepackage{ulem}
\usepackage{tabularx}
\AtBeginEnvironment{proof}{\color{blue}}
\newtheorem{theorem}{Theorem}
\newtheorem{proposition}[theorem]{Proposition}
\newtheorem{lemma}[theorem]{Lemma}
\newtheorem*{facts}{Facts}
\newtheorem*{remark}{Remark}
\newtheorem{corollary}[theorem]{Corollary}
\newtheorem{definition}[theorem]{Definition}
\newtheorem{defProp}[theorem]{Definition/Proposition}

\newtheorem*{hint}{Hint}
\usepackage{enumerate}
\usepackage{hyperref}
\usepackage{fancyhdr}\pagestyle{fancy}
\newcommand{\la}{\langle}
\newcommand{\ra}{\rangle}
\newcommand{\tors}{\mathrm{tors}}
\newcommand{\ab}{\mathrm{ab}}
\newcommand{\Aut}{\operatorname{Aut}}
\newcommand{\Inn}{\operatorname{Inn}}
\newcommand{\Out}{\operatorname{Out}}
\newcommand{\im}{\operatorname{im}}
\newcommand{\lcm}{\operatorname{lcm}}
\newcommand{\ch}{\operatorname{char}}
\newcommand{\maps}{\operatorname{Maps}}

%Math blackboard:
\newcommand{\bC}{\mathbb{C}}
\newcommand{\bF}{\mathbb{F}}
\newcommand{\bH}{\mathbb{H}}
\newcommand{\bN}{\mathbb{N}}
\newcommand{\bQ}{\mathbb{Q}}
\newcommand{\bR}{\mathbb{R}}
\newcommand{\bS}{\mathbb{S}}
\newcommand{\bZ}{\mathbb{Z}}

%Math caligraphy
\newcommand{\cA}{\mathcal{A}}
\newcommand{\cC}{\mathcal{C}}
\newcommand{\cK}{\mathcal{K}}
\newcommand{\cM}{\mathcal{M}}
\newcommand{\cO}{\mathcal{O}}

%Math scripts:
\newcommand{\sC}{\mathscr{C}}
\newcommand{\sP}{\mathscr{P}}

%Mathfrak:
\newcommand{\fJ}{\mathfrak{J}}
\newcommand{\fN}{\mathfrak{N}}
\newcommand{\fm}{\mathfrak{m}}
\newcommand{\fp}{\mathfrak{p}}
\newcommand{\fq}{\mathfrak{q}}

%Greek blackboard font:
\newcommand{\bmu}{\mbox{$\raisebox{-0.59ex}
  {$l$}\hspace{-0.18em}\mu\hspace{-0.88em}\raisebox{-0.98ex}{\scalebox{2}
  {$\color{white}.$}}\hspace{-0.416em}\raisebox{+0.88ex}
  {$\color{white}.$}\hspace{0.46em}$}{}}

\lhead{University of California, Berkeley}
\rhead{Math 113, Spring 2021}

\begin{document}
\begin{center}
  \Large {Homework Assigment 13}\\
  \small {Due Friday, May 7}
\end{center}
\begin{enumerate}
  \item{
  Let $R$ be a unique factorization domain.
  \begin{enumerate}
    \item{
    Fix $r\in R$.  Show that $r$ is irreducible if and only if it is prime.
    }
    \item{
    Let $a,b\in R$.  Show that a greatest common denominator of $a$ and $b$ exists, and is unique up to multiplication by a unit.
    }
  \end{enumerate}
  }
  \item{
  Let's turn our attention to $\bZ[\sqrt{-5}]$.
  \begin{enumerate}
    \item{
    Show that $3$ is an irreducible element but not a prime element of $\bZ[\sqrt{-5}]$.
    }
    \item{
    Deduce from part (a) that $\bZ[\sqrt{-5}]$ is not a unique factorization domain.  Explain why this means $\bZ[\sqrt{-5}]$ is not a principal ideal domain.
    }
  \end{enumerate}
  We now know abstractly that $\bZ[\sqrt{-5}]$ is not a principal ideal domain.  Let's exhibit an explicit nonprincipal ideal.
  \begin{enumerate}
    \setcounter{enumii}{2}
    \item{
    Let $\fp\subseteq\bZ[\sqrt{-5}]$ be any prime ideal containing 3.  Prove that $\fp$ cannot be principal.
    }
    \item{
    Prove that the ideal $I = (3,2+\sqrt{-5})$ is a maximal ideal of $\bZ[\sqrt{-5}]$ containing 3.  Conclude that it cannot be principal.  (\textit{Hint:} Show $\bZ[\sqrt{-5}]/(3)$ has 9 elements and $I/(3)$ has 3 elements.  Then leverage the third isomorphism theorem for rings to compute $\bZ[\sqrt{-5}]/I$.)
    }
  \end{enumerate}
  }
  \item{
  Let $R$ be a Euclidean domain, and $N:R\to\bZ_{\ge0}$ a Euclidean norm.  Let's explore how the norm can help us characterize the units in $R$.
  \begin{enumerate}
    \item{
    Let $m = \min\{N(x):x\not=0\}$.  Show that if $N(x)=m$, then $x\in R^\times$.
    }
    \item{
    Let $\hat N:R\to\bZ$ be given by the following rule.
    \[\hat N(r) = \min_{x\in R\setminus\{0\}} N(xr).\]
    Prove that $\hat N$ is a Euclidean norm on $R$, and also that it satisfies the further condition that if $a|b$ and $b\not=0$, then $\hat N(a)\le\hat N(b)$.
    }
    \item{
    Prove that $x\in R^\times$ if and only if $\hat N(x)=\hat N(1)$.
    }
  \end{enumerate}
  }
  \item{
  Let $R$ be a principal ideal domain.
  \begin{enumerate}
    \item{
    Show that if $\fp$ is a prime ideal, then $R/\fp$ is also a principal ideal domain.
    }
    \item{
    Show that if $S$ is a multiplicative subset not containing $0$, then $S^{-1}R$ is a principal ideal domain.
    }
  \end{enumerate}
  }
  \item{
  Let $p$ a prime number so that $p\equiv3\mod 4$.
  \begin{enumerate}
    \item{
    Prove that $p$ generates a maximal ideal of $\bZ[i]$.
    }
    \item{
    Show that $\bZ[i]/(p)$ is a field with $p^2$ elements.  Denote it by $\bF_{p^2}$.
    }
    \item{
    Explain why $\bF_{p^2}\not\cong\bZ/p^2\bZ$.
    }
    \item{
    Prove that there is an injective homomorphism $\bF_p\hookrightarrow\bF_{p^2}$.
    }
  \end{enumerate}
  }
\end{enumerate}
\end{document}
