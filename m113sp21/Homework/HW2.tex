\documentclass[11pt]{article}
\usepackage[top = 1in, bottom = 1in, left =1in, right = 1in]{geometry}
\usepackage{graphicx}
\usepackage{amsmath}
\usepackage{tabu}
\usepackage{amssymb}
\usepackage{etoolbox}
\usepackage{xcolor}
\usepackage{amsthm}
\usepackage{tikz-cd}
\usepackage{tikz}
\usepackage{tkz-graph}
\usepackage{seqsplit}
\usepackage{ulem}
\usepackage{tabularx}
\AtBeginEnvironment{proof}{\color{blue}}
\newtheorem{theorem}{Theorem}
\newtheorem{proposition}{Proposition}
\newtheorem{lemma}{Lemma}
\newtheorem*{facts}{Fact}
\newtheorem*{remark}{Remark}
\newtheorem{corollary}{Corollary}
\newtheorem{definition}{Definition}
\usepackage{enumerate}
\usepackage{hyperref}
\usepackage{fancyhdr}\pagestyle{fancy}
\newcommand{\la}{\langle}
\newcommand{\ra}{\rangle}
\newcommand{\tors}{\mathrm{tors}}
\newcommand{\ab}{\mathrm{ab}}
\newcommand{\Aut}{\operatorname{Aut}}
\newcommand{\Inn}{\operatorname{Inn}}
\newcommand{\im}{\operatorname{im}}
\newcommand{\lcm}{\operatorname{lcm}}
\newcommand{\ch}{\operatorname{char}}

%Math blackboard:
\newcommand{\bC}{\mathbb{C}}
\newcommand{\bF}{\mathbb{F}}
\newcommand{\bN}{\mathbb{N}}
\newcommand{\bQ}{\mathbb{Q}}
\newcommand{\bR}{\mathbb{R}}
\newcommand{\bS}{\mathbb{S}}
\newcommand{\bZ}{\mathbb{Z}}

%Math caligraphy
\newcommand{\cC}{\mathcal{C}}
\newcommand{\cK}{\mathcal{K}}
\newcommand{\cM}{\mathcal{M}}
\newcommand{\cO}{\mathcal{O}}

%Greek blackboard font:
\newcommand{\bmu}{\mbox{$\raisebox{-0.59ex}
  {$l$}\hspace{-0.18em}\mu\hspace{-0.88em}\raisebox{-0.98ex}{\scalebox{2}
  {$\color{white}.$}}\hspace{-0.416em}\raisebox{+0.88ex}
  {$\color{white}.$}\hspace{0.46em}$}{}}

\lhead{University of California, Berkeley}
\rhead{Math 113, Spring 2021}

\begin{document}
\begin{center}
	\Large {Homework Assignment 2}\\
	\small {Due: Friday, February 5}
\end{center}
\begin{enumerate}
    \item{
    Let $m\in\bN$ be a natural number.  Recall that the \textit{residue of an integer $x$ modulo $m$} is the remainder $r$ when applying the division algorithm (HW1 \#8) to divide $x$ by $m$.  We say that integers $x$ and $y$ are \textit{congruent modulo $m$} if they have the same residue modulo $m$.
    \begin{enumerate}
  	\item{
      Show that $x$ and $y$ have the same residue modulo $m$ if and only if $m$ divides $x-y$.
    	}
      \item{
      Show that congruence modulo $m$ is an equivalence relation on $\bZ$.
  	  }
      \item{
      Suppose $a\equiv a'\mod m$ and $b\equiv b'\mod m$.  Show that:
      \[a+b\equiv a'+b'\mod m\hspace{40pt}\text{and}\hspace{40pt}ab\equiv a'b'\mod m.\]
      }
    \end{enumerate}
    }
  \item{
    \begin{enumerate}
      \item{
      Let $p$ be a prime number, and let $x,y\in\bZ/p\bZ$ be nonzero.  Show that $xy$ is also nonzero.
      }
      \item{
      On the other hand, let $m$ be a composite number greater than 3.  Show that one can always find two nonzero elements of $\bZ/m\bZ$ whose product is zero.
      }
    \end{enumerate}
  }
  \item{
  Fix a natural number $m$.
  \begin{enumerate}
  	\item{
    Let $x,y\in(\bZ/m\bZ)^\times$.  Show that $xy\in(\bZ/m\bZ)^\times$.
    }
  	\item{
    Show that $(\bZ/m\bZ)^\times$ is a group under multiplication modulo $m$.
    }
  	\item{
    Compute the order of each element of $(\bZ/7\bZ)^\times$
    }
  \end{enumerate}
  }
	\item Let $*$ denote multiplication modulo 15, and consider the set $\{3,6,9,12\}$.  Fill in the following multiplication table.
	      \begin{center}
		      \begin{tabular}{c|c|c|c|c}
			      *  & 3 & 6 & 9 & 12 \\
			      \hline
			      3  &   &   &   &    \\
			      \hline
			      6  &   &   &   &    \\
			      \hline
			      9  &   &   &   &    \\
			      \hline
			      12 &   &   &   &
		      \end{tabular}
	      \end{center}
	      Use the table to prove that $\left(\{3,6,9,12\},*\right)$ is a group.  What is the identity element?
	\item Let $A$ be a nonempty set, and define $S_A:=\{f:A\to A$ $|$ $f$ is bijective$\}$.  Define a binary operation  on $S_A$ using composition of functions.   Explicitly, for any $f,g\in S_A$ we define their product as follows: $f*g := f\circ g$.  Show that $S_A$ is a group.  We will call this the \textit{permutation group of }$A$.
  \item{
  Let $(A,*)$ and $(B,\boldsymbol{\cdot})$ be two groups.  Define multiplication on the Cartesian product $A\times B$ via the following rule:
  \[(a_1,b_1)(a_2,b_2) = (a_1*a_2,b_1\boldsymbol{\cdot}b_2).\]
  Show that this makes $A\times B$ into a group.  We call this group the \textit{direct product of $A$ and $B$}.
  }
	\item{
  Fix elements $x,y$ of a group $G$.
  \begin{enumerate}
    \item{
    Show that if $xy=e$ then $x^{-1}=y$ and $y^{-1}=x$.
    }
    \item{
    Show that $(xy)^{-1} = y^{-1}x^{-1}$.
    }
    \item{
    Show that $(x^n)^{-1} = x^{-n}$.
    }
  \end{enumerate}
  }
	\item Fix an element $x$ of a group $G$ and suppose $|x| = n$.
  \begin{enumerate}[(a)]
    \item Show that $x^{-1}$ is a nonnegative power of $x$.
		\item Show that the all of $1,x,x^2,\cdots,x^{n-1}$ are distinct.  Conclude that $|x|\le|G|$.  (We will later show that if $|G|$ is finite then $|x|$ \textit{divides} $|G|$.)
		\item Show that $x^i=x^j$ if and only if $i\equiv j\mod n$.
	\end{enumerate}
	\item{In class we developed the theory of the group $D_{12}$ of rigid symmetries of the regular hexagon.  In fact, everything we developed should go through almost exactly the same way for $D_{2n}$: the rigid symmetries of regular $n$-sided polygon, pictured below:
	\begin{center}
	   \begin{tikzpicture}
       \fill (-1.56366296494,1.24697960372) circle (0.1) node[anchor=south east] {\small $n$};
       \fill (0,2) circle (0.1) node[anchor = south west] {1};
       \fill (1.56366296494,1.24697960372) circle (0.1) node[anchor = south west] {2};
			 \fill (1.94985582436,-0.44504186791) circle (0.1) node [anchor = south west] {3};
			 \draw (-1.56366296494,1.24697960372) -- (0,2) --  (1.56366296494,1.24697960372) -- (1.94985582436,-0.44504186791);
			 \draw[dashed] (-1.94985582436,-0.44504186791) -- (-1.56366296494,1.24697960372);
			 \draw[dashed] (1.94985582436,-0.44504186791) -- (0.86776747823,-1.8019377358);
     \end{tikzpicture}
   \end{center}
   }
   \begin{enumerate}
     \item{
     Explain why $D_{2n}$ is a group under composition of symmetries.
     }
     \item{
     Show that there are exactly $2n$ rigid symmetries of the regular $n$-gon.
     }
     \item{
     Let $r$ be the rotation by $2\pi/n$ in the clockwise direction, and $s$ be the reflection along the vertical line going through the vertex labelled `1'.  Compute the elements of $D_{2n}$ in terms of $r$ and $s$ in the following steps:
     \begin{enumerate}
       \item{
       Compute the order of $r$ and $s$ (justifying your answers).
       }
       \item{
       Let $i_1,i_2\in\{0,1\}$ and $j_1,j_2\in\{0,1,\cdots,n-1\}$.  Show that:
       \[s^{i_1}r^{j_1}=s^{i_2}r^{j_2}\text{ if and only if }i_1=i_2\text{ and }j_1=j_2.\]
       }
       (Hint: You could first show $s\not=r^i$ for any $i$ using geometry.  The rest of the cases should follow from this and part (i) by using cancellation and 8(b).)
       \item{
       Conclude that $D_{2n} = \{s^ir^j|i=0,1$ and $j=0,1,\cdots,n-1\}$.  In particular, $r$ and $s$ generate $D_{2n}$.
       }
     \end{enumerate}
     }
     \item{
     Show that $rs = sr^{-1}$.  Deduce inductively from this that $r^ns = sr^{-n}$ for all $n$.
     }
   \end{enumerate}
   We now completely understand the algebraic structure of $D_{2n}$.  In particular, we know what every element looks like (in terms of $r$ and $s$) by (c), and we know how to multiply any two elements using the relation in part (d).  We summarize this by saying that $D_{2n}$ has the following presentation:
   \[D_{2n} = \la r,s | r^n = s^2 = 1, rs = sr^{-1}\ra.\]
   \begin{enumerate}
     \setcounter{enumii}{4}
     \item{
     Use this presentation to give an algebraic proof that every element which is not a power of $r$ has order 2.
     }
     \item{
     Bonus: Can you give a geometric interpretation of part (e)?
     }
   \end{enumerate}
 \end{enumerate}
\end{document}
