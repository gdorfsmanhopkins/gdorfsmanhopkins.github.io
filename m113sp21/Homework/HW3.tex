\documentclass[11pt]{article}
\usepackage[top = 1in, bottom = 1in, left =1in, right = 1in]{geometry}
\usepackage{graphicx}
\usepackage{amsmath}
\usepackage{tabu}
\usepackage{amssymb}
\usepackage{amsmath}
\usepackage{etoolbox}
\usepackage{xcolor}
\usepackage{amsthm}
\usepackage{tikz-cd}
\usepackage{tikz}
\usepackage{tkz-graph}
\usepackage{seqsplit}
\usepackage{ulem}
\usepackage{tabularx}
\AtBeginEnvironment{proof}{\color{blue}}
\newtheorem{theorem}{Theorem}
\newtheorem{proposition}{Proposition}
\newtheorem{lemma}{Lemma}
\newtheorem*{facts}{Fact}
\newtheorem*{remark}{Remark}
\newtheorem{corollary}{Corollary}
\newtheorem{definition}{Definition}
\usepackage{enumerate}
\usepackage{hyperref}
\usepackage{fancyhdr}\pagestyle{fancy}
\newcommand{\la}{\langle}
\newcommand{\ra}{\rangle}
\newcommand{\tors}{\mathrm{tors}}
\newcommand{\ab}{\mathrm{ab}}
\newcommand{\Aut}{\operatorname{Aut}}
\newcommand{\Inn}{\operatorname{Inn}}
\newcommand{\im}{\operatorname{im}}
\newcommand{\lcm}{\operatorname{lcm}}
\newcommand{\ch}{\operatorname{char}}

%Math blackboard:
\newcommand{\bC}{\mathbb{C}}
\newcommand{\bF}{\mathbb{F}}
\newcommand{\bN}{\mathbb{N}}
\newcommand{\bQ}{\mathbb{Q}}
\newcommand{\bR}{\mathbb{R}}
\newcommand{\bS}{\mathbb{S}}
\newcommand{\bZ}{\mathbb{Z}}

%Math caligraphy
\newcommand{\cC}{\mathcal{C}}
\newcommand{\cK}{\mathcal{K}}
\newcommand{\cM}{\mathcal{M}}
\newcommand{\cO}{\mathcal{O}}

%Greek blackboard font:
\newcommand{\bmu}{\mbox{$\raisebox{-0.59ex}
  {$l$}\hspace{-0.18em}\mu\hspace{-0.88em}\raisebox{-0.98ex}{\scalebox{2}
  {$\color{white}.$}}\hspace{-0.416em}\raisebox{+0.88ex}
  {$\color{white}.$}\hspace{0.46em}$}{}}

\lhead{University of California, Berkeley}
\rhead{Math 113, Spring 2021}

\begin{document}
\begin{center}
\Large {Homework Assignment 3}\\
\small {Due Friday, February 12}
\end{center}
\begin{enumerate}
  \item We begin by establishing important basic facts about group homomorphisms that we will use repeatedly throughout the course.  Let $G,H,K$ be groups, and let $\varphi:G\to H$ and $\psi:H\to K$ a homomorphisms.
  \begin{enumerate}
    \item Show that $\varphi(1_G) = 1_H$.
    \item Show that $\varphi(x^{-1}) = \varphi(x)^{-1}$ for all $x\in G$.
    \item Show that if $g\in G$ has finite order, then $|\varphi(g)|$ divides $|g|$.
    \item Show that if $\varphi$ is an isomorphism, then so is $\varphi^{-1}$.
    \item Show that if $\varphi$ is an isomorphism, $|\varphi(g)| = |g|$.
    \item Show that the composition $\psi\circ \varphi:G\to K$ is a homomorphism.
    \item Suppose $\varphi$ and $\psi$ are both isomorphisms.  Show that the composition $\psi\circ\varphi$ is as well.
    \item Conclude that the relation \textit{is isomorphic to} is an equivalence relation on the set of all groups.
  \end{enumerate}
  \item Given a homomorphism $\varphi:G\to H$, we obtain 2 important subgroups, one of $G$ and one of $H$.  They are called the \textit{kernel of $\varphi$} and \textit{image of $\varphi$} and are defined by the following rules:
  \begin{eqnarray*}
    \ker\varphi &=& \{g\in G:\varphi(g) = 1_H\},\\
    \operatorname{im}\varphi &=& \{h\in H:h =\varphi(g)\text{ for some }g\in G\}.
  \end{eqnarray*}
  \begin{enumerate}
    \item Show that $\ker\varphi$ is a subgroup of $G$.
    \item Show that $\im\varphi$ is a subgroup of $H$.
    \item{\textit{Important:} Show that $\varphi$ is injective if and only if $\ker\varphi = \{1_G\}$.  (This is an incredibly useful fact!)}
  \end{enumerate}
  \item The kernel has the following important generalization.  For $h\in H$ define the \textit{fiber over $h$} as
  \[\varphi^{-1}(h) = \{g\in G:\varphi(g) = h\}.\]
  This is sometimes also called the \textit{preimage of $h$}.  Observe that by definition, the kernel of $\varphi$ is the fiber over 1.
  \begin{enumerate}
    \item{Show that the fiber over $h$ is a subgroup if and only if $h=1_H$.}
    \item{Show that the \textit{nonempty} fibers of $\varphi$ form a partition of $G$.  (In particular, if $\varphi$ is surjective its fibers partition $G$.)}
    \item{Show that all nonempty fibers have the same cardinality.  (Hint: if $\varphi^{-1}(h)$ is nonempty, build a bijection between it and $\ker\varphi$.)  Observe that this generalizes 2(c).}
  \end{enumerate}
  \item Recall that we defined the kernel of a group action in class.  Let's justify our terminology.  Let $G\times A\to A$ be an action of $G$ on a set $A$ and let $\varphi:G\to S_A$ be the associated permutation representation.
  \begin{enumerate}
    \item Show that the kernel of the group action is equal to $\ker\varphi$.
    \item Show that the action is faithful if and only if the $\varphi$ is injective.  (Hint: Use 2(c).)
  \end{enumerate}
  \item We've seen that there is a relationship between the dihedral and symmetric groups.  Let's explore this a bit.
  \begin{enumerate}
    \item Describe an injective homomorphism from $\varphi:D_{2n}\to S_n$ (you may describe this in words, but make sure to justify injectivity).
    \item In the map you described, what is the cycle decomposition of $\varphi(r)$ (where as usual $r$ is the generator corresponding to clockwise rotation of the $n$-gon by $2\pi/n$)?
    \item Prove that $D_6\cong S_3$.
  \end{enumerate}
  \item In this exercise we show that you can compute the order of a permutation from its cycle decomposition.
  \begin{enumerate}
    \item Let $G$ be a group.  Two elements $x,y\in G$ are called \textit{commuting elements} if $xy = yx$.  Show that if $x$ and $y$ are commuting elements, then $(xy)^n = x^ny^n$.
    \item Give a counterexample to part (a) if the chosen elements do not commute.
    \item Let $\sigma = (a_1,a_2,\cdots,a_r)\in S_n$ be an $r$-cycle.  Show that $|\sigma| = r$.
    \item Prove that the order of a permutation is the least common multiple of the lengths of the cycles in its cycle decomposition.  (Hint: You may freely use that disjoint cycles are commuting elements.  You may find it useful to establish that the product of nontrivial disjoint cycles is never 1).
  \end{enumerate}
  \item We hinted in class that if $A$ and $B$ are sets of the same cardinality, then their permutation groups $S_A$ and $S_B$ (defined in HW2\#5) are isomorphic.  Let's prove it.  To begin, fix a bijective function $\theta:A\to B$.
  \begin{enumerate}
    \item Let $f:A\to A$ be bijective.  Show that $\theta\circ f\circ \theta^{-1}:B\to B$ is bijective.  (Hint: what is its inverse?)
    \item Part (a) allows us to construct the following function:
    \begin{eqnarray*}
      S_A&\stackrel{\varphi}{\longrightarrow}&S_B\\
      f&\longmapsto&\theta\circ f\circ\theta^{-1}.
    \end{eqnarray*}
    Show that $\varphi$ is an isomorphism, thereby proving the result.  (Note: There are two parts to this.  You must show that $\varphi$ is bijctive, and that it is a homomorphism.)
  \end{enumerate}
  \item The set $S_3$ has 6 elements.  Compute the order and cycle decomposition of each element.
\end{enumerate}
\end{document}
