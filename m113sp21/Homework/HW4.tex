\documentclass[11pt]{article}
\usepackage[top = 1in, bottom = 1in, left =1in, right = 1in]{geometry}
\usepackage{graphicx}
\usepackage{amsmath}
\usepackage{tabu}
\usepackage{amssymb}
\usepackage{amsmath}
\usepackage{etoolbox}
\usepackage{xcolor}
\usepackage{amsthm}
\usepackage{tikz-cd}
\usepackage{tikz}
\usepackage{tkz-graph}
\usepackage{seqsplit}
\usepackage{ulem}
\usepackage{tabularx}
\AtBeginEnvironment{proof}{\color{blue}}
\newtheorem{theorem}{Theorem}
\newtheorem{proposition}{Proposition}
\newtheorem{lemma}{Lemma}
\newtheorem*{facts}{Fact}
\newtheorem*{remark}{Remark}
\newtheorem{corollary}{Corollary}
\newtheorem{definition}{Definition}
\newtheorem*{hint}{Hint}
\usepackage{enumerate}
\usepackage{hyperref}
\usepackage{fancyhdr}\pagestyle{fancy}
\newcommand{\la}{\langle}
\newcommand{\ra}{\rangle}
\newcommand{\tors}{\mathrm{tors}}
\newcommand{\ab}{\mathrm{ab}}
\newcommand{\Aut}{\operatorname{Aut}}
\newcommand{\Inn}{\operatorname{Inn}}
\newcommand{\im}{\operatorname{im}}
\newcommand{\lcm}{\operatorname{lcm}}
\newcommand{\ch}{\operatorname{char}}

%Math blackboard:
\newcommand{\bC}{\mathbb{C}}
\newcommand{\bF}{\mathbb{F}}
\newcommand{\bN}{\mathbb{N}}
\newcommand{\bQ}{\mathbb{Q}}
\newcommand{\bR}{\mathbb{R}}
\newcommand{\bS}{\mathbb{S}}
\newcommand{\bZ}{\mathbb{Z}}

%Math caligraphy
\newcommand{\cA}{\mathcal{A}}
\newcommand{\cC}{\mathcal{C}}
\newcommand{\cK}{\mathcal{K}}
\newcommand{\cM}{\mathcal{M}}
\newcommand{\cO}{\mathcal{O}}

%Greek blackboard font:
\newcommand{\bmu}{\mbox{$\raisebox{-0.59ex}
  {$l$}\hspace{-0.18em}\mu\hspace{-0.88em}\raisebox{-0.98ex}{\scalebox{2}
  {$\color{white}.$}}\hspace{-0.416em}\raisebox{+0.88ex}
  {$\color{white}.$}\hspace{0.46em}$}{}}

\lhead{University of California, Berkeley}
\rhead{Math 113, Spring 2021}

\begin{document}
\begin{center}
\Large {Homework Assignment 4}\\
\small {Due Friday, February 19}
\end{center}
\begin{enumerate}
  \item Let $G$ be a group and $H$ a \textit{nonempty} subset of $G$.  Let's introduce a few tricks to speed up testing if something is a subgroup.
  \begin{enumerate}
    \item \textit{(Subgroup Criterion)} Suppose that for all $x,y\in H$, $xy^{-1}\in H$.  Show that $H$ is a subgroup of $G$.
    \item \textit{(Finite Subgroup Criterion)} Show that if $H$ is finite and closed under multiplication, then $H$ is a subgroup of $G$.
    \item Suppose now that $H$ is a subgroup of $G$, and that $K$ is another subgroup of $G$.  Show that if $K\subseteq H$, then $K\le H$.
  \end{enumerate}
	\item Let $G$ be a group.  Let $H,K\le G$ be two subgroups.
  \begin{enumerate}
    \item Show that the intersection $H\cap K$ is a subgroup of $G$.
    \item Give an example to show that the union $H\cup K$ need not be a subgroup of $G$.
    \item Show that $H\cup K$ is a subgroup of $G$ if and only if $H\subset K$ or $K\subset H$.
    \item Adjust your proof from part (a) to show that the intersection of an arbitrary collection of subgroups is a subgroup.  That is, let $\cA$ be a collection of subgroups of $G$.  Show that
    \[\bigcap_{H\in\cA}H\]
    is a subgroup of $G$.  This completes the proof that the subgroup generated by a subset is in fact a subgroup.
    \begin{hint}
      For part (d), the proof should be very similar to part (a), with only cosmetic modifications.  You won't need to use induction.  In fact, since $\cA$ is could in principle be uncountable, induction won't work without modifications (think about why this is).
    \end{hint}
  \end{enumerate}
  \item Let $G$ be a group, and let $A$ be a subset of $G$.  Let's establish some facts about centralizers and normalizers.
  \begin{enumerate}
    \item Let $A$ be a subset of $G$.  Prove that $N_G(A)\le G$.
    \item Deduce the following chain of inclusions.
    \[Z(G)\le C_G(A)\le N_G(A)\le G.\]
    \item Show that $C_G(A) = C_G(\la A\ra)$.
    \item Give an example to show the analog of part (c) for normalizers is not true.  That is, give $A\subseteq A$ where $N_G(A)\not=N_G(\la A\ra)$.
    \item Show that if $H$ is a subgroup of $G$, then $H\le N_G(H)$.
    \item Show that $H\le C_G(H)$ if and only if $H$ is abelian.
  \end{enumerate}
  \item Compute the center of the dihedral group.  Explicitly, let $n$ be an integer $\ge3$.  Compute $Z(D_{2n})$.  (Note: you will need to split into the two cases, where $n$ is even or $n$ is odd).
  \item In this exercise we study products of finite cyclic groups.  Recall that we denote by $Z_n$ the cyclic group of order $n$ (written multiplicatively).
  \begin{enumerate}
    \item Prove that $Z_2\times Z_2$ is not a cyclic group.
    \item Prove that $Z_2\times Z_3\cong Z_6$.  Conclude that $Z_2\times Z_3$ is a cyclic group.
  \end{enumerate}
  Those two examples really cover all the bases.  Use the intuition you gained from them to prove the following classification result.
  \begin{enumerate}
    \setcounter{enumii}{2}
    \item Show that $Z_n\times Z_m$ is cyclic if and only if $\gcd(n,m)=1$.  (Hint: recall that up to isomorphism there is only one cyclic group of order $N$ for every positive integer $N$).
  \end{enumerate}
  \item For $n\ge 2$ let $G = S_n$ be the symmetric group equipped with it's natural action on $\Omega_n = \{1,2,\cdots,n\}$ by permutations.  For $i\in\Omega_n$, let $G_i = \{\sigma\in G|\sigma(i)=i\}$ be the stabilizer of $i$.  Describe an isomorphism between $G_i$ and $S_{n-1}$.
  \item In this problem we will introduce the following very important class of subgroups.  A subgroup $H\le G$ is called \textit{normal} if $N_G(H) = G$.  Recall that this means, for $x\in G$, the set $xHx^{-1} = H$.  If $H$ is a normal subgroup, we write $H\unlhd G$.
  \begin{enumerate}
    \item Let $H$ be a subgroup, and $x\in G$.  Give a bijection between $H$ and $xHx^{-1}$.
    \item Part (a) makes it easy to check if something is normal.  In particular, suppose that for every $h\in H$, the element $xhx^{-1}\in H$ for every $x\in G$.  Show that $H$ is normal.
    \item Let $\varphi:G\to G'$ be a homomorphism with kernel $K$.  Show that $K$ is a normal subgroup of $G$.
    \item Give an example of a subgroup that is not normal.  Conclude that not every subgroup can be the kernel of some homomorphism.
  \end{enumerate}
  \item Let's study the converse of the previous question.  We will give an intrinsic definitnion of quotient groups along the way.  A lot of this problem is covered in class (with some details for you to fill in), but I think it is very important to work through these constructions carefully for yourself.  This should feel very similar to the construction of $\bZ/n\bZ$.

  Recall the following definition from class: Let $K\le G$ be a subgroup.  For $x,y\in G$  we say that $x$ and $y$ are congruent mod $K$, $x\equiv y\mod K$ if $y^{-1}x\in K$ (or equivalently if $x=yk$ for some $k\in K$).
  \begin{enumerate}
    \item Show that congruence modulo $K$ is an equivalence relation on $G$.  Observe that the the equivalence classes of congruence mod $K$ are the sets
    \[xK = \{xk:k\in K\}.\]
    We call these the \textit{cosets} of $K$.
    \item Suppose $K\unlhd G$.  If $x\equiv x_1\mod K$ and $y\equiv y_1\mod K$, show $xy\equiv x_1y_1\mod K$.  (You will need normality here.  Be careful not to assume your group is abelian).
    \item Define $G/K$ to be the set of cosets of $K$.
     \[G/K = \{xK:x\in X\}.\]
     If $K$ is normal, show that the operation $(xK)(yK) = xyK$ is a well defined binary operation making $G/K$ into a group.  What is the identity element?  (Note: You already did the work to show it's well defined.)
    \item Suppose $K$ is a normal subgroup.  Let $\pi:G\to G/K$ be the map $x\mapsto xK$.  Show that $\pi$ is a group homomorphisms with kernel $K$.  This is often called \textit{the natural projection}.
    \item Suppose that $G/K$ is a group under the operation described in part (c).  Show that $K$ must be normal (\textit{Hint:} Rather than trying to explicitly compute things with elements, use the then natural projection together with 7(c)).
    \item Putting everything together, conclude the following are equivalent for a subgroup $K\le G$.
    \begin{enumerate}[(i)]
      \item $K$ is normal in $G$.
      \item $K$ is the kernel of a homomorphism.
      \item $G/K$ is a group.
    \end{enumerate}
    \begin{hint}
      You've already done all the work for this.  Each implication should be easily accessible appealing to something proven in question 7 or 8.
    \end{hint}
  \end{enumerate}

\end{enumerate}
\end{document}
