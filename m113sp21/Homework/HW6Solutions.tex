\documentclass[11pt]{article}
\usepackage[top = 1in, bottom = 1in, left =1in, right = 1in]{geometry}
\usepackage{graphicx}
\usepackage{amsmath}
\usepackage{tabu}
\usepackage{amssymb}
\usepackage{amsmath}
\usepackage{etoolbox}
\usepackage{xcolor}
\usepackage{amsthm}
\usepackage{tikz-cd}
\usepackage{tikz}
\usepackage{tkz-graph}
\usepackage{seqsplit}
\usepackage{ulem}
\usepackage{tabularx}
\AtBeginEnvironment{proof}{\color{blue}}
\newtheorem{theorem}{Theorem}
\newtheorem{proposition}{Proposition}
\newtheorem{lemma}{Lemma}
\newtheorem*{facts}{Fact}
\newtheorem*{remark}{Remark}
\newtheorem{corollary}{Corollary}
\newtheorem{definition}{Definition}
\newtheorem*{hint}{Hint}
\usepackage{enumerate}
\usepackage{hyperref}
\usepackage{fancyhdr}\pagestyle{fancy}
\newcommand{\la}{\langle}
\newcommand{\ra}{\rangle}
\newcommand{\tors}{\mathrm{tors}}
\newcommand{\ab}{\mathrm{ab}}
\newcommand{\Aut}{\operatorname{Aut}}
\newcommand{\Inn}{\operatorname{Inn}}
\newcommand{\im}{\operatorname{im}}
\newcommand{\lcm}{\operatorname{lcm}}
\newcommand{\ch}{\operatorname{char}}

%Math blackboard:
\newcommand{\bC}{\mathbb{C}}
\newcommand{\bF}{\mathbb{F}}
\newcommand{\bN}{\mathbb{N}}
\newcommand{\bQ}{\mathbb{Q}}
\newcommand{\bR}{\mathbb{R}}
\newcommand{\bS}{\mathbb{S}}
\newcommand{\bZ}{\mathbb{Z}}

%Math caligraphy
\newcommand{\cA}{\mathcal{A}}
\newcommand{\cC}{\mathcal{C}}
\newcommand{\cK}{\mathcal{K}}
\newcommand{\cM}{\mathcal{M}}
\newcommand{\cO}{\mathcal{O}}

%Greek blackboard font:
\newcommand{\bmu}{\mbox{$\raisebox{-0.59ex}
  {$l$}\hspace{-0.18em}\mu\hspace{-0.88em}\raisebox{-0.98ex}{\scalebox{2}
  {$\color{white}.$}}\hspace{-0.416em}\raisebox{+0.88ex}
  {$\color{white}.$}\hspace{0.46em}$}{}}

\lhead{University of California, Berkeley}
\rhead{Math 113, Spring 2021}

\begin{document}
\begin{center}
\Large {Homework Assignment 6: Solutions}\\
\end{center}
\begin{enumerate}
  \item Let $G$ be a group, and $M,N\unlhd G$ normal subgroups such that $MN = G$.
  \begin{enumerate}
    \item Show $G/(M\cap N)\cong (G/M)\times (G/N)$
    \begin{proof}
      We freely use the fact that for $g\in G$, $gM=M$ if and only if $g\in M$, and similarly for $N$, which follows from HW4\#8(a).

      We build a homomorphism $\pi:G\to(G/M)\times(G/N)$ via the rule $\pi(g)=(gM,gN)$.  This is clearly a homomorphism since:
      \[\pi(xy) = (xyM,xyN) = (xMyM,xNyN) = (xM,xN)(yM,yN) = \pi(x)\pi(y).\]
      We now observe that $\pi$ is surjective.  Fix $(xM,yN)$ in the target.  Since $MN= G$, there is $m\in M$ and $n\in N$ such that $x^{-1}y = mn$.  Solving one gets $xm=yn^{-1}$, call this value $g$.  Then:
      \[\pi(g) = (gM,gN) = (xmM,yn^{-1}N) = (xM,yN).\]
      Finally, notice that the kernel of $\pi$ is the set of $g\in G$ such that $gM=M$ and $gN=N$.  But this is precisely $M\cap N$.  Therefore, the first isomorphism theorem gives the result.
    \end{proof}
    \item Suppose further that $M\cap N=\{1\}$.  Show that $G\cong M\times N$.
    \begin{proof}
      We will find the following lemma useful.
      \begin{lemma}
        Suppose $H_1\cong H_2$ and $K_1\cong K_2$.  Then $H_1\times K_1\cong H_2\cong K_2$.
      \end{lemma}
      \begin{proof}
        Let $\varphi:H_1\to H_2$ and $\psi:K_1\to K_2$ be isomorphisms.  Then we build:
        \begin{eqnarray*}
          \varphi\times\psi:H_1\times K_1&\longrightarrow&H_2\times K_2\\
          (h,k)&\mapsto&(\varphi(h),\psi(k)).
        \end{eqnarray*}
        It is easy to verify that $\varphi\times\psi$ is a homomorphism and that $(\varphi\times\psi)^{-1} = \varphi^{-1}\times\psi^{-1}$.
      \end{proof}
      Now to prove the result, we consider the diamond:
      \[
      \begin{tikzcd}
        &G = MN\ar[dl,dash]\ar[dr,dash]&\\
        M\ar[dr,dash]&&N\ar[dl,dash]\\
        &\{1\} = M\cap N&
      \end{tikzcd}
      \]
      By the second isomorphism theorem we have $G/M\cong N$ and $G/N\cong M$.  Therefore, the result follows from the following chain of isomorphisms, where the first is part (a), and the second is the lemma.
      \[G\cong(G/M)\times(G/N)\cong N\times M.\]
    \end{proof}
  \end{enumerate}
  \item Let $G$ be a group and $Z(G)$ its center.
  \begin{enumerate}
    \item Suppose $H\le Z(G)$.  Show that $H$ is a normal subgroup of $G$.  (In particular, $Z(G)$ is normal).
    \begin{proof}
      Fix $z\in H$ and $g\in G$.  It suffices to show $gzg^{-1}\in H$.  But since $z\in Z(G)$ we have $gzg^{-1} = gg^{-1}z = z\in H$, so we are done.
    \end{proof}
    \item Show that if $G/Z(G)$ is cyclic, then $G$ is abelian.
    \begin{proof}
      If $G/Z(G)$ is cyclic then we can fix a generator: $G/Z(G) = \la xZ(G)\ra$.  Then the cosets $x^i Z(G)$ for $i\in\bZ$ form a partition of $G$.  In particular, fix $a,b\in G$.  Then $a = x^iz$ and $b=x^jw$ for $z,w\in Z(G)$.  Therefore we can leverage that we can freely commute with $z$ and $w$, and $x^i$ and $x^j$ commute with eachother to conclude that
      \[ab = x^izy^jw = zx^i x^j w = zx^j x^i w = x^j zw x^i = x^j wz x^i = x^j w x^i z = ba.\]
      Thus $a$ and $b$ commute, but since they were arbitrary we conclude that $G$ is abelian.
    \end{proof}
    \item Let $p$ and $q$ be prime numbers (not necessarily distinct), and $G$ a group of order $pq$.  Show that if $G$ is not abelian, then $Z(G) = \{1\}$.
    \begin{proof}
      Since $G$ is not abelian then $Z(G)\not=G$.  If $Z(G)\not=1$ then by Lagrange's theorem, $Z(G)$ has either order $p$ or $q$.  Assume without loss of generality that it has order $q$.  Then $|G/Z(G)| = |G|/|Z(G)| = q$, so that $G/Z(G)$ has prime order and therefore must be cyclic (by TH1\#4(a)).  But then by part (b) $G$ must be abelian, a contradiction.  Therefore $Z(G)$ must be 1.
    \end{proof}
  \end{enumerate}
  \item Let's classify all groups of order 6.  To begin, let $G$ be a nonabelian group of order $6$.  We will show $G\cong S_3$.
  \begin{enumerate}
    \item Show that there is an element $x\in G$ of order 2.  (Once we have Cauchy's theorem for nonabelian groups this part becomes easy, but since $G$ has 6 elements, one can do this by inspection using Lagrange's theorem).
    \begin{proof}
      Since $G$ is not abelian, there is no element of order 6.  If there is also no element of order 2, then by Lagrange's theorem, $G = \{1,a,b,c,d,e\}$ where the order of $a,b,c,d,e$ are all 3.  Then $a^{-1}$ has order 3 as well, so without loss of generality $a^{-1}=b$, and similarly we may assume $c^{-1}=d$.  But this implies that $e^{-1}=e$ contradicting that it has order 3.
    \end{proof}
    \item Let $x\in G$ have order 2, and let $H = \la x\ra$.  Show that $H$ is not normal in $G$.  (\textit{Hint:} Show that if $H$ is normal then $H\le Z(G)$, then apply 2(c) to find a contradiction.)
    \begin{proof}
      Suppose $H$ is normal, so for all $g\in G,$ $gxg^{-1}\in H = \{1,x\}$.  If $gxg^{-1} = 1$ then $x=1$, so we must have $gxg^{-1} = x$.  This implies that $x\in Z(G)$ and so $H\le Z(G)$.  But since $G$ is nonabelian of order $6 = 2\cdot 3$, 2(c) says that its center must be trivial.
    \end{proof}
    \item Consider the action of $G$ on $A = G/H$ by left multiplication.  Show that the associated permutation representation is injective.  Conclude that $G\cong S_3$.
    \begin{proof}
      The action of $G$ on $A$ gives a homomorphism $\varphi:G\to S_A$, and the target (by HW3\#7) is isomorphic to $S_3$.  If the action of $G$ on $A$ is faithful, then (by HW3\#4), $\varphi$ is injective, so that we get an injective homomorphism $G\to S_3$,  Since they both have order 6, HW1\#5 says this has to be an isomorphism.  It therefore suffices to show that the action of $G$ on $A$ is faithful.

      Let $K$ be the kernel of the action, and suppose that $g\in G$ acts trivially on $A$.  In particular, this means that $g\cdot H = gH = H$, so that $g\in H$.  This shows that $K\le H$.  Since $H$ has order 2, this means $K = 1$ or $K=H$.  But $K$ is normal, and by part (b), $H$ is not normal, so the only possibility is that $K=1$, which was our goal.
    \end{proof}
    \item Complete the classification of all groups of order 6 by showing that if $Z$ is an abelian group of order 6 then $Z\cong Z_6$.  (\textit{Hint:} We do have Cauchy's theorem for abelian groups.)  \textit{We've now classified groups of order $\le7$.}
    \begin{proof}
      By Cauchy's theorem, there are $x,y\in Z$ of order 2 and 3 respectively.  We will show that $|xy|=6$, which gives the result.  By Lagrange's theorem, we know $\langle x\rangle\cap\langle y\rangle = 1$.  Notice that if $x^iy^j=1$, then $x^i=y^{-j}$, so that $y^{-j}\in\langle x\rangle$ and so it must be 1, and so $x^i=1$ as well.  In particular, if $(xy)^n = x^ny^n=1$, then $x^n=y^n=1$.  By HW2\#8(c), this means $2|n$ and $3|n$, so that $6|n$.  Therefore $|xy|=6$ as desired.
    \end{proof}
  \end{enumerate}
  \item Let $G$ be a group.  Let $[G,G] = \la x^{-1}y^{-1}xy | x,y\in G\ra$.
  \begin{enumerate}
    \item Show that $[G,G]$ is a normal subgroup of $G$.
    \begin{proof}
      Notice that $[G,G]$ is not the set of elements of the form $x^{-1}y^{-1}xy$, it is the subgroup \textit{generated} by elements of that form.  So we need not show it is a subgroup.  Lets first prove a lemma.
      \begin{lemma}
        Let $H$ be a group and consider a subset $S$.  To see that $\la S\ra$ is normal it suffices to show $hsh^{-1}\in\la S\ra$ for all $h\in H$ and $s\in S$.
      \end{lemma}
      \begin{proof}
        An arbitrary element in $\la S\ra$ looks like $s = s_1s_2\cdots s_n$ for $s_i$ or $s_i^{-1}$ in $S$.  Then by assumption $g s_i g^{-1}\in\la S\ra$, so that:
        \[gsg^{-1} = g(s_1s_2\cdots s_n)g^{-1} = (gs_1g^{-1})(gs_2g^{-1})\cdots (gs_ng^{-1})\in\la S\ra.\]
      \end{proof}
      Therefore for $g$ and a commutator $x^{-1}y^{-1}xy$, we notice:
      \[g(x^{-1}y^{-1}xy)g^{-1} = gx^{-1}(g^{-1}g)y^{-1}(g^{-1}g)x(g^{-1}g)yg^{-1} = (gxg^{-1})^{-1}(gyg^{-1})^{-1}(gxg^{-1})(gyg^{-1}),\]
      is also a commutator.  Therefore the subgroup is normal.

      We concluded the proof above, but there is a slightly slicker way to see this, following from the next lemma.
      \begin{lemma}
        Let $\varphi:H\to K$ is a homomorphism of groups.  Then the image of a commutator is a commutator.
      \end{lemma}
      \begin{proof}
        This is immediate, as $\varphi(x^{-1}y^{-1}xy) = \varphi(x)^{-1}\varphi(y)^{-1}\varphi(x)\varphi(y)$.
      \end{proof}
      Then we need only notice that for every $g\in G$, the conjugation map $\varphi_g:G\to G$ given by $\varphi_g(x) = gxg^{-1}$ is a homomorphism.  But we showed this in class: indeed,
      \[\varphi_g(xy) = gxyg^{-1} = gxg^{-1}gyg^{-1} = \varphi_g(x)\varphi_g(y).\]
      Then we immediatly conclude that conjugating a commutator gives a commutator.
    \end{proof}
    \item Show that $G/[G,G]$ is abelian.
    \begin{proof}
      We must show that the cosets $xy[G,G]$ and $yx[G,G]$ are equal.  But $x^{-1}y^{-1}xy\in[G,G]$ so that
      \[xy = yx(x^{-1}y^{-1}xy)\in yx[G,G].\]
      Since the cosets form a partition, we are done.
    \end{proof}
  \end{enumerate}
  $[G,G]$ is called the \textit{commutator subgroup} of $G$, and $G/[G,G]$ is called the \textit{abelianization} of $G$, denoted $G^\ab$.  The rest of this exercise explains why.
  \begin{enumerate}
    \setcounter{enumii}{2}
    \item Let $\varphi:G\to H$ be a homomorhism with $H$ abelian.  Show $[G,G]\subseteq\ker\varphi$.
    \begin{proof}
      It suffices to show that every element $x^{-1}y^{-1}xy\in G$ is in the kernel of $\varphi$, since then $[G,G]$ is generated by elements in the kernel.  But then:
      \[\varphi(x^{-1}y^{-1}xy) = \varphi(x)^{-1}\varphi(y)^{-1}\varphi(x)\varphi(y) = \varphi(x)\varphi(x)^{-1}\varphi(y)^{-1}\varphi(y)=1,\]
      as $H$ is abelian.  (Notice we also just showed that the commutator subgroup of an abelian group is always the trivial subgroup).
    \end{proof}
    \item Conclude that for $H$ an abelian group there is a bijection:
    \[\left\{
    \begin{array}{c}
      \text{Homomorphisms }\varphi:G\to H\\
    \end{array}\right\}
    \Longleftrightarrow
    \left\{
    \begin{array}{c}
      \text{Homomorphisms }\tilde\varphi:G^\ab\to H\\
    \end{array}
    \right\}
    \]
    \begin{hint}
      Recall the technique of passing to the quotient described at the beginning of the 2/23 lecture
    \end{hint}
    \begin{proof}
      We remind the reader of the statement of ``Passing to the Quotient."
      \begin{lemma}[Passing to the Quotient]
        Let $N\unlhd G$ be a normal subgroup, and $\varphi:G\to H$ a homomorphism.  If $N\le\ker\varphi$, then there is a unique homomorphism $\tilde\varphi:G/N\to H$ such that $\tilde\varphi\circ\pi=\varphi$, defined by the rule $\tilde\varphi(gN) = \varphi(g)$.  This is summarized by the following diagram.
        \[
        \begin{tikzcd}
          G\ar[d,swap,"\pi"]\ar[rr,"\varphi"]&& H\\
          G/N \ar[urr,dotted,swap,"\tilde\varphi"].
        \end{tikzcd}
        \]
      \end{lemma}
      With this lemma we prove part (d).  In the righthand direction we define a function $\Phi$ which takes a map $\varphi:G\to H$ to the unique map $\tilde\varphi$ from the lemma, which exists because $[G,G]\le\ker\varphi$ by part (c).  In the other direction define $\Psi$ which takes a map $\tilde\varphi$ to the composition $\varphi = \tilde\varphi\circ\pi$:
      \[
      \begin{tikzcd}
        G\ar[r,"\pi"]\ar[rr,bend left = 40,"\varphi"] & G^\ab \ar[r,"\tilde\varphi"] & H.
      \end{tikzcd}
      \]
      We must prove these processes are inverses to eachother.  But this is obvious.  $\Psi\circ\Phi(\varphi) = \tilde\varphi\circ\pi = \varphi$ by definition, and $\Phi\circ\Psi(\tilde\varphi) = \Phi(\tilde\varphi\circ\pi) = \tilde\varphi$ by the uniqueness of $\tilde\varphi$.

      We make a remark that this is a sort of \textit{universal property}, in that $G^\ab$ is the universal abelianization of $G$.  I won't get into precisely what this means at the moment, but it can be understood via the slogan: Maps from $G$ to abelian things are the same as maps from $G^\ab$ to abelian things.
    \end{proof}
  \end{enumerate}
  \item Let's now compute $D_{2n}^\ab$.  We should begin computing $xyx^{-1}y^{-1}$.  There are 3 cases.
  \begin{enumerate}
    \item Compute $x^{-1}y^{-1}xy$ in each of the following 3 cases. (\textit{Hint:} HW2\#9(e) gives the inverse for a reflection.)
    \begin{enumerate}[(i)]
      \item $x,y$ both reflections.  So $x=sr^i$ and $y=sr^j$.
      \begin{proof}
        Since reflections always have order two, we have $x^{-1}=x$ and $y^{-1} = y$.  That is:
        \[x^{-1}y^{-1}xy = (sr^i)(sr^j)(sr^i)(sr^j) = r^{j-i}r^{j-i} = r^{2(j-i)}\]
        As $i$ and $j$ vary we collect all even powers of $r$.
      \end{proof}
      \item $x$ a reflection and $y$ not a reflection.  So $x=sr^i$ and $y=r^j$.
      \begin{proof}
        In this case $x^{-1}=x$, but that is not true for $y$.  We computeL
        \[x^{-1}y^{-1}xy = (sr^i)(r^{-j})(sr^i)(r^j) = (sr^{i-j})(sr^{i+j}) = r^{2j},\]
        and as above we collect precisely the even powers of $r$.
      \end{proof}
      \item Neither $x$ nor $y$ are reflections.  So $x=r^i$ and $y=r^j$.
      \begin{proof}
        Here $x$ and $y$ commute so their commutator is 1.
      \end{proof}
    \end{enumerate}
    \item Prove that $[D_{2n},D_{2n}] = \la r^2\ra$.  If $n$ is odd one could choose another generator.  What is it?
    \begin{proof}
      We saw in part (a) that the commutators of $D_{2n}$ are precisely the even powers of $r$, proving the first statement.  If $n$ is odd, then $(n+1)/2$ is an integer and we can compute
      \[(r^2)^{(n+1)/2} = r^{n+1} = r,\]
      so that in fact the commutator subgroup is $\la r\ra$.
    \end{proof}
    \item Now prove that $D_{2n}^\ab$ is either $V_4$ or $Z_2$ depending on whether $n$ is odd or even.  Note that since this is so small we should interpret this as suggesting that $D_{2n}$ is far from abelian.
    \begin{proof}
      Note that:
      \[|D_{2n}^\ab| = |D_{2n}/|[D_{2n},D_{2n}]| = |D_{2n}|/|[D_{2n},D_{2n}]|.\]
      If $n$ is odd, then $|[D_{2n},D_{2n}]| = n$ which is half the order of $D_{2n}$.  Thus $|D_{2n}^\ab| = 2$, and so it must be $Z_2$ by TH1\#4(a).\\

      If $n$ is even then $|[D_{2n},D_{2n}]| = n/2$, a quarter of the order of $D_{2n}$, and so $|D_{2n}^\ab| = 4$ so it must be $Z_4$ or $V_4$ by TH1\#4(d).  To see it is $V_4$ we will show every element has order 2.  The cosets are represented by $r$, $s$, and $sr$.  The latter two have order two already in $D_{2n}$, so it remains to show that the coset represented by $r$ does too, but its square is $r^2$ which generates the commutator subgroup. Since every element of $D_{2n}^\ab$ has order 2, it must be the group $V_4$.
    \end{proof}
  \end{enumerate}
\end{enumerate}
For the remainder we will study the quaternion group $Q_8$.  It is a nonabelian group with very interesting properties.
\begin{definition}
  The \textit{quaternion group of order 8}, denoted $Q_8$ is the group of the following 8 elements:
  \[Q_8 = \{\pm1,\pm i, \pm j, \pm k\}\]
  subject to the relations:
  \[(-1)^2 = 1\]
  \[i^2 = j^2 = k^2 = -1,\]
  \[(-1)x = -x = x(-1)\text{ for all }x,\]
  \begin{eqnarray*}
    ij = k, & \hspace{20pt} & ji = -k,\\
    jk = i, & \hspace{20pt} & kj = -i,\\
    ki = j, & \hspace{20pt} & ik = -j.
  \end{eqnarray*}
\end{definition}
\begin{enumerate}
  \setcounter{enumi}{5}
  \item Let's start with a few simple facts.  Much of this is worked out in the book.
  \begin{enumerate}
    \item Write the entire multiplication table for $Q_8$.
    \begin{proof}
      The group is nonabelian, so we make sure to stick to the convention that in row $a$ and column $b$ we are writing $ab$ (rather than $ba$),
      \[
      \begin{tabu}{c|c|c|c|c|c|c|c|c|}
        & 1 & -1 & i & -i & j & -j & k & -k\\
        \hline
        1 & 1 & -1 & i & -i & j & -j & k & -k\\
        \hline
        -1 & -1 & 1 & -i & i & -j & j & -k & k\\
        \hline
        i & i & -i & -1 & 1 & k & -k & -j & j\\
        \hline
        -i & -i & i & 1 & -1 & -k & k & j & -j\\
        \hline
        j & j & -j & -k & k & -1 & 1 & i & -i\\
        \hline
        -j & -j & j & k & -k & 1 & -1 & -i & i\\
        \hline
        k & k & -k & j & -j & -i & i & -1 & 1\\
        \hline
        -k & -k & k & -j & j & i & -i & 1 & -1
        \end{tabu}
      \]
    \end{proof}
    \item Find 2 elements which generate all of $Q_8$.  (\textit{Bonus:} Can you give a presentation of $Q_8$?)
    \begin{proof}
      Notice that $i$ and $j$ generate everything.  Indeed:
      \[
      \begin{tabu}{c c c}
        -1 = i^2 & -i = i^3 & -j = j^3\\
        1 = i^4 & k = ij & -k = ji.
      \end{tabu}
      \]
      The following is an intuitive presentation, but I want to point out that $-1$ is tacitly a generator here:
      \[\la i,j\text{ }|\text{ }i^2 = j^2 = -1, ij = -ji\ra.\]
      This answer is acceptable on this assignment, but not precisely correct.  We probably want to assume in our presentation that we don't know what $-1$ is (i.e., that its square is 1).   The correct presentation, that doesn't include $-1$ secretly is:
      \[\la i,j\text{ }|\text{ }i^4 = j^4 = 1\text{, }i^2 = j^2\text{ and }ji = i^3j\ra.\]
      Where translating back to the more intuitive notation $i^2=j^2=-1$, $ij=k$, and $ji = i^3j = (i^2)ij = -k$.
    \end{proof}
    \item Prove that $Q_8$ is not isomorphic to $D_8$.
    \begin{proof}
      The easiest way to see this is to notice that if they were isomorphic, they would need to have the same number of elements of order $n$ for each $n$.  Then we can consider the order of every element in each group.
      \[
      \begin{tabu}{c|c|c|c|c}
        Q_8 & \text{order} & & D_8 & \text{order}\\
        \hline
        1 & 1 & & 1 & 1\\
        -1 & 2 & & r & 4\\
        i & 4 & & r^2 & 2\\
        -i & 4 & & r^3 & 4\\
        j & 4 & & s & 2\\
        -j & 4 & & sr & 2\\
        k & 4 & & sr^2 & 2\\
        -k & 4 & & sr^3 & 2
      \end{tabu}
      \]
      In particular, $Q_8$ only has one element of order 2 whereas $D_8$ has 5.
    \end{proof}
    \item Find all the subgroups of $Q_8$, and draw its lattice.  (\textit{Hint}: there are 6 total subgroups).
    \begin{proof}
      The nontrivial subgroups (i.e., those which aren't $Q_8$ and $\{1\}$) must have orders 2 or 4 by Lagranges theorem.  The order 2 subgroups must be cyclic, generated by an element of order 2.  The only element of order $2$ is $-1$, so the only subgroup of order 2 is $\{\pm1\}$.  As for subgroups of order four, they are either cyclic or isomorphic to the Klein 4 group $V_4$.  But $V_4$ must be generated by 2 elements of order 2, and $Q_8$ only has one.  Thus each subgroup of order 4 is cyclic.  There are 6 elements of order 4, but $-i = i^3$, and similarly for $j$ and $k$, so there 3 subgroups of order 4 generated by $i$ and $j$ and $k$.  As $i^2 = j^2 = k^2 = -1$, the subgroup $\{\pm1\}$ is contained in all of them.  thus the lattice is as follows.
      \[
      \begin{tikzcd}
        & Q_8\ar[d,dash]\ar[dl,dash]\ar[dr,dash] &\\
        \la i\ra & \la j\ra & \la k\ra\\
        & \{\pm1\}\ar[u,dash]\ar[ur,dash]\ar[ul,dash]\ar[d,dash] &\\
        &\{1\}&
      \end{tikzcd}
      \]\
    \end{proof}
    \item Prove that every subgroup of $Q_8$ is normal.
    \begin{proof}
      $Q_8$ and $\{1\}$ are automatically normal.  Next notice that since $-1*a = a*-1$ for each $a\in Q_8$.  Thus $\{\pm1\}$ is contained in the center of $Q_8$ and is therefore normal by 2(a) above.

      The cases for $\la i\ra, \la j\ra$ and $\la k\ra$ are completely symmetric, so we just treat the case of $H = \la i\ra$.  Notice that
      \[H\le N_{Q_8}(H)\le Q_8.\]
      Also $|H| = 4$ and $|N_{Q_8}(H)|$ divides 8 by Lagrange's theorem, so that $N_{Q_8}(H)$ is either $H$ or all of $Q_8$.  Thus if we exhibit one element of the normalizer which is not in $H$, the normalizer is all of $Q_8$, which precisely means that $H\unlhd Q_8$.  Notice that:
      \[jij^{-1} = ji(-j) = (-k)(-j) = kj = -i\in\la i\ra.\]
      Thus $j\in N_{D_8}(H)$ and we are done.
    \end{proof}
    \item Prove that every \textit{proper} subgroup and quotient group of $Q_8$ is abelian (\textit{Hint}: TH1\#4).
    \begin{proof}
      Let $H$ be a proper subgroup or quotient of $Q_8$.  Then by Lagrange's theorem, $|H| = 1,2$ or $4$.  In the first case $H$ is the trivial group which is abelian, in the second it is isomorphic to $Z_2$ which is abelian, and in the third it is isomorphic to either $Z_4$ or $V_4$ which are abelian.
    \end{proof}
    \item Compute $Z(Q_8)$ and $Q_8/Z(Q_8)$ (\textit{Hint for the second part}: you can do this by hand, but it might be slicker to apply 2(b)).
    \begin{proof}
      It is readily checked using the multiplication table in part (a) that $Z(Q_8) = \{\pm1\}$.  Then
      \[|Q_8/Z(Q_8)| = |Q_8|/|\{\pm1\}| = 8/2 = 4.\]
      Then in particular, it is either cyclic or isomorphic to $V_4$.  If it is cyclic, then 2(b) says that $Q_8$ is abelian, which is false.  So the quotient is $V_4$.  (Note, one could also use the lattice from part (d) together with the fourth isomorphism theorem to see that the lattice of the quotient has to be the lattice above $\{\pm1\}$, which is the lattice of $V_4$).
    \end{proof}

  \end{enumerate}
  \item Now let's follow the proof of Cayley's theorem to exhibit $Q_8$ as a subgroup of $S_8$.
  \begin{enumerate}
    \item Label $\{1,-1,i,-i,j,-j,k,-k\}$ as the numbers $\{1,2,\cdots,8\}$. Then the action of $Q_8$ on itself by left multiplication gives an injective map $Q_8\to S_8$.  Write the permutation representations for $-1$ and $i$ as elements $\sigma_{-1},\sigma_i\in S_8$, and verify that $\sigma_i^2 = \sigma_{-1}$.  (Using the multiplication table from question 1 will make this easier).
    \begin{proof}
      Let's first compute $\sigma_{-1}$.
      \[
      \begin{tabu}{c c c}
        -1 * 1 = -1 & \leftrightarrow & \sigma_{-1}(1) = 2\\
        -1 * -1 = 1 & \leftrightarrow & \sigma_{-1}(2) = 1\\
        -1 * i = -i & \leftrightarrow & \sigma_{-1}(3) = 4\\
        -1 * -i = i & \leftrightarrow & \sigma_{-1}(4) = 3\\
        -1 * j = -j & \leftrightarrow & \sigma_{-1}(5) = 6\\
        -1 * -j = j & \leftrightarrow & \sigma_{-1}(6) = 5\\
        -1 * k = -k & \leftrightarrow & \sigma_{-1}(7) = 8\\
        -1 * -k = k & \leftrightarrow & \sigma_{-1}(8) = 7
      \end{tabu}
      \]
      Thus $\sigma_{-1}$ swaps 1 and 2, 3 and 4, 5 and 6, 7 and 8.  That is:
      \[\sigma_{-1} = (12)(34)(56)(78)\in S_8.\]
      Let's do a similar computation for $\sigma_{i}$.
      \[
      \begin{tabu}{c c c}
        i * 1 = i & \leftrightarrow & \sigma_{i}(1) = 3\\
        i * -1 = -i & \leftrightarrow & \sigma_{i}(2) = 4\\
        i * i = -1 & \leftrightarrow & \sigma_{i}(3) = 2\\
        i * -i = 1 & \leftrightarrow & \sigma_{i}(4) = 1\\
        i * j = k & \leftrightarrow & \sigma_{i}(5) = 7\\
        i * -j = -k & \leftrightarrow & \sigma_{i}(6) = 8\\
        i * k = -j & \leftrightarrow & \sigma_{i}(7) = 6\\
        i * -k = j & \leftrightarrow & \sigma_{i}(8) = 5
      \end{tabu}
      \]
      Thus $\sigma_i$ takes 1 to 3 to 2 to 4 to 1, while taking 5 to 7 to 6 to 8 and back to 5.  Thus we have:
      \[\sigma_i = (1324)(5768)\in S_8.\]
      Next we compute the square of $\sigma_i$ by hand, using in the first equality that disjoint cycles commute.
      \begin{eqnarray*}
        (\sigma_i)^2 &=& (1324)^2(5768)^2\\
        &=&(1324)(1324)(5768)(5768)\\
        &=&(12)(34)(56)(78).
      \end{eqnarray*}
    \end{proof}
    \item Use the generators from question 6(b) to give two elements of $S_8$ which generate a subgroup $H\le S_8$ isomorphic to $Q_8$.
    \begin{proof}
      Since $i$ and $j$ generate $Q_8$, the permutations $\sigma_i$ and $\sigma_j$ generate the isomorphic subgroup of $S_8$.  Thus we must also compute $\sigma_j$ like we did for $i$ and $-1$ in part (a).
      \[
      \begin{tabu}{c c c}
        j * 1 = j & \leftrightarrow & \sigma_{j}(1) = 5\\
        j * -1 = -j & \leftrightarrow & \sigma_{j}(2) = 6\\
        j * i = -k & \leftrightarrow & \sigma_{j}(3) = 8\\
        j * -i = k & \leftrightarrow & \sigma_{j}(4) = 7\\
        j * j = -1 & \leftrightarrow & \sigma_{j}(5) = 2\\
        j * -j = 1 & \leftrightarrow & \sigma_{j}(6) = 1\\
        j * k = i & \leftrightarrow & \sigma_{j}(7) = 3\\
        j * -k = -i & \leftrightarrow & \sigma_{j}(8) = 4
      \end{tabu}
      \]
      Therefore we get:
      \[\sigma_j = (1526)(3847).\]
      Thus we have:
      \[Q_8\cong\la\sigma_i,\sigma_j\ra = \la(1324)(5768),(1526)(3847)\ra\le S_8.\]
    \end{proof}
    \item Is $\sigma_i$ even or odd?
    \begin{proof}
      Let's compute the sign.  We use the fact that the sign of an $m$-cycle is even if and only if $m$ is odd.  Then,
      \[\epsilon((1324)(5768)) = \epsilon((1324))\epsilon((5768)) = (1)(1) = 1.\]
      Thus $\sigma_i$ is even.
    \end{proof}
    \item $A_8\cap H$ is isomorphic to a subgroup of $Q_8$.  Which one?
    \begin{proof}
      As in part (c) one can easily compute that $\sigma_j$ is even as well, so that the entire subgroup they generate is contained in $A_8$.  Thus $A_8\cap H = H \cong Q_8$.
    \end{proof}
  \end{enumerate}
  \item Cayley's theorem says that if $|G|=n$ then $G$ embeds at $S_n$.  One could ask if this $n$ is \textit{sharp}, or if perhaps $G$ can embed in some smaller symmetric group.  For example, $D_8$ embeds in $S_4$ (thinking about symmetries of the square as permutations of the vertices, cf HW3\#5).  Nevertheless, for $Q_8$ the symmetric group given by Cayley's theorem is the smallest.
  \begin{enumerate}
    \item Let $Q_8$ act on a set $A$ with $|A|\le 7$.  Let $a\in A$.  Show that the stabilizer of $a$,  $(Q_8)_a\le Q_8$ must contain the subgroup $\{\pm1\}$.  (\textit{Hint:} The orbit stabilizer theorem might help.)
    \begin{proof}
      Let $a\in A$, and denote the stabilizer of $a$ by the subgroup $(Q_8)_a\le Q_8$.  Then recall that the index of the stabilizer of $a$ is $Q_8$ is the same as the size of the orbit of a $Q_8\cdot a$ which is a subset of $A$.  That is:
      \[|Q_8:(Q_8)_a| = |Q_8\cdot a|\le |A|\le 7<8.\]
      The left hand size is $8/|(Q_8)_a|$ by Lagrange's theorem, so that $(Q_8)_a$ cannot be the trivial subgroup of $Q_8$.  But in the lattice from 1(d), we saw that every nontrivial subgroup of $Q_8$ contains $\{\pm1\}$, completing the proof.
    \end{proof}
    \item Deduce that the kernel of the action of $Q_8$ on $A$ contains $\{\pm1\}$.
    \begin{proof}
      $\{\pm1\}$ is contained in the stabilizer of every element of $A$ by part (a), and so it acts trivially on all of $A$.  This is precisely what it means to be in the kernel.
    \end{proof}
    \item Conclude that $Q_8$ cannot embed into $S_n$ for $n\le7$.  That is, show there is no injective homomorphisms $Q_8\hookrightarrow S_n$ for $n\le7$.
    \begin{proof}
      By HW3\#4, an embedding $Q_8\hookrightarrow S_n$ corresponds to a faithful action on the set $\{1,2,\cdots,n\}$.  But we just saw that if $n\le 7$, any action on $\{1,2,\cdots,n\}$ has a nontrivial kernel.
    \end{proof}
  \end{enumerate}
\end{enumerate}
\end{document}
