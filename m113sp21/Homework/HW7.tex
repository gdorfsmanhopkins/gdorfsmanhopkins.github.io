\documentclass[11pt]{article}
\usepackage[top = 1in, bottom = 1in, left =1in, right = 1in]{geometry}
\usepackage{graphicx}
\usepackage{amsmath}
\usepackage{tabu}
\usepackage{amssymb}
\usepackage{amsmath}
\usepackage{etoolbox}
\usepackage{xcolor}
\usepackage{amsthm}
\usepackage{tikz-cd}
\usepackage{tikz}
\usepackage{tkz-graph}
\usepackage{seqsplit}
\usepackage{ulem}
\usepackage{tabularx}
\AtBeginEnvironment{proof}{\color{blue}}
\newtheorem{theorem}{Theorem}
\newtheorem{proposition}{Proposition}
\newtheorem{lemma}{Lemma}
\newtheorem*{facts}{Fact}
\newtheorem*{remark}{Remark}
\newtheorem{corollary}{Corollary}
\newtheorem{definition}{Definition}
\newtheorem*{hint}{Hint}
\usepackage{enumerate}
\usepackage{hyperref}
\usepackage{fancyhdr}\pagestyle{fancy}
\newcommand{\la}{\langle}
\newcommand{\ra}{\rangle}
\newcommand{\tors}{\mathrm{tors}}
\newcommand{\ab}{\mathrm{ab}}
\newcommand{\Aut}{\operatorname{Aut}}
\newcommand{\Inn}{\operatorname{Inn}}
\newcommand{\Out}{\operatorname{Out}}
\newcommand{\im}{\operatorname{im}}
\newcommand{\lcm}{\operatorname{lcm}}
\newcommand{\ch}{\operatorname{char}}

%Math blackboard:
\newcommand{\bC}{\mathbb{C}}
\newcommand{\bF}{\mathbb{F}}
\newcommand{\bN}{\mathbb{N}}
\newcommand{\bQ}{\mathbb{Q}}
\newcommand{\bR}{\mathbb{R}}
\newcommand{\bS}{\mathbb{S}}
\newcommand{\bZ}{\mathbb{Z}}

%Math caligraphy
\newcommand{\cA}{\mathcal{A}}
\newcommand{\cC}{\mathcal{C}}
\newcommand{\cK}{\mathcal{K}}
\newcommand{\cM}{\mathcal{M}}
\newcommand{\cO}{\mathcal{O}}

%Greek blackboard font:
\newcommand{\bmu}{\mbox{$\raisebox{-0.59ex}
  {$l$}\hspace{-0.18em}\mu\hspace{-0.88em}\raisebox{-0.98ex}{\scalebox{2}
  {$\color{white}.$}}\hspace{-0.416em}\raisebox{+0.88ex}
  {$\color{white}.$}\hspace{0.46em}$}{}}

\lhead{University of California, Berkeley}
\rhead{Math 113, Spring 2021}

\begin{document}
\begin{center}
\Large {Homework Assignment 7}\\
\small {Due Friday, March 12}
\end{center}
\begin{enumerate}
  \item Let $n\ge3$.  Show that $Z(S_n) = \{(1)\}$.
  \item Let $G$ be a group.  Prove that that $\Inn(G)\unlhd\Aut(G)$.  The quotient $\Aut(G)/\Inn(G)$ is called the \textit{outer automorphism group} of $G$, and is denoted by $\Out(G)$
  \item The converse to Lagrange's theorem holds for groups of prime power order.  To prove this we will need to strengthen the fourth isomorphism theorem (HW5\#1).
  \begin{enumerate}
    \item Let $G$ be a group and $N\unlhd G$.  Let $N\le H\le K\le G$, and let $\overline H,\overline K$ be the corresponding subgroups of $G/N$ as in HW5\#1.  Show that $|K:H| = |\overline K:\overline H|$.  (\textit{Hint}: There is an obvious map $K/H\to\overline K/\overline H$.  Prove it is bijective.  Be careful though, we don't know that $K/H$ is a group, just a set of cosets.)
    \item Suppose $|G| = p^d$ for a prime $p$ and $d\ge1$.  Show that for every $a = 1,2,\cdots,d$, $G$ has a subgroup of order $p^a$.  (\textit{Hint}: Use what we know about the center of a group of $p$-power order and proceed by induction using part (a)).
  \end{enumerate}
  \item Find all groups with exactly 2 conjugacy classes.  (\textit{Hint}: Use the class equation.)
\end{enumerate}
For the next question we remind the reader of the following definitions from linear algebra.
\begin{definition}
  Let $F$ be a field, with additive identity 0 and multiplicative identity 1.  An \textit{$F$-vector space} $V$ is an abelian group $(V,+)$ together with a \textit{scalar multiplication function} $F\times V\to V$ denoted $(\lambda,v)\mapsto \lambda v$ such that for all $u,v\in V$ and $\lambda,\tau\in F$:
  \begin{enumerate}[(1)]
    \item $0v = 0.$
    \item $1v = v.$
    \item $\lambda(\tau v) = (\lambda\tau)v.$
    \item $\lambda(u+v) = \lambda u+\lambda v.$
  \end{enumerate}
  Let $V,W$ be two $F$-vector spaces.  A function $\varphi:V\to W$ is called \textit{$F$-linear} if for all $u,v\in V$ and $\lambda\in F$:
  \begin{enumerate}[(1)]
    \item $\varphi(u+v) = \varphi(u) + \varphi(v).$
    \item $\varphi(\lambda v) = \lambda\varphi(v).$
  \end{enumerate}
\end{definition}
\begin{enumerate}
  \setcounter{enumi}{4}
  \item Fix a prime $p$ and let
  \[V = \underbrace{\bZ/p\bZ\times\bZ/p\bZ\cdots\times\bZ/p\bZ}_{n\text{ times}}.\]
  For $v = (x_1,\cdots,x_n)\in V$, and $\lambda\in\bF_p$, define $\lambda v = (\lambda x_1,\cdots,\lambda x_n)$ where multiplication in the coordinates is defined modulo $p$.
  \begin{enumerate}
    \item Show that $V$ with scalar multiplication as defined above is an $\bF_p$-vector space.
    \item Show that a function $\varphi:V\to V$ is a group homomorphism if and only if it is $\bF_p$-linear.
    \item Show that $\Aut(V)\cong GL_n(\bF_p)$.  (\textit{Hint:} You may cite Proposition 1 from HW5.)
    \item Let $p$ be a prime number and $G$ a group of order $p^2$.  What are the possible values for for $|\Aut(G)|$?  (Use the classification of groups of order $p^2$ and HW\#5 3(d).)
  \end{enumerate}
  \item We can apply part (5) as follows.  Let $G$ be a group of order $63 = 3^2*7$ and suppose that there is a normal subgroup $P\unlhd G$ of order 9.  We will show that $G$ is abelian.
  \begin{enumerate}
    \item Construct an injective map $G/C_G(P)\to\Aut P$.  (\textit{Hint:} Since $P$ consider the action of $G$ on $P$ by conjugation).
    \item Use 5(d) and Lagrange's theorem to show that $C_G(P) = G$.  Conclude that $G$ is abelian.  (\textit{Hint:} HW6\#2b may be helpful).
  \end{enumerate}
  \item Let's finish by computing the automorphism group of a $D_8$.
  \begin{enumerate}
    \item For $n\in\bZ$, define a homomorphism $\iota:D_{2n}\to D_{4n}$ on the generators of $D_{2n}$ by sending $\iota(r)=r^2$ and $\iota(s)=s$.  Show that $\iota$ is injective and its image is a normal subgroup of $D_{4n}$.  We abuse notation by saying $D_{2n}\unlhd D_{4n}$.
    \item Show that $|\Aut(D_8)|\le 8$.  (\textit{Hint}: If $\varphi:D_8\to D_8$ is an isomorphism, how many options are there for $\varphi(r)$.  What about for $\varphi(s)$?)
    \item By part (a), $D_{16}$ acts on $D_8$ by conjugation.  Use the associated permutation representation to prove $\Aut(D_8) \cong D_8$.  (\textit{Hint:} This last part requires a couple of steps.  Rather than have parts (d),(e),(f),..., let's see if you can follow your nose!  If you get stuck you can always ask for hints on the discord.)
  \end{enumerate}
\end{enumerate}
\end{document}
