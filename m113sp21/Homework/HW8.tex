\documentclass[11pt]{article}
\usepackage[top = 1in, bottom = 1in, left =1in, right = 1in]{geometry}
\usepackage{graphicx}
\usepackage{amsmath}
\usepackage{tabu}
\usepackage{amssymb}
\usepackage{amsmath}
\usepackage{etoolbox}
\usepackage{xcolor}
\usepackage{amsthm}
\usepackage{tikz-cd}
\usepackage{tikz}
\usepackage{tkz-graph}
\usepackage{seqsplit}
\usepackage{ulem}
\usepackage{tabularx}
\AtBeginEnvironment{proof}{\color{blue}}
\newtheorem{theorem}{Theorem}
\newtheorem{proposition}{Proposition}
\newtheorem{lemma}{Lemma}
\newtheorem*{facts}{Fact}
\newtheorem*{remark}{Remark}
\newtheorem{corollary}{Corollary}
\newtheorem{definition}{Definition}
\newtheorem*{hint}{Hint}
\usepackage{enumerate}
\usepackage{hyperref}
\usepackage{fancyhdr}\pagestyle{fancy}
\newcommand{\la}{\langle}
\newcommand{\ra}{\rangle}
\newcommand{\tors}{\mathrm{tors}}
\newcommand{\ab}{\mathrm{ab}}
\newcommand{\Aut}{\operatorname{Aut}}
\newcommand{\Inn}{\operatorname{Inn}}
\newcommand{\Out}{\operatorname{Out}}
\newcommand{\im}{\operatorname{im}}
\newcommand{\lcm}{\operatorname{lcm}}
\newcommand{\ch}{\operatorname{char}}

%Math blackboard:
\newcommand{\bC}{\mathbb{C}}
\newcommand{\bF}{\mathbb{F}}
\newcommand{\bN}{\mathbb{N}}
\newcommand{\bQ}{\mathbb{Q}}
\newcommand{\bR}{\mathbb{R}}
\newcommand{\bS}{\mathbb{S}}
\newcommand{\bZ}{\mathbb{Z}}

%Math caligraphy
\newcommand{\cA}{\mathcal{A}}
\newcommand{\cC}{\mathcal{C}}
\newcommand{\cK}{\mathcal{K}}
\newcommand{\cM}{\mathcal{M}}
\newcommand{\cO}{\mathcal{O}}

%Greek blackboard font:
\newcommand{\bmu}{\mbox{$\raisebox{-0.59ex}
  {$l$}\hspace{-0.18em}\mu\hspace{-0.88em}\raisebox{-0.98ex}{\scalebox{2}
  {$\color{white}.$}}\hspace{-0.416em}\raisebox{+0.88ex}
  {$\color{white}.$}\hspace{0.46em}$}{}}

\lhead{University of California, Berkeley}
\rhead{Math 113, Spring 2021}

\begin{document}
\begin{center}
\Large {Homework Assignment 8}\\
\small {Due Friday, March 19}
\end{center}
Recall the following important Lemma from the March 11th lecture.
\begin{lemma}\label{LemmaA}
  Let $G$ be a finite group, and $H\unlhd G$ a normal subgroup.  Let $P\le H$ be a Sylow $p$ subgroup of $H$.  If $P\unlhd H$ then $P\unlhd G$.
\end{lemma}
We noted in class that this feels like a normal Sylow subgroup is somehow \textit{strongly} normal, in such a way that we get transitivity of normal subgroups.  The following definition makes this precise.
\begin{definition}[Characteristic Subgroups]
  A subgroup $H\le G$ is called \textit{characteristic} in $G$ if for every automorphism $\varphi\in\Aut G$, we have $\varphi(H) = H$.  This is denoted by $H\ch G$.
\end{definition}
\begin{enumerate}
  \item{
  Let's prove some basic facts about characteristic subgroups and use them to prove Lemma \ref{LemmaA}.
  \begin{enumerate}
    \item{
    Show that characteristic subgroups are normal.  That is, if $H\ch G$ then $H\unlhd G$.
    }
    \item{
    Let $H\le G$ be the unique subgroup of $G$ of a given order.  Then $H\ch G$.
    }
    \item{
    Let $K\ch H$ and $H\unlhd G$, then $K\unlhd G$.  (This is the transitivity statement alluded to, and justifies the feeling that a characteristic subgroup is somehow \textit{strongly normal}).
    }
    \item{
    Let $G$ be a finite group and $P$ a Sylow $p$-subgroup of $G$.  Show that $P\unlhd G$ if and only if $P\ch G$.
    }
    \item{
    Put all this together to deduce Lemma \ref{LemmaA}.
    }
  \end{enumerate}
  }
\end{enumerate}
Sylow's theorem and some of the work you did last week makes it easy to prove Cauchy's theorem:
\begin{theorem}[Cauchy's Theorem]
  Let $G$ be a finite group and $p$ a prime number dividing the order of $G$.  Show that $G$ has an element of order $p$.
\end{theorem}
\begin{enumerate}
  \setcounter{enumi}{1}
  \item{
  \begin{enumerate}
    \item Prove the following strong version of Cauchy's theorem:  Suppose $G$ is a finite group of order $n$, and that $p$ a prime number such that $p^d|n$ for some $d\ge0$.  Prove that $G$ has a subgroup $H$ of order $p^d$.
    \item Deduce Cauchy's theorem as a special case of part (a).
  \end{enumerate}
  }
  \item{
  Let $G$ be a group of order $p^2q$ for primes $p\not=q$.  We will show that $G$ always has a nontrivial \textit{normal} Sylow subgroup.
  \begin{enumerate}
    \item Suppose $p>q$.  Show that $G$ has a normal subgroup of order $p^2$.
    \item Suppose $q>p$.  Show that either $G$ has a normal subgroup of order $q$, or else $G\cong A_4$.
    \item Explain why a group of order $p^2q$ for primes $p\not=q$ can never be simple.
  \end{enumerate}
  }
  \item{In class we've alluded many times to the fact that if $G$ is an abelian group of order $pq$ for primes $p\not=q$, then $G\cong Z_{pq}$.  Let's prove it.
  \begin{enumerate}
    \item{
    Let $x,y\in G$ be two elements of finite order and suppose that $xy=yx$.  Conclude that $|xy|$ divides the least common multiple of $|x|$ and $|y|$.
    }
    \item{
    Let $G$ be an abelian group of order $pq$ for primes $p<q$.  Use Cauchy's theorem and part (a) to conclude that $G$ is cyclic.  (This completes the argument from class about groups of order $pq$).
    }
  \end{enumerate}
  }
  \item{
  Next lets poke and prod $GL_2(\bF_p)$.
  \begin{enumerate}
    \item{
    Recall the order of $GL_2(\bF_p)$ from HW5 problem 3(d).  What is the maximal $p$ divisor of $|GL_2(\bF_p)|$?
    }
    \item{
    The subset of \textit{upper triangular matrices} of $GL_2(\bF_p)$ is:
    \[T = \left\{\begin{pmatrix}a & b\\0 & d\end{pmatrix}\in GL_2(\bF_p)\right\}.\]
    The subset of \textit{strictly upper triangular matrices} is:
    \[\overline T = \left\{\begin{pmatrix}1 & b\\0 & 1\end{pmatrix}\in GL_2(\bF_p)\right\}.\]
    Show that $T$ and $\overline T$ are subgroups of $GL_2(\bF_p)$.  We will see that they are not normal.
    }
    \item{
    Show that $\overline T$ is a Sylow $p$-subgroup of $GL_2(\bF_p)$ and of $T$.
    }
    \item{
    Show that $GL_2(\bF_p)$ has $p+1$ Sylow $p$-subgroups.
    }
    \item{
    Prove that $T$ is not normal in $GL_2(\bF_p)$.  (Hint: use Lemma \ref{LemmaA}).
    }
  \end{enumerate}
  }
  \item Prove that a group of order 200 cannot be simple.
  \item{
  Let $G_1,G_2,\cdots,G_n$ be groups.  Show that:
  \[Z(G_1\times G_2\times\cdots\times G_n) = Z(G_1)\times Z(G_2)\times\cdot\times Z(G_n).\]
  Conclude that a product of groups is abelian if and only if the factors are.
  }
\end{enumerate}
Let's finish with an important cancellation lemma for direct products.
\begin{lemma}\label{cancellation}
  Let $M,M',N,N'$ groups, and suppose $M\times N\cong M'\times N'$.  If $M$ and $M'$ are finite and $M\cong M'$ then $N\cong N'$.
\end{lemma}
\begin{enumerate}
  \setcounter{enumi}{7}
  \item{
  Let's explore and prove Lemma \ref{cancellation}.  It is actually more subtle then you might think.
  \begin{enumerate}
    \item{
    You will need to make use of the following fact, so we prove it first.  If $G_1,G_2$ are groups and $H_i\unlhd G_i$ for $i=1,2$.  Then under the usual identifications, $H_1\times H_2\unlhd G_1\times G_2$ and:
    \[(G_1\times G_2)/(H_1\times H_2)\cong(G_1/H_1)\times(G_2/H_2).\]
    }
    \item{
    Give an example to show that Lemma \ref{cancellation} is not true without the finiteness assumption.  (Hint: Let $G$ a nontrivial group and $M = G\times G\times G\times\cdots$ an infinite product of copies of $G$).
    }
    \item{
    Identify $M\times N$ and $M'\times N'$ as the same group $G$.  Show that if either  $M'\cap N = 1$, or if $M\cap N'=1$ then Lemma \ref{cancellation} holds.  (Hint: 2nd isomorphism theorem).
    }
    \item{
    Prove Lemma \ref{cancellation} by induction on $|M|$.  (Hint: The base case is easy (why?).  For the general case, notice that if $H = M\cap N'$ or $K = M'\cap N$ are trivial, we are done by part (b).  Otherwise, try manipulating $G/(H\times K)$ to apply induction).
    }
  \end{enumerate}
  }
\end{enumerate}
\end{document}
