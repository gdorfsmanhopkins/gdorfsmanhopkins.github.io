\documentclass[11pt]{article}
\usepackage[top = 1in, bottom = 1in, left =1in, right = 1in]{geometry}
\usepackage{graphicx}
\usepackage{amsmath}
\usepackage{tabu}
\usepackage{amssymb}
\usepackage{amsmath}
\usepackage{mathrsfs}
\usepackage{etoolbox}
\usepackage{xcolor}
\usepackage{amsthm}
\usepackage{tikz-cd}
\usepackage{tikz}
\usepackage{tkz-graph}
\usepackage{seqsplit}
\usepackage{ulem}
\usepackage{tabularx}
\AtBeginEnvironment{proof}{\color{blue}}
\newtheorem{theorem}{Theorem}
\newtheorem{proposition}[theorem]{Proposition}
\newtheorem{lemma}[theorem]{Lemma}
\newtheorem*{facts}{Facts}
\newtheorem*{remark}{Remark}
\newtheorem{corollary}[theorem]{Corollary}
\newtheorem{definition}[theorem]{Definition}
\newtheorem{defProp}[theorem]{Definition/Proposition}

\newtheorem*{hint}{Hint}
\usepackage{enumerate}
\usepackage{hyperref}
\usepackage{fancyhdr}\pagestyle{fancy}
\newcommand{\la}{\langle}
\newcommand{\ra}{\rangle}
\newcommand{\tors}{\mathrm{tors}}
\newcommand{\ab}{\mathrm{ab}}
\newcommand{\Aut}{\operatorname{Aut}}
\newcommand{\Inn}{\operatorname{Inn}}
\newcommand{\Out}{\operatorname{Out}}
\newcommand{\im}{\operatorname{im}}
\newcommand{\lcm}{\operatorname{lcm}}
\newcommand{\ch}{\operatorname{char}}
\newcommand{\maps}{\operatorname{Maps}}

%Math blackboard:
\newcommand{\bC}{\mathbb{C}}
\newcommand{\bF}{\mathbb{F}}
\newcommand{\bH}{\mathbb{H}}
\newcommand{\bN}{\mathbb{N}}
\newcommand{\bQ}{\mathbb{Q}}
\newcommand{\bR}{\mathbb{R}}
\newcommand{\bS}{\mathbb{S}}
\newcommand{\bZ}{\mathbb{Z}}

%Math caligraphy
\newcommand{\cA}{\mathcal{A}}
\newcommand{\cC}{\mathcal{C}}
\newcommand{\cK}{\mathcal{K}}
\newcommand{\cM}{\mathcal{M}}
\newcommand{\cO}{\mathcal{O}}

%Math scripts:
\newcommand{\sC}{\mathscr{C}}
\newcommand{\sP}{\mathscr{P}}

%Mathfrak:
\newcommand{\fJ}{\mathfrak{J}}
\newcommand{\fN}{\mathfrak{N}}
\newcommand{\fm}{\mathfrak{m}}
\newcommand{\fp}{\mathfrak{p}}
\newcommand{\fq}{\mathfrak{q}}

%Greek blackboard font:
\newcommand{\bmu}{\mbox{$\raisebox{-0.59ex}
  {$l$}\hspace{-0.18em}\mu\hspace{-0.88em}\raisebox{-0.98ex}{\scalebox{2}
  {$\color{white}.$}}\hspace{-0.416em}\raisebox{+0.88ex}
  {$\color{white}.$}\hspace{0.46em}$}{}}

\lhead{University of California, Berkeley}
\rhead{Math 113, Spring 2021}

\begin{document}
\begin{center}
  \Large {Takehome Assigment 3}\\
  \small {Due Monday, April 26}
\end{center}
In this assignment we establish some basic facts about prime and maximal ideals in \textit{commutative unital} rings.  In this assignment \textbf{all rings will be commutative rings with identity.}
\begin{enumerate}
  \item{
  Let $\varphi:R\to S$ be a homomorphism between commutative unital rings with $\varphi(1_R)=1_S$.
  \begin{enumerate}
    \item{
    Let $\fq\subseteq S$ be a prime ideal.  Show that $\varphi^{-1}(\fq)$ is a prime ideal of $R$.
    }
    \item{
    Suppose $\varphi$ is surjective, and $\fm\subseteq S$ is a maximal ideal.  Show that $\varphi^{-1}(\fm)$ is a maximal ideal of $R$.
    }
    \item{
    Give a counterexample to part (b) if $\varphi$ is not surjective.
    }
  \end{enumerate}
  }
  \item{
  In class we defined the ring of fractions for a \textit{good multiplicative subset} of a ring, i.e., a subset of $R$ which contains no zero divisors and is closed under multiplication.  If $R$ is a unital ring, then one can define this slightly more generally.  We define a subset $S\subseteq R$ to a be \textit{multiplicative subset} if it is closed under multiplication and contains $1$.  In this exercise we will describe the ring of fractions $S^{-1}R$.
  \begin{enumerate}
    \item{
    Consider the subset $\{(a,b):a\in R,b\in S\}\subseteq R\times R$.  Prove that:
    \[(a_1,b_1)\sim(a_2,b_2)\text{ if there exits }t\in S\text{ such that }t(a_1b_2-b_1a_2)=0,\]
    is an equivalence relation on $R$.  The equivalence class of $(a,b)$ will be denoted $\frac{a}{b}$.  Explain why if $S$ contains no zero divisors, this is the same equivalence relation as the one defined in class.
    }
    \item{
    Let $S^{-1}R = \{\frac{a}{b}:a\in R,b\in S\}$ be the set of equivalence classes of the relation described above.  Define addition and multiplication on $S^{-1}R$ by the rules:
    \begin{eqnarray*}
      \frac{a_1}{b_1}+\frac{a_2}{b_2} &=& \frac{a_1b_2+a_2b_1}{b_1b_2}\\
      \frac{a_1}{b_1}\times\frac{a_2}{b_2} &=& \frac{a_1a_2}{b_1b_2}.
    \end{eqnarray*}
    Show that these rules make $S^{-1}R$ into a commutative ring with identity.  (You must first show that they are well defined.  Then show that the ring axioms are satisfied)
    }
    \item{
    Define $\iota:R\to S^{-1}R$ by the rule $\iota(r) = \frac{r}{1}$.  Show that $\iota$ is a ring homomorphism, that $\iota(1_R)=1_{S^{-1}R}$ and that if $s\in S\subseteq R$, the $\iota(s)$ is a unit in $S^{-1}R$.   Prove also that $\iota$ is injective if and only if $S$ contains no zero divisors (or zero),
    }
    \item{
    Show that $S^{-1}R$ satisfies the following \textit{universal property}.  For any commutative unital ring $A$, and ring homomorphisms $\varphi:R\to A$ such that $\varphi(s)\in A^\times$ for every $s\in S$, there is a unique homomorphism $\tilde\varphi:S^{-1}R\to A$ such that $\tilde\varphi\circ\iota = \varphi$.
    \[
    \begin{tikzcd}
      S^{-1}R\ar[dr,dotted,"\tilde\varphi"]&\\
      R\ar[u,"\iota"]\ar[r,swap,"\varphi"]&A.
    \end{tikzcd}
    \]
    Deduce that there is a bijection:
    \[\{\text{Homomorphisms }\varphi:R\to A\text{ such that elements of }S\text{ map to }A^\times\}\]
    \[\Updownarrow\]
    \[\{\text{Homomorphisms }\tilde\varphi:S^{-1}R\to A\}.\]
    }
    \item{
    Let $r\in R$ be nonzero and consider the multiplicative set $S = \{1,r,r^2,r^3,\cdots\}$.  Define $R[1/r]:=S^{-1}R$.  Show that $R[1/r]=0$ if and only if $r$ is nilpotent.
    }
  \end{enumerate}
  }
  \item{
  In this exercise we calculate the intersection of all the prime ideals in a commutative unital ring $R$.
  \begin{enumerate}
    \item{
    Show that the element 0 is contained in every ideal of $R$.
    }
    \item{
    Let $r$ be a nilpotent element of $R$.  Show that $r$ is contained in every prime ideal of $R$.
    }
    \item{
    Conversely, suppose $r$ is not nilpotent.  Show that there is some prime ideal not containing $r$.  Deduce that:
    \[\fN(R) = \bigcap_{\fp\subseteq R\text{ prime}}\fp.\]
    (Hint: To find such a prime ideal, try applying 1(a) and 2(e) to the map $\iota:R\to R[1/r]$.)
    }
    \item{
    Deduce that the intersection of all the prime ideals in an integral domain is the 0 ideal.
    }
    \item{
    Suppose that $r$ is in the intersection of all the prime ideals of $R$.  Show that $1-ry\in R^\times$ for every $y\in R$.  (We will see below that the converse is not true in general, but that we can characterize all elements satisfying this property).
    }
  \end{enumerate}
  }
  \item{
  In this exercise we calculate the intersection of all the maximal ideals in a commutative unital ring $R$.  Given a ring $R$, we define the \textit{Jacobson radical} of $R$ to be:
  \[\fJ(R) = \bigcap_{\fm\subseteq R\text{ maximal}}\fm.\]
  \begin{enumerate}
    \item{
    Show that $\fN(R)\subseteq\fJ(R)$.
    }
    \item{
    Show that an element $r\in R$ is a unit if and only if it is not contained in any maximal ideal.
    }
    \item{
    Suppose $\fm$ is a maximal ideal and $r\in R\setminus\fm$.  Compute the ideal $(\fm,r)$ generated by $\fm$ and $r$.
    }
    \item{
    Prove that the condition from 3(e) actually characterizes elements in the \textit{Jacobson Radical}!  That is, prove that $r\in\fJ(R)$ if and only if $1-ry\in R^\times$ for every $y\in R$.  (Parts (b) and (c) might help!)
    }
  \end{enumerate}
  }
\end{enumerate}
\end{document}
