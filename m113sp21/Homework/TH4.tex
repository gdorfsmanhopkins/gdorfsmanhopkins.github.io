\documentclass[11pt]{article}
\usepackage[top = 1in, bottom = 1in, left =1in, right = 1in]{geometry}
\usepackage{graphicx}
\usepackage{amsmath}
\usepackage{tabu}
\usepackage{amssymb}
\usepackage{amsmath}
\usepackage{mathrsfs}
\usepackage{etoolbox}
\usepackage{xcolor}
\usepackage{amsthm}
\usepackage{tikz-cd}
\usepackage{tikz}
\usepackage{tkz-graph}
\usepackage{seqsplit}
\usepackage{ulem}
\usepackage{tabularx}
\usepackage{comment}
\AtBeginEnvironment{proof}{\color{blue}}
\newtheorem{theorem}{Theorem}
\newtheorem{proposition}[theorem]{Proposition}
\newtheorem{lemma}[theorem]{Lemma}
\newtheorem*{facts}{Facts}
\newtheorem*{remark}{Remark}
\newtheorem{corollary}[theorem]{Corollary}
\newtheorem{definition}[theorem]{Definition}
\newtheorem{defProp}[theorem]{Definition/Proposition}

\newtheorem*{hint}{Hint}
\usepackage{enumerate}
\usepackage{hyperref}
\usepackage{fancyhdr}\pagestyle{fancy}
\newcommand{\la}{\langle}
\newcommand{\ra}{\rangle}
\newcommand{\tors}{\mathrm{tors}}
\newcommand{\ab}{\mathrm{ab}}
\newcommand{\Aut}{\operatorname{Aut}}
\newcommand{\Inn}{\operatorname{Inn}}
\newcommand{\Out}{\operatorname{Out}}
\newcommand{\im}{\operatorname{im}}
\newcommand{\lcm}{\operatorname{lcm}}
\newcommand{\ch}{\operatorname{char}}
\newcommand{\maps}{\operatorname{Maps}}

%Math blackboard:
\newcommand{\bC}{\mathbb{C}}
\newcommand{\bF}{\mathbb{F}}
\newcommand{\bH}{\mathbb{H}}
\newcommand{\bN}{\mathbb{N}}
\newcommand{\bQ}{\mathbb{Q}}
\newcommand{\bR}{\mathbb{R}}
\newcommand{\bS}{\mathbb{S}}
\newcommand{\bZ}{\mathbb{Z}}

%Math caligraphy
\newcommand{\cA}{\mathcal{A}}
\newcommand{\cC}{\mathcal{C}}
\newcommand{\cK}{\mathcal{K}}
\newcommand{\cM}{\mathcal{M}}
\newcommand{\cO}{\mathcal{O}}

%Math scripts:
\newcommand{\sC}{\mathscr{C}}
\newcommand{\sP}{\mathscr{P}}

%Mathfrak:
\newcommand{\fJ}{\mathfrak{J}}
\newcommand{\fN}{\mathfrak{N}}
\newcommand{\fm}{\mathfrak{m}}
\newcommand{\fp}{\mathfrak{p}}
\newcommand{\fq}{\mathfrak{q}}

%arrows
\newcommand{\into}{\hookrightarrow}

%Greek blackboard font:
\newcommand{\bmu}{\mbox{$\raisebox{-0.59ex}
  {$l$}\hspace{-0.18em}\mu\hspace{-0.88em}\raisebox{-0.98ex}{\scalebox{2}
  {$\color{white}.$}}\hspace{-0.416em}\raisebox{+0.88ex}
  {$\color{white}.$}\hspace{0.46em}$}{}}

\lhead{University of California, Berkeley}
\rhead{Math 113, Spring 2021}

\begin{document}
\begin{center}
  \Large {Takehome Assigment 4}\\
  \small {Due Friday, May 14}
\end{center}
In this assignment unless otherwise indicated, \textbf{all rings are unital rings} (although they will not necessarily be commutative), and \textbf{all homomorphisms are unital homomorphisms}.
\begin{enumerate}
  \item{
  Let's begin by exploring unit groups. Recall that if $R$ is a (unital) ring, then $R^\times$ is the set of units, endowed with a group structure given by multiplication in $R$ (cf. HW10 Problem 2).
  \begin{enumerate}
    \item{
    Let $\varphi:R\to S$ be a (unital) homomorphism of rings.  Show that if $r\in R^\times$ then $\varphi(r)\in S^\times$.  Give a counterexample where $\varphi$ is not unital.
    }
    \item{
    Show that the restriction of $\varphi$ to $R^\times$ is a group homomorphism $\varphi^\times: R^\times\to S^\times$, which is injective if $\varphi$ is.
    }
    \item{
    The analogous statement does not hold for $\varphi$ surjective.  Give an example of a surjective (unital) homomorphism $\varphi:R\to S$, but such that the induced map on unit groups $\varphi^\times:R^\times\to S^\times$ is not surjective.
    }
    \item{
    Let $\varphi:R\to S$ be a surjective (unital) homomorphism of \textit{commutative} rings, and suppose that $\ker\varphi\subseteq\fJ(R)$ (where $\fJ$ is the \textit{Jacobson radical} from TH3 Problem 4).  Prove that the induced map $\varphi^\times:R^\times\to S^\times$ is surjective.
    }
  \end{enumerate}
  }
  \item{
  In elementary calculus one often uses the fact that a polynomial of degree $n$ over the real numbers has at most $n$ roots.  This turns out to be true over any field!  For this problem we fix a field $F$.
  \begin{enumerate}
    \item{
    Let $f(x)\in F[x]$, and suppose that $f(a)=0$ for some $a\in F$.  Show that $(x-a)$ divides $f(x)$.  (Hint: recall that $F[x]$ is Euclidean domain).
    }
    \item{
    Let $f(x)\in F[x]$, and suppose $f(a_1)=f(a_2)=\cdots=f(a_r)=0$, for $a_i\in F$ all distinct.  Prove by induction that $(x-a_1)(x-a_2)\cdots(x-a_r)$ divides $f(x)$.
    }
    \item{
    Deduce from part (b) that if the degree of $f(x)$ is $n$, then $f(x)$ has at most $n$-roots.
    }
    \item{
    As a corollary, let $f(x)\in F[x]$ be a polynomial of degree 2 or 3.  Prove that $F[x]/(f(x))$ is a field if and only if $f(x)$ has no roots in $F$.  Give an example to show this is not true for polynomials of degree 4.
    }
  \end{enumerate}
  }
  \item{
  We used many times this semester, (for example when classifying groups like in HW9) that if $p$ is prime, the unit group $(\bZ/p\bZ)^\times$ is cyclic of order $p-1$, and more generally that if $p$ is an odd prime then $(\bZ/p^n\bZ)^\times$ is cyclic.  But if you've been paying close attention you should notice that we haven't actually proved that fact yet!  So let's come full circle and deduce this fact as a consequence of Problems 1 and 2.
  \begin{enumerate}
    \item{
    Consider a finite abelian group $G = Z_{n_1}\times Z_{n_2}\times\cdots\times Z_{n_k}$ in invariant factor form (so that $n_k|n_{k-1}|\cdots|n_2|n_1$).  Prove that if $k\not=1$ then there are more than $n_k$ elements in $G$ whose order divides $n_k$.
    }
    \item{
    Let $F$ be a field, and let $G\le F^\times$ be a finite subgroup of the unit group of $F$.  Prove that $G$ is cyclic.  Deduce that $(\bZ/p\bZ)^\times\cong Z_{p-1}$.  (\textit{Hint:} Can you express the condition in (a) in terms of solutions to a polynomial in $F[x]$?)
    }
  \end{enumerate}
  Let's now deduce the analogous result of $(\bZ/p^n\bZ)^\times$ for an odd prime $p$.
  \begin{enumerate}
    \setcounter{enumii}{2}
    \item{
    Let $G$ be a finite abelian group and suppose all it's Sylow subgroups are cyclic.  Show that $G$ is cyclic.
    }
    \item{
    Show that the surjection of rings $\pi:\bZ/p^n\bZ\to\bZ/p\bZ$ induces a surjection of groups $\pi^\times:(\bZ/p^n\bZ)^\times\to(\bZ/p\bZ)^\times$ whose kernel has order $p^{n-1}$.  (Hint: use 1(d) and Lagrange's theorem).
    }
    \item{
    Deduce from part (d) that for all primes $p\not=q$, the Sylow $q$-subgroups of $(\bZ/p^n\bZ)^\times$ are cyclic.
    }
  \end{enumerate}
  It remains to show that the Sylow $p$-subgroup of $(\bZ/p^n\bZ)^\times$ is cyclic.  We will need the following technical result.
  \begin{enumerate}
    \setcounter{enumii}{5}
    \item{
    Let $p$ be an odd prime.  Prove the following identities by induction on $k$.
    \begin{itemize}
      \item{$(1+p)^{p^k}\equiv 1\mod p^{k+1}$}
      \item{$(1+p)^{p^k}\equiv1+p^{k+1}\mod p^{k+2}$}
    \end{itemize}
    }
    \item{
    Deduce from part (f) that the Sylow $p$-subgroup of $(\bZ/p^n\bZ)^\times$ is cyclic.  (\textit{Hint:} Prove $(1+p)$ is a generator!).  Conclude that that $(\bZ/p^n\bZ)^\times\cong Z_{p^{n-1}(p-1)}$.
    }
  \end{enumerate}
  By TH2 we know abstractly that for any $n$, $(\bZ/n\bZ)^\times$ can be expressed as a product of cyclic groups.  In the case that $n$ is odd we can now compute exactly which ones!
  \begin{enumerate}
    \setcounter{enumii}{7}
    \item{
    Fix an odd integer $n$ with prime factorization $p_1^{\alpha_1}\cdots p_t^{\alpha_t}$.  Express $(\bZ/n\bZ)^\times$ as a product of cyclic groups in terms of the prime factorization.  (\textit{Note:} Putting this into invariant factor form depends on the factorizations of the $p_i-1$, which can vary wildly as the primes do, so don't worry about doing that).
    }
  \end{enumerate}
  }
\end{enumerate}
\textbf{Congratulations!!}  We've covered a ton of material and done a ton of problems this semester.  \textbf{Good work!}
\end{document}
