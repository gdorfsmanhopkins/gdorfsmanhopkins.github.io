\documentclass[11pt]{article}
\usepackage[top = 1in, bottom = 1in, left =1in, right = 1in]{geometry}
\usepackage{graphicx}
\usepackage{amsmath}
\usepackage{tabu}
\usepackage{amssymb}
\usepackage{amsmath}
\usepackage{etoolbox}
\usepackage{xcolor}
\usepackage{amsthm}
\usepackage{tikz-cd}
\usepackage{tikz}
\usepackage{tkz-graph}
\usepackage{seqsplit}
\usepackage{ulem}
\usepackage{tabularx}
\AtBeginEnvironment{proof}{\color{blue}}
\newtheorem{theorem}{Theorem}
\newtheorem{proposition}{Proposition}
\newtheorem{lemma}{Lemma}
\newtheorem*{facts}{Fact}
\newtheorem*{remark}{Remark}
\newtheorem{corollary}{Corollary}
\newtheorem{definition}[theorem]{Definition}
\newtheorem{question}[theorem]{Question}
\newtheorem*{hint}{Hint}
\usepackage{enumerate}
\usepackage{hyperref}
\usepackage{fancyhdr}\pagestyle{fancy}
\newcommand{\la}{\langle}
\newcommand{\ra}{\rangle}
\newcommand{\tors}{\mathrm{tors}}
\newcommand{\ab}{\mathrm{ab}}
\newcommand{\Aut}{\operatorname{Aut}}
\newcommand{\Inn}{\operatorname{Inn}}
\newcommand{\im}{\operatorname{im}}
\newcommand{\lcm}{\operatorname{lcm}}
\newcommand{\ch}{\operatorname{char}}

%Math blackboard:
\newcommand{\bC}{\mathbb{C}}
\newcommand{\bF}{\mathbb{F}}
\newcommand{\bN}{\mathbb{N}}
\newcommand{\bQ}{\mathbb{Q}}
\newcommand{\bR}{\mathbb{R}}
\newcommand{\bS}{\mathbb{S}}
\newcommand{\bZ}{\mathbb{Z}}

%Math caligraphy
\newcommand{\cA}{\mathcal{A}}
\newcommand{\cC}{\mathcal{C}}
\newcommand{\cK}{\mathcal{K}}
\newcommand{\cM}{\mathcal{M}}
\newcommand{\cO}{\mathcal{O}}

%Greek blackboard font:
\newcommand{\bmu}{\mbox{$\raisebox{-0.59ex}
  {$l$}\hspace{-0.18em}\mu\hspace{-0.88em}\raisebox{-0.98ex}{\scalebox{2}
  {$\color{white}.$}}\hspace{-0.416em}\raisebox{+0.88ex}
  {$\color{white}.$}\hspace{0.46em}$}{}}

\lhead{University of California, Berkeley}
\rhead{Math 113, Spring 2022}

\begin{document}
\begin{center}
\Large {Homework Assignment 10}\\
\small {Due Friday, April 15}
\end{center}
Recall the following important Lemma from the April 5th lecture.
\begin{lemma}\label{LemmaA}
  Let $G$ be a finite group, and $H\unlhd G$ a normal subgroup.  Let $P\le H$ be a Sylow $p$ subgroup of $H$.  If $P\unlhd H$ then $P\unlhd G$.
\end{lemma}
We noted in class that this feels like a normal Sylow subgroup is somehow \textit{strongly} normal, in such a way that we get transitivity of normal subgroups.  The following definition makes this precise.
\begin{definition}[Characteristic Subgroups]
  A subgroup $H\le G$ is called \textit{characteristic} in $G$ if for every automorphism $\varphi\in\Aut G$, we have $\varphi(H) = H$.  This is denoted by $H\ch G$.
\end{definition}
\begin{enumerate}
  \item{
  Characteristic subgroups will turn out to be the right type of subgroup to guarantee a transitive property for normality.  In this exercise we will establish basic facts about characteristic subgroups, and use it prove Lemma \ref{LemmaA}.  First we will make sure this definition is even necessary.
  \begin{enumerate}
    \item{
    Give an example to show that the relation \textit{is a normal subgroup of} is not transitive.  That is, give a chain of subgroups $H\le K\le G$ such that $H\unlhd K$ and $K\unlhd G$ but $H\not\unlhd K$.
    }
    \item{
    Show that characteristic subgroups are normal.  That is, if $H\ch G$ then $H\unlhd G$.
    }
    \item{
    Let $H\le G$ be the unique subgroup of $G$ of a given order.  Then $H\ch G$.
    }
    \item{
    Let $K\ch H$ and $H\unlhd G$, then $K\unlhd G$.  (This is the transitivity statement alluded to, and justifies the feeling that a characteristic subgroup is somehow \textit{strongly normal}).
    }
    \item{
    Let $G$ be a finite group and $P$ a Sylow $p$-subgroup of $G$.  Show that $P\unlhd G$ if and only if $P\ch G$.
    }
    \item{
    Put all this together to deduce Lemma \ref{LemmaA}.
    }
  \end{enumerate}
  }
  \item Next let's poke and prod $GL_2(\bF_p)$.
  \begin{enumerate}
    \item{
    Recall the order of $GL_2(\bF_p)$ from HW6 problem 7(d).  What is the maximal $p$ divisor of $|GL_2(\bF_p)|$?
    }
    \item{
    The subset of \textit{upper triangular matrices} of $GL_2(\bF_p)$ is:
    \[T = \left\{\begin{pmatrix}a & b\\0 & d\end{pmatrix}\in GL_2(\bF_p)\right\}.\]
    The subset of \textit{strictly upper triangular matrices} is:
    \[\overline T = \left\{\begin{pmatrix}1 & b\\0 & 1\end{pmatrix}\in GL_2(\bF_p)\right\}.\]
    Show that $T$ and $\overline T$ are subgroups of $GL_2(\bF_p)$.  We will see that they are not normal.
    }
    \item{
    Show that $\overline T$ is a Sylow $p$-subgroup of $GL_2(\bF_p)$ and of $T$.
    }
    \item{
    Show that $N_{GL_2(\bF_p)}(\overline T) = T$.
    }
    \item{
    Show that $GL_2(\bF_p)$ has $p+1$ Sylow $p$-subgroups.
    }
    \item{
    Prove that $T$ is not normal in $GL_2(\bF_p)$.  (Hint: you could do this directly, or you could use Lemma \ref{LemmaA}).
    }
  \end{enumerate}
  \item Let's establish a few fundamentals about direct products.
  \begin{enumerate}
    \item{
  Suppose $M\cong M'$ and $N\cong N'$.  Show that $M\times N\cong M'\times N'$.
  }
   \item{
  Let $G_1,G_2,\cdots,G_n$ be groups.  Show that:
  \[Z(G_1\times G_2\times\cdots\times G_n) = Z(G_1)\times Z(G_2)\times\cdot\times Z(G_n).\]
  Conclude that a product of groups is abelian if and only if the factors are.
  }
  \end{enumerate}
  \end{enumerate}
  The notion of characteristic subgroups will allow us to compute automorphism groups of certain direct products.
  \begin{lemma}\label{autDecomp}
  Let $H$ and $K$ be finite groups whose orders are coprime.  Then \[\Aut(H\times K)\cong\Aut H\times \Aut K.\]
\end{lemma}
The following definition will be useful.
\begin{definition}
  Let $\varphi:G\to G'$ be a homomorphism, and let $H\le G$.  The \textit{restriction of} $\varphi$ \textit{to }$H$ is the map $\varphi|_H:H\to G'$ given by evaluating $\varphi$ on elements of $H$.
\end{definition}
Let's consider it obvious that $\varphi|_H$ is a homomorphism (why?), and so you may use this fact without proof.
\begin{enumerate}
  \setcounter{enumi}{3}
  \item{
  Let's study and prove Lemma \ref{autDecomp}.
  \begin{enumerate}
    \item{
    Give an example to show that the condition on the orders of $H$ and $K$ are necessary.  That is, give an example of an $H$ and $K$ whose order is not coprime, and where $\Aut(H\times K)\not\cong \Aut H\times\Aut K$.
    }
    \item{
    Let $G$ be a group and let $H\ch G$ be a \textit{characteristic subgroup}.  Fix any automorphism $\varphi\in\Aut G$. Show that $\varphi|_H$ is an automorphism of $H$.  (Hint: you must first show its image lands in $H$ so you can consider it as a map from $H$ to itself).
    }
    \item{
    With $H$ and $G$ as in part (a), show that the rule $\varphi\mapsto\varphi|_H$ is a homomorphism $\Aut G\to\Aut H$.
    }
    \item{
    Let $H,K$ be finite groups of coprime orders.  Show that $H$ and $K$ are characteristic in $H\times K$.
    }
    \item{
    With $H,K$ as in (c), construct an isomorphism $\Aut(H\times K)\to\Aut H\times\Aut K$.
    }
  \end{enumerate}
  }
\end{enumerate}
To understand groups, it is often useful to break them down into direct products.  The following theorem allows us to do this.
\begin{theorem}[Recognition Theorem for Direct Products]\label{recog}
Suppose $G$ is a group and $H,K\unlhd G$ are normal subgroups such that $H\cap K=1$.  Then $HK\cong H\times K$.  In particular, if we further assume $HK=G$ then $G\cong H\times K$.
\end{theorem}
(Recall from the 2nd Isomorphism Theorem that because $H,K\unlhd G$ then $HK\le G$ is a subgroup).
\begin{enumerate}
\setcounter{enumi}{4}
\item Let's prove Theorem \ref{recog}
\begin{enumerate}
\item Let $G$ be a group and $H,K\le G$ subgroups.  Fix $g\in HK$.  Show that there are precisely $|H\cap K|$ distinct ways to write $g = hk$ for $h\in H$ and $k\in K$.  Deduce that if $H\cap K=1$ then $g$ can be written uniquely as a product $hk$ for $h\in H$ and $k\in K$.
\item Suppose that $H,K\unlhd G$ are normal subgroups, and that $H\cap K=1$.  Show that for any $h\in H$ and $k\in K$, $hk = kh$.  (\textit{Hint:} show that the commutator $[h,k]=k^{-1}h^{-1}kh$ is in both $H$ and $K$).
\item Deduce that the function $\varphi:H\times K\to HK$ given by $\varphi(h,k) = hk$ is an isomorphism, thereby proving Theorem \ref{recog}.
\end{enumerate}
One way to state the fundamental theorem of finite abelian groups is that a finite abelian group is a product of cyclic groups.  A consequence of Theorem \ref{recog} is that an abelian group is the product of it's Sylow subgroups, which reduces the proof of the fundamental theorem to the case of $p$-groups.
\item{Let $G$ be an abelian group
\begin{enumerate}
    \item{
    Explain why $G$ has a \textit{unique} Sylow $p$-subgroup for each prime $p$.  This justifies our use of the word \textit{the} in the following.
    }
    \item{
    Suppose $G$ has order $p^\alpha q^\beta$ for distinct primes $p$ and $q$.  Let $P$ be the Sylow $p$-subgroup, and $Q$ the Sylow $q$-subgroup.  Show that $G\cong P\times Q$.
    }
    \item{
    In general the prime factorization of $|G|$ is $p_1^{\alpha_1}p_2^{\alpha_2}\cdots p_t^{\alpha_t}$.  Show by induction on $t$ that $G$ is the product of its Sylow subgroups.  Explicitly, this means that if $P_i$ is the Sylow $p_i$-subgroup for $i= 1,\cdots,t$, then
    \[G\cong P_1\times P_2\times\cdots\times P_t.\]
    }
    \end{enumerate}
   }
\end{enumerate}
\end{document}
