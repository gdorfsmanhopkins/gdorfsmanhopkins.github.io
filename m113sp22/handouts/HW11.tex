\documentclass[11pt]{article}
\usepackage[top = 1in, bottom = 1in, left =1in, right = 1in]{geometry}
\usepackage{graphicx}
\usepackage{amsmath}
\usepackage{tabu}
\usepackage{amssymb}
\usepackage{amsmath}
\usepackage{etoolbox}
\usepackage{xcolor}
\usepackage{amsthm}
\usepackage{tikz-cd}
\usepackage{tikz}
\usepackage{tkz-graph}
\usepackage{seqsplit}
\usepackage{ulem}
\usepackage{tabularx}
\AtBeginEnvironment{proof}{\color{blue}}
\newtheorem{theorem}{Theorem}
\newtheorem{proposition}{Proposition}
\newtheorem{lemma}[theorem]{Lemma}
\newtheorem*{facts}{Fact}
\newtheorem*{remark}{Remark}
\newtheorem{corollary}{Corollary}
\newtheorem{definition}[theorem]{Definition}
\newtheorem{question}[theorem]{Question}
\newtheorem*{hint}{Hint}
\usepackage{enumerate}
\usepackage{hyperref}
\usepackage{fancyhdr}\pagestyle{fancy}
\newcommand{\la}{\langle}
\newcommand{\ra}{\rangle}
\newcommand{\tors}{\mathrm{tors}}
\newcommand{\ab}{\mathrm{ab}}
\newcommand{\Aut}{\operatorname{Aut}}
\newcommand{\Inn}{\operatorname{Inn}}
\newcommand{\im}{\operatorname{im}}
\newcommand{\lcm}{\operatorname{lcm}}
\newcommand{\ch}{\operatorname{char}}

%Math blackboard:
\newcommand{\bC}{\mathbb{C}}
\newcommand{\bF}{\mathbb{F}}
\newcommand{\bN}{\mathbb{N}}
\newcommand{\bQ}{\mathbb{Q}}
\newcommand{\bR}{\mathbb{R}}
\newcommand{\bS}{\mathbb{S}}
\newcommand{\bZ}{\mathbb{Z}}

%Math caligraphy
\newcommand{\cA}{\mathcal{A}}
\newcommand{\cC}{\mathcal{C}}
\newcommand{\cK}{\mathcal{K}}
\newcommand{\cM}{\mathcal{M}}
\newcommand{\cO}{\mathcal{O}}

%Greek blackboard font:
\newcommand{\bmu}{\mbox{$\raisebox{-0.59ex}
  {$l$}\hspace{-0.18em}\mu\hspace{-0.88em}\raisebox{-0.98ex}{\scalebox{2}
  {$\color{white}.$}}\hspace{-0.416em}\raisebox{+0.88ex}
  {$\color{white}.$}\hspace{0.46em}$}{}}

\lhead{University of California, Berkeley}
\rhead{Math 113, Spring 2022}

\begin{document}
\begin{center}
\Large {Homework Assignment 11}\\
\small {Due Friday, April 22}
\end{center}
This assignment will fill in many details from lecture, and do a few hands on classifications.  To begin we will confirm that the semidirect product is indeed a group.  First recall the definition.
\begin{definition}\label{semiD}
  Let $H,K$ be groups, and $\varphi:K\to\Aut(H)$ a group homomorphism.  Denote the induced action of $K$ on $H$ by:
  \[k\cdot h = \varphi(k)(h).\]
  The \textit{semidirect product} of $H$ and $K$ with respect to $\varphi$ is the set $H\rtimes K = \{(h,k):h\in H,k\in K\}$, where multiplication is defined by the rule:
  \[(h_1,k_1)(h_2,k_2) = (h_1(k_1\cdot h_2),k_1k_2).\]
\end{definition}
\begin{enumerate}
  \item{
  Let's make sure that $H\rtimes K$ is a group.
  \begin{enumerate}
    \item Show that $(1,1)\in H\rtimes K$ is the identity.  (Remember you have to check both sides).
    \item Show that $(h,k)^{-1} = (k^{-1}\cdot h^{-1},k^{-1})$. (As above, you have to check both sides).
    \item Prove that multiplication is associative.
  \end{enumerate}
  }
  \item{Consider again the setup of Definition \ref{semiD}.  Let's prove some basic properties about $G = H\rtimes K$.
  \begin{enumerate}
  \item Show that the subset $\{(h,1):h\in H\}\subseteq G$ is a subgroup isomorphic to $H$.  Similarly, show that $\{(1,k):k\in K\}\subseteq G$ is a subgroup isomorphic to $K$.  In what follows we identify $H$ and $K$ with these subgroups, and write $H,K\le G$.
  \item Prove that $H\cap K=\{1_G\}$.
  \item Show that $H\unlhd G$ and $G/H\cong K$.
   \end{enumerate}}
\end{enumerate}
In HW 10 Problem 5 we proved the \textit{Recognition Theorem for Direct Products} (HW10 Theorem 3).  There is an analogous result for semidirect products, and in fact you already did most of the work.  Let's state the result.
\begin{theorem}[Recognition Theorem for Semidirect Products]\label{recog}
Suppose $G$ is a group and $H,K\le G$ are subgroups.  Suppose that $H\unlhd G$ is normal, and that $H\cap K = 1$.  Then \[HK\cong H\rtimes_\varphi K,\]
where $\varphi:K\to\Aut(H)$ corresponds to the action of $K$ on $H$ by conjugation (in $G$).  In particular, if $HK=G$ then $G\cong H\rtimes_\varphi K$.
\end{theorem}
\begin{enumerate}
\setcounter{enumi}{2}
\item Prove Theorem \ref{recog} by showing that function $H\rtimes_{\varphi} K\to HK$ defined by the rule $(h,k)\mapsto hk$ is an isomorphim.  (\textit{Note:} Bijectivity should follow from HW10 Problem 5(a), so the main verification is that it is a homomorphism).
 \item A lot of studying semidirect products comes down to enumerating and classifying homomorphisms.
 \begin{enumerate}
    \item{
    Show that giving a homomorphism $Z_n\to G$ is the same as selecting an element $g\in G$ with $|g|$ dividing $n$.  That is, give a bijection between the following sets:
    \[\left\{
    \begin{array}{c}
      \text{Homomorphisms}\\
      Z_n\to G
    \end{array}\right\}
    \Longleftrightarrow
    \left\{
    \begin{array}{c}
      \text{Elements }g\in G\\
      \text{where }|g|\text{ divide }n
    \end{array}
    \right\}
    \]
    }
    \item{
    If $p$ is prime show that giving a \textit{nontrivial} map $Z_p\to G$ is the same as choosing an element of order $p$ in $G$. (Note: the trivial map is the one that sends every element to the identity of $G$).
    }
    \item{
    Show that giving a homomorphism $Z_{n_1}\times\cdots\times Z_{n_r}\to G$ is the same as chosing elements $g_1,\cdots,g_r\in G$ such that all the $g_i$ commute with eachother and each $|g_i|$ divides $n_i$.
    }
    \item{
    Suppose $G$ is abelian and $p$ is prime.  Describe the set of homomorphisms $Z_p\times Z_p\to G$ as a subset of $G\times G$.
    }
  \end{enumerate}

\end{enumerate}
Any homomorphism $\varphi:K\to\Aut H$ allows us to build a semidirect product $H\rtimes_\varphi K$.  An interesting question is when different maps give us isomorphic semidirect products.  In class we stated and used a special case of the following lemma.
\begin{lemma}\label{leftSemidirect}
  Let $\varphi,\psi:K\to\Aut H$ be two homomorphisms, and suppose they differ by an automorphism of $K$.  That is, suppose there is some $\gamma\in\Aut(K)$ such that $\psi\circ\gamma = \varphi$:
  \[
  \begin{tikzcd}
    K\ar[dr,"\varphi"]\ar[dd,swap,"\gamma"] & \\
     & \Aut H\\
    K\ar[ur,swap, "\psi"] &
  \end{tikzcd}
  \]
  Then $H\rtimes_\varphi K\cong H\rtimes_\psi K$.
\end{lemma}
One could ask if this is the only thing that could allow different $\varphi$ to give different semidirect products.  The answer would be no, as the following lemma shows.
\begin{lemma}\label{rightSemidirect}
  Let $\varphi,\psi:K\to\Aut H$ be two homomorphisms, and suppose they are conjugate in $\Aut H$.  Explicitely, suppose there is some $\alpha\in\Aut H$, corresponding to the inner automorphism $\sigma_\alpha:\beta\mapsto \alpha\beta\alpha^{-1}$, and suppose that $\psi = \sigma_\alpha\circ\varphi$:
  \[
  \begin{tikzcd}
    &\Aut H\ar[dd,"\sigma_\alpha"]\\
    K\ar[ur,"\varphi"]\ar[dr,swap,"\psi"]\\
    &\Aut H
  \end{tikzcd}
  \]
  Then $H\rtimes_\varphi K\cong H\rtimes_\psi K$.
\end{lemma}
\begin{enumerate}
  \setcounter{enumi}{4}
  \item{Lemmas \ref{leftSemidirect} and \ref{rightSemidirect} say that if we alter $\varphi$ by an automorphism of $K$, or an inner automorphism of $\Aut H$, (or both), we don't change the semidirect products.  Let's prove this.
  \begin{enumerate}
    \item{
    Consider the setup of Lemma \ref{leftSemidirect}.  Show that the map:
    \begin{eqnarray*}
      H\rtimes_\varphi K&\longrightarrow& H\rtimes_\psi K\\
      (h,k)&\mapsto&(h,\gamma(k))
    \end{eqnarray*}
    is an isomorphism, thereby proving the lemma.
    }
    \item{
    Consider the setup of Lemma \ref{rightSemidirect}.  Show that the map:
    \begin{eqnarray*}
      H\rtimes_\varphi K&\longrightarrow&H\rtimes_\psi K\\
      (h,k) &\mapsto& (\alpha(h),k)
    \end{eqnarray*}
    is an isomorphism, thereby proving the lemma.  (Notice that $\alpha\in\Aut H$ is an automorphism of $H$, wheras $\sigma_\alpha$ is an automorphism of $\Aut H$, given by conjugation by $\alpha$.  In unweildy notation, this says $\sigma_\alpha\in\Aut(\Aut H)$.)
    }
    \item{
    Now suppose $\varphi,\psi:K\to\Aut H$ are two homomorphisms, and suppose there is an automorphism $\gamma\in\Aut K$ and an inner automorphism $\sigma\in\Inn(\Aut(H))$ such that the following diagram commutes:
    \[
    \begin{tikzcd}
      K\ar[r,"\varphi"]\ar[d,swap,"\gamma"]&\Aut H\ar[d,"\sigma"]\\
      K\ar[r,"\psi"]&\Aut H.
    \end{tikzcd}
    \]
    That is, $\sigma\circ\varphi = \psi\circ\gamma$.  Then $H\rtimes_\varphi K\cong H\rtimes_\psi K$.  (Hint: This should follow formally from Lemmas \ref{leftSemidirect} and \ref{rightSemidirect}, so you shouldn't have to do any lengthy computations).
    }
  \end{enumerate}
  }
  \item We've seen 5 groups of order 12: $Z_{12},Z_6\times Z_2, D_{12},A_4$, and a nontrivial semidirect product $Z_3\rtimes Z_4$ where the generator of $Z_4$ acts on $Z_3$ by inverting elements.  Let's prove this is all of them!
  \begin{enumerate}
  \item Let $G$ be a group of order 12.  Show that if $G\not\cong A_4$, then $G\cong Q\rtimes P$ where $P$ is a Sylow 2-subgroup and $Q$ is a Sylow 3-subgroup.  (\textit{Hint}: \textbf{(Sylow 3)} and the Theorem \ref{recog} should help).
  \item Show that there is only one abelian and one nonabelian semidirect product $Z_3\rtimes Z_4$ up to isomorphism.
  \item Show that there is only one abelian and one nonabelian semidirect prodcut $Z_3\rtimes (Z_2\times Z_2)$ up to isomorphism.  (You might need Lemma \ref{leftSemidirect}).
  \item Put together parts (a)-(c) to deduce that there are exactly 5 groups of order 12 up to isomorphism.  Of the semidirect products classified in parts (b) and (c), which one corresponds to $D_{12}$?
  \end{enumerate}
    \item{
  In this problem we classify all groups of order 75 up to isomorphism.  (There should be 3 total).
  \begin{enumerate}
    \item{List all the abelian groups of order 75 using the fundamental theorem of finite abelian groups.}
    \item{Prove that a group of order 75 is isomorphic to $P\rtimes Q$ where $P$ is a Sylow 5-subgroup and $Q$ is a Sylow 3-subgroup.}
    \item{Prove that if a group of order 75 has a \textit{cyclic} Sylow 5-subgroup, then it is abelian.}
    \item{Show that there is a unique nonabelian group of order 75 up to isomorphism.  (\textit{Hint:} Show that 3 is a maximal 3-divisor of $|GL_2(\bF_5)|$.  Then use Sylow's theorems and 5(c).)}
  \end{enumerate}
  }
\end{enumerate}
\end{document}
