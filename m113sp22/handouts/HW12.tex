\documentclass[11pt]{article}
\usepackage[top = 1in, bottom = 1in, left =1in, right = 1in]{geometry}
\usepackage{graphicx}
\usepackage{amsmath}
\usepackage{tabu}
\usepackage{amssymb}
\usepackage{amsmath}
\usepackage{etoolbox}
\usepackage{xcolor}
\usepackage{amsthm}
\usepackage{tikz-cd}
\usepackage{tikz}
\usepackage{tkz-graph}
\usepackage{seqsplit}
\usepackage{ulem}
\usepackage{tabularx}
\AtBeginEnvironment{proof}{\color{blue}}
\newtheorem{theorem}{Theorem}
\newtheorem{proposition}{Proposition}
\newtheorem{lemma}[theorem]{Lemma}
\newtheorem*{facts}{Fact}
\newtheorem*{remark}{Remark}
\newtheorem{corollary}{Corollary}
\newtheorem{definition}[theorem]{Definition}
\newtheorem{question}[theorem]{Question}
\newtheorem*{hint}{Hint}
\usepackage{enumerate}
\usepackage{hyperref}
\usepackage{fancyhdr}\pagestyle{fancy}
\newcommand{\la}{\langle}
\newcommand{\ra}{\rangle}
\newcommand{\tors}{\mathrm{tors}}
\newcommand{\ab}{\mathrm{ab}}
\newcommand{\Aut}{\operatorname{Aut}}
\newcommand{\Inn}{\operatorname{Inn}}
\newcommand{\im}{\operatorname{im}}
\newcommand{\lcm}{\operatorname{lcm}}
\newcommand{\ch}{\operatorname{char}}

%Math blackboard:
\newcommand{\bC}{\mathbb{C}}
\newcommand{\bF}{\mathbb{F}}
\newcommand{\bN}{\mathbb{N}}
\newcommand{\bQ}{\mathbb{Q}}
\newcommand{\bR}{\mathbb{R}}
\newcommand{\bS}{\mathbb{S}}
\newcommand{\bZ}{\mathbb{Z}}

%Math caligraphy
\newcommand{\cA}{\mathcal{A}}
\newcommand{\cC}{\mathcal{C}}
\newcommand{\cK}{\mathcal{K}}
\newcommand{\cM}{\mathcal{M}}
\newcommand{\cO}{\mathcal{O}}

\newcommand{\maps}{\operatorname{Maps}}


%Greek blackboard font:
\newcommand{\bmu}{\mbox{$\raisebox{-0.59ex}
  {$l$}\hspace{-0.18em}\mu\hspace{-0.88em}\raisebox{-0.98ex}{\scalebox{2}
  {$\color{white}.$}}\hspace{-0.416em}\raisebox{+0.88ex}
  {$\color{white}.$}\hspace{0.46em}$}{}}

\lhead{University of California, Berkeley}
\rhead{Math 113, Spring 2022}

\begin{document}
\begin{center}
\Large {Homework Assignment 12}\\
\small {Due Saturday, April 30}
\end{center}
\begin{enumerate}
  \item{
  Let $R$ be a ring.  Recall that for $a\in R$ we denote the \textit{additive} inverse of $a$ by $-a$.  Establish the following identities.
  \begin{enumerate}
    \item{$(-a)b = a(-b) = -ab$}
    \item{$(-a)(-b) = ab$}
    \item{If $1\in R$ then $(-1)a = -a$.}
    \item{Suppose $R$ is an integral domain.  Show that if $a^2=1$ then $a=\pm1$.  (\textit{Recall} A ring is an integral domain if it is commutative, with multiplicative identity $1\not=0$, and such that if $ab=0$ then $a=0$ or $b=0$}
  \end{enumerate}
  }
  \item{
  Let $R$ be a ring with $1\not=0$.
  \begin{enumerate}
    \item{
    Let $R^\times\subseteq R$ be the set of units of $R$.  Show that $R^\times$ is a group under the multiplication operation of $R$.
    }
    \item{
    Suppose that $a\in R$ is a zero divisor.  Show that $a\notin R^\times$.
    }
    \end{enumerate}
    }
  \item{
  Let $R$ be a commutative ring.  An element $r\in R$ is called \textit{nilpotent} if there exists a positive $n$ such that $r^n=0$.  A commutative ring is called \textit{reduced} if it has no nonzero nilpotent elements.
  \begin{enumerate}
    \item{
    Show that a nilpotent element of a ring is either 0 or a zero divisor.
    }
    \item{
    Give an example of a ring with a nonzero nilpotent element.
    }
    \item{
    Show that the sum of nilpotent elements is nilpotent.
    }
    \item{
    Suppose $r$ is nilpotent.  Show that $rx$ is nilpotent for all $x\in R$.  (\textit{Note}, in future terminology, (c) and (d) prove that the set of nilpotent elements is an \textit{ideal} of $R$, which we will call the \textit{nilradical}).
    }
    \item{
    Suppose $R$ is a commutative ring with $1\not=0$, and suppose $r\in R$ is nilpotent.  Show that $1+r\in R^\times$.
    }
  \end{enumerate}
  }
  \item{
  Let $R$ be ring, and $X$ any set.  Define
  \[\maps(X,R) = \{f:X\to R\text{ }|\text{ }f\text{ is a function}\}.\]
  Define binary operations $+$ and $\times$ as follows.
  \[(f+g)(x) = f(x) + g(x)\hspace{30pt}(f\times g)(x) = f(x)g(x).\]
  \begin{enumerate}
    \item{
    Show that $\maps(X,R)$ is a ring.
    }
    \item{
    Suppose $R$ is commutative, show that $\maps(X,R)$ is too.
    }
    \item{
    Suppose $R$ is unital, show that $\maps(X,R)$ is too.
    }
    \item{
    Suppose $R$ is reduced (defined in Problem 3), show that $\maps(X,R)$ is too.
    }
    \item{
    Give an example to show that even if $R$ is a field, $\maps(X,R)$ need not be.
    }
    \item{
    Give an example to show that even if $R$ is an integral domain, $\maps(X,R)$ need not be.
    }
  \end{enumerate}
  }
  \item{
  Let $A$ be an abelian group (with binary operation $+$).  Define the \textit{endomorphism ring} of $A$ as follows:
  \[\operatorname{End}(A) = \{f:A\to A\text{ }|\text{ }f\text{ is a homomorpism}\}.\]
  Give $\operatorname{End}(A)$ 2 binary operations $+$ and $\times$ as follows:
  \[(f+g)(a) = f(a)+g(a)\hspace{20pt}(f\times g)(a) = f(g(a)).\]
  \begin{enumerate}
    \item{
    Prove that $\operatorname{End}(A)$ is a ring.
    }
    \item{
    Prove that $(\operatorname{End}(A))^\times\cong\Aut(A)$.
    }
    \item{
    Let $E$ be an elementary abelian $p$-group of order $p^n$.  Show that $\operatorname{End}(E)\cong M_n(\bF_p)$, where we give the latter the operations matrix addition and multiplication.  Conclude that $M_n(\bF_p)$ is a ring and that $M_n(\bF_p)^\times = GL_n(\bF_p)$.  (You may use Proposition 1 from HW6, after which this should be completely formal.)
    }
  \end{enumerate}
  }
\end{enumerate}
Had we been not been in lockdown on Thursday, we would have encountered the following definition:
\begin{definition}
Let $R$ be a ring.  A subset $S\subseteq R$ is called a \textit{subring} if it is a subgroup under addition, and also if $a,b\in S$ then $ab\in S$.
\end{definition}
\begin{enumerate}
\setcounter{enumi}{5}
\item{
\begin{enumerate}
	\item{Let $R$ be a ring and $S\subseteq R$ a subring.  Show that $S$ is a ring.}
      \item{
    Let $\{S_i\subseteq R\}$ be a nonempty collection of subrings of $R$.  Show that $\bigcap_i S_i$ is a subring of $R$.
    }
    \item{
    Suppose $S$ is a subring of $R$, and $R$ is a subring of $T$.  Show that $S$ is a subring of $T$.
    }
  \end{enumerate}
  }
  \item{
  Let $D$ be an integer which is not a perfect square.  One forms a \textit{quadratic integer ring}
  \[\bZ[\sqrt{D}] = \{a+b\sqrt{D}:a,b\in\bZ\},\]
  with the standard notions of addition and multiplication.  We will see that the structure of this ring depends heavily on $D$.
  \begin{enumerate}
  \item{Show that $\bZ[\sqrt{D}]$ is a ring.  (\textit{Hint:} You could do this directly, or observe it is a subring of a well known field, and leverage the previous exercise).}
  \item{Define the norm of a quadratic integer to be
  \[N(a+b\sqrt{D}) = (a+b\sqrt{D})(a-b\sqrt{D}).\]
  Prove that the norm gives a map $N:\bZ[\sqrt{D}]\to\bZ$ satisfying $N(xy) = N(x)N(y)$.
    }
    \item{Let $x\in\bZ[\sqrt{D}]$.  Show $x$ is a unit if and only if $N(x) = \pm1$.}
    \item{Use part (c) to establish the following.
    \begin{enumerate}
    	\item Let $i=\sqrt{-1}$.  Show $(\bZ[i])^\times = \{\pm1,\pm i\}$.
	\item Let $D<-2$.  Show $(\bZ[\sqrt{D}])^\times = \{\pm1\}$.
	\item Show $|(\bZ[\sqrt{2}])^\times|=\infty.$
	\end{enumerate}
	}
\end{enumerate}
}
\end{enumerate}
\end{document}
