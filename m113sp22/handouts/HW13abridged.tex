\documentclass[11pt]{article}
\usepackage[top = 1in, bottom = 1in, left =1in, right = 1in]{geometry}
\usepackage{graphicx}
\usepackage{amsmath}
\usepackage{tabu}
\usepackage{amssymb}
\usepackage{amsmath}
\usepackage{etoolbox}
\usepackage{xcolor}
\usepackage{amsthm}
\usepackage{tikz-cd}
\usepackage{tikz}
\usepackage{tkz-graph}
\usepackage{seqsplit}
\usepackage{ulem}
\usepackage{tabularx}
\AtBeginEnvironment{proof}{\color{blue}}
\newtheorem{theorem}{Theorem}
\newtheorem{proposition}{Proposition}
\newtheorem{lemma}[theorem]{Lemma}
\newtheorem*{facts}{Fact}
\newtheorem*{remark}{Remark}
\newtheorem{corollary}{Corollary}
\newtheorem{definition}[theorem]{Definition}
\newtheorem{question}[theorem]{Question}
\newtheorem*{hint}{Hint}
\usepackage{enumerate}
\usepackage{hyperref}
\usepackage{fancyhdr}\pagestyle{fancy}
\newcommand{\la}{\langle}
\newcommand{\ra}{\rangle}
\newcommand{\tors}{\mathrm{tors}}
\newcommand{\ab}{\mathrm{ab}}
\newcommand{\Aut}{\operatorname{Aut}}
\newcommand{\Inn}{\operatorname{Inn}}
\newcommand{\im}{\operatorname{im}}
\newcommand{\lcm}{\operatorname{lcm}}
\newcommand{\ch}{\operatorname{char}}

%Math blackboard:
\newcommand{\bC}{\mathbb{C}}
\newcommand{\bF}{\mathbb{F}}
\newcommand{\bN}{\mathbb{N}}
\newcommand{\bQ}{\mathbb{Q}}
\newcommand{\bR}{\mathbb{R}}
\newcommand{\bS}{\mathbb{S}}
\newcommand{\bZ}{\mathbb{Z}}

%Math caligraphy
\newcommand{\cA}{\mathcal{A}}
\newcommand{\cC}{\mathcal{C}}
\newcommand{\cK}{\mathcal{K}}
\newcommand{\cM}{\mathcal{M}}
\newcommand{\cO}{\mathcal{O}}

%Math Frakture
\newcommand{\fN}{\mathfrak{N}}

\newcommand{\maps}{\operatorname{Maps}}


%Greek blackboard font:
\newcommand{\bmu}{\mbox{$\raisebox{-0.59ex}
  {$l$}\hspace{-0.18em}\mu\hspace{-0.88em}\raisebox{-0.98ex}{\scalebox{2}
  {$\color{white}.$}}\hspace{-0.416em}\raisebox{+0.88ex}
  {$\color{white}.$}\hspace{0.46em}$}{}}

\lhead{University of California, Berkeley}
\rhead{Math 113, Spring 2022}

\begin{document}
\begin{center}
\Large {Homework Assignment 13}\\
\small {Due Saturday, May 7}
\end{center}
\begin{enumerate}
  \item{In class we proved a cancellation law for integral domains.  We can actually say something a bit stronger (and quite useful).  Let $R$ be a ring and $a,b,c\in R$.  Suppose that a is not zero or a zero divisor, and that $ab = ac$.  Prove $b=c$.}
  \item{
  Let $R$ and $S$ be rings and $\varphi:R\to S$ a ring homomorphism.
  \begin{enumerate}
    \item{Show that $\im\varphi$ is a subring of $S$.}
    \item{Show that $\ker\varphi$ is a (two-sided) ideal of $R$.}
    \item{Suppose $J\subseteq S$ is an ideal.  Show that $\varphi^{-1}(J)$ is an ideal of $R$.}
    \item{Suppose $R$ and $S$ are unital rings with \textit{nonzero} identities $1_R$ and $1_S$ respectively.  Prove that if $\varphi(1_R)\not=1_S$ then $\varphi(1_R)$ is either zero, or a zero divisor in $S$.}
    \item{Deduce that if $S$ is an integral domain and $\varphi$ is nonzero then $\varphi(1_R)=1_S$.  (\textit{Remark:} many authors require rings to be unital, and also require ring homomorphisms to take the identity to the identity.)}
  \end{enumerate}
  }
  \setcounter{enumi}{5}
  \item{
  Let $R$ be a commutative ring with $1\not=0$.
  \begin{enumerate}
    \item{
    Fix $a\in R$.  Show that $(a)=R$ if and only if $a\in R^\times$.
    }
    \item{
    Fix $a,b\in R$, and suppose that $a$ is not a zero divisor.  Show that $(a)=(b)$ if and only if $a = ub$ for some unit $u\in R^\times$.
    }
    \item{
    Let $I$ be any ideal.  Show that $I=R$ if and only if $I$ contains a unit $u\in R^\times$.
    }
    \item{
    Prove that $R$ is a field if and only if the only ideals in $R$ are $(0)$ and $R$ itself.
    }
  \end{enumerate}
  }
  \item{
  Let $R$ be a commutative ring.  The \textit{nilradical} of $R$ is $\fN(R)=\{r\in R:r$ is nilpotent$\}$.  By HW12 Problem 3 we know that $\fN(R)$ is an ideal of $R$.
  \begin{enumerate}
    \item{Show that $R/\fN(R)$ is reduced.  This is often called the \textit{reduction of $R$,} and is denoted $R_{red}$.}
    \item{Compute $\fN(R)$ and $R_{red}$ for the following two rings.
    \begin{enumerate}
    \item $R = \bZ[x]/(x)^n$ for $n\ge2$.
    \item $R = \bZ/p^n\bZ$ for $n\ge2$.
    \end{enumerate}}
  \end{enumerate}
  }

\end{enumerate}
\end{document}
