\documentclass[11pt]{article}
\usepackage[top = 1in, bottom = 1in, left =1in, right = 1in]{geometry}
\usepackage{graphicx}
\usepackage{amsmath}
\usepackage{tabu}
\usepackage{amssymb}
\usepackage{amsmath}
\usepackage{etoolbox}
\usepackage{xcolor}
\usepackage{amsthm}
\usepackage{tikz-cd}
\usepackage{tikz}
\usepackage{tkz-graph}
\usepackage{seqsplit}
\usepackage{ulem}
\usepackage{tabularx}
\AtBeginEnvironment{proof}{\color{blue}}
\newtheorem{theorem}{Theorem}
\newtheorem{proposition}{Proposition}
\newtheorem{lemma}{Lemma}
\newtheorem*{facts}{Fact}
\newtheorem*{remark}{Remark}
\newtheorem{corollary}{Corollary}
\newtheorem{definition}{Definition}
\usepackage{enumerate}
\usepackage{hyperref}
\usepackage{fancyhdr}\pagestyle{fancy}
\newcommand{\la}{\langle}
\newcommand{\ra}{\rangle}
\newcommand{\tors}{\mathrm{tors}}
\newcommand{\ab}{\mathrm{ab}}
\newcommand{\Aut}{\operatorname{Aut}}
\newcommand{\Inn}{\operatorname{Inn}}
\newcommand{\im}{\operatorname{im}}
\newcommand{\lcm}{\operatorname{lcm}}
\newcommand{\ch}{\operatorname{char}}

%Math blackboard:
\newcommand{\bC}{\mathbb{C}}
\newcommand{\bF}{\mathbb{F}}
\newcommand{\bN}{\mathbb{N}}
\newcommand{\bQ}{\mathbb{Q}}
\newcommand{\bR}{\mathbb{R}}
\newcommand{\bS}{\mathbb{S}}
\newcommand{\bZ}{\mathbb{Z}}

%Math caligraphy
\newcommand{\cC}{\mathcal{C}}
\newcommand{\cK}{\mathcal{K}}
\newcommand{\cM}{\mathcal{M}}
\newcommand{\cO}{\mathcal{O}}

%Greek blackboard font:
\newcommand{\bmu}{\mbox{$\raisebox{-0.59ex}
  {$l$}\hspace{-0.18em}\mu\hspace{-0.88em}\raisebox{-0.98ex}{\scalebox{2}
  {$\color{white}.$}}\hspace{-0.416em}\raisebox{+0.88ex}
  {$\color{white}.$}\hspace{0.46em}$}{}}

\lhead{University of California, Berkeley}
\rhead{Math 113, Spring 2022}

\begin{document}
\begin{center}
\Large {Homework Assignment 3}\\
\small {Due Friday, February 11}
\end{center}
\begin{enumerate}
	\item{In class we developed the theory of the group $D_{12}$ of rigid symmetries of the regular hexagon (on bCourses: `Lecture 4' from 55:00 until the end).  In fact, everything we developed should go through almost exactly the same way for $D_{2n}$: the rigid symmetries of regular $n$-sided polygon, pictured below:
	\begin{center}
	   \begin{tikzpicture}
       \fill (-1.56366296494,1.24697960372) circle (0.1) node[anchor=south east] {\small $n$};
       \fill (0,2) circle (0.1) node[anchor = south west] {1};
       \fill (1.56366296494,1.24697960372) circle (0.1) node[anchor = south west] {2};
			 \fill (1.94985582436,-0.44504186791) circle (0.1) node [anchor = south west] {3};
			 \draw (-1.56366296494,1.24697960372) -- (0,2) --  (1.56366296494,1.24697960372) -- (1.94985582436,-0.44504186791);
			 \draw[dashed] (-1.94985582436,-0.44504186791) -- (-1.56366296494,1.24697960372);
			 \draw[dashed] (1.94985582436,-0.44504186791) -- (0.86776747823,-1.8019377358);
     \end{tikzpicture}
   \end{center}
   }
   \begin{enumerate}
     \item{
     Explain why $D_{2n}$ is a group under composition of symmetries.
     }
     \item{
     Show that there are exactly $2n$ rigid symmetries of the regular $n$-gon.
     }
     \item{
     Let $r$ be the rotation by $2\pi/n$ in the clockwise direction, and $s$ be the reflection along the vertical line going through the vertex labelled `1'.  Compute the elements of $D_{2n}$ in terms of $r$ and $s$ in the following steps:
     \begin{enumerate}
       \item{
       Compute the order of $r$ and $s$ (justifying your answers).
       }
       \item{
       Let $i_1,i_2\in\{0,1\}$ and $j_1,j_2\in\{0,1,\cdots,n-1\}$.  Show that:
       \[s^{i_1}r^{j_1}=s^{i_2}r^{j_2}\text{ if and only if }i_1=i_2\text{ and }j_1=j_2.\]
       }
       (Hint: You could first show $s\not=r^i$ for any $i$ using geometry.  The rest of the cases should follow from this and part (i) by using cancellation and 8(b).)
       \item{
       Conclude that $D_{2n} = \{s^ir^j|i=0,1$ and $j=0,1,\cdots,n-1\}$.  In particular, $r$ and $s$ generate $D_{2n}$.
       }
     \end{enumerate}
     }
     \item{
     Show that $rs = sr^{-1}$.  Deduce inductively from this that $r^ns = sr^{-n}$ for all $n$.
     }
   \end{enumerate}
   We now completely understand the algebraic structure of $D_{2n}$.  In particular, we know what every element looks like (in terms of $r$ and $s$) by (c), and we know how to multiply any two elements using the relation in part (d).  We summarize this by saying that $D_{2n}$ has the following presentation:
   \[D_{2n} = \la r,s | r^n = s^2 = 1, rs = sr^{-1}\ra.\]
   \begin{enumerate}
     \setcounter{enumii}{4}
     \item{
     Use this presentation to give an algebraic proof that every element which is not a power of $r$ has order 2.
    }
   \end{enumerate}
     \item The set $S_3$ has 6 elements.  Compute the order and cycle decomposition of each element.
   \newpage
   \item Some of the arguments in problem 1 used a connection between symmetries of polygons and permutations of the vertices.  Let's make this explicit!
   \begin{enumerate}
    \item Describe an injective homomorphism from $\varphi:D_{2n}\to S_n$ (you may describe this in words, but make sure to justify injectivity).
    \item In the map you described, what is the cycle decomposition of $\varphi(r)$ (where $r$ is the generator corresponding to clockwise rotation of the $n$-gon by $2\pi/n$)?
    \item Prove that $D_6\cong S_3$.
  \end{enumerate}

  \item No we important basic facts about group homomorphisms that we will use repeatedly throughout the course.  Let $G,H,K$ be groups, and let $\varphi:G\to H$ and $\psi:H\to K$ a homomorphisms.
  \begin{enumerate}
    \item Show that $\varphi(1_G) = 1_H$.
    \item Show that $\varphi(x^{-1}) = \varphi(x)^{-1}$ for all $x\in G$.
    \item Show that if $g\in G$ has finite order, then $|\varphi(g)|$ divides $|g|$.
    \item Show that if $\varphi$ is an isomorphism, then so is $\varphi^{-1}$.
    \item Show that if $\varphi$ is an isomorphism, $|\varphi(g)| = |g|$.
    \item Show that the composition $\psi\circ \varphi:G\to K$ is a homomorphism.
    \item Suppose $\varphi$ and $\psi$ are both isomorphisms.  Show that the composition $\psi\circ\varphi$ is as well.
    \item Conclude that the relation \textit{is isomorphic to} is an equivalence relation on the set of all groups.
    \end{enumerate}
  \item In this exercise we show that you can compute the order of a permutation from its cycle decomposition.
  \begin{enumerate}
    \item Let $G$ be a group.  Two elements $x,y\in G$ are called \textit{commuting elements} if $xy = yx$.  Show that if $x$ and $y$ are commuting elements, then $(xy)^n = x^ny^n$.
    \item Give a counterexample to part (a) if the chosen elements do not commute.
    \item Let $\sigma = (a_1,a_2,\cdots,a_r)\in S_n$ be an $r$-cycle.  Show that $|\sigma| = r$.
    \item Prove that the order of a permutation is the least common multiple of the lengths of the cycles in its cycle decomposition.  (Hint: You may freely use that disjoint cycles are commuting elements.  You may find it useful to establish that the product of nontrivial disjoint cycles is never 1).
  \end{enumerate}
  \item We suggested in class that if $A$ and $B$ are sets of the same cardinality, then their permutation groups $S_A$ and $S_B$ (defined in HW2\#5) are isomorphic.  Let's prove it.  To begin, fix a bijective function $\theta:A\to B$.
  \begin{enumerate}
    \item Let $f:A\to A$ be bijective.  Show that $\theta\circ f\circ \theta^{-1}:B\to B$ is bijective.  (Hint: what is its inverse?)
    \item Part (a) allows us to construct the following function:
    \begin{eqnarray*}
      S_A&\stackrel{\varphi}{\longrightarrow}&S_B\\
      f&\longmapsto&\theta\circ f\circ\theta^{-1}.
    \end{eqnarray*}
    Show that $\varphi$ is an isomorphism, thereby proving the result.  (Note: There are two parts to this.  You must show that $\varphi$ is bijective, and that it is a homomorphism.)
    \item{Use (a) and (b) to conclude that if $A$ be a finite set with $n$ elements, then $S_A\cong S_n$.}
  \end{enumerate}
\end{enumerate}
\end{document}
