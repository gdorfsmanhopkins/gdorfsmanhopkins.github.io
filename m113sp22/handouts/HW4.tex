\documentclass[11pt]{article}
\usepackage[top = 1in, bottom = 1in, left =1in, right = 1in]{geometry}
\usepackage{graphicx}
\usepackage{amsmath}
\usepackage{tabu}
\usepackage{amssymb}
\usepackage{amsmath}
\usepackage{etoolbox}
\usepackage{xcolor}
\usepackage{amsthm}
\usepackage{tikz-cd}
\usepackage{tikz}
\usepackage{tkz-graph}
\usepackage{seqsplit}
\usepackage{ulem}
\usepackage{tabularx}
\AtBeginEnvironment{proof}{\color{blue}}
\newtheorem{theorem}{Theorem}
\newtheorem{proposition}{Proposition}
\newtheorem{lemma}{Lemma}
\newtheorem*{facts}{Fact}
\newtheorem*{remark}{Remark}
\newtheorem{corollary}{Corollary}
\newtheorem{definition}{Definition}
\newtheorem*{hint}{Hint}
\usepackage{enumerate}
\usepackage{hyperref}
\usepackage{fancyhdr}\pagestyle{fancy}
\newcommand{\la}{\langle}
\newcommand{\ra}{\rangle}
\newcommand{\tors}{\mathrm{tors}}
\newcommand{\ab}{\mathrm{ab}}
\newcommand{\Aut}{\operatorname{Aut}}
\newcommand{\Inn}{\operatorname{Inn}}
\newcommand{\im}{\operatorname{im}}
\newcommand{\lcm}{\operatorname{lcm}}
\newcommand{\ch}{\operatorname{char}}

%Math blackboard:
\newcommand{\bC}{\mathbb{C}}
\newcommand{\bF}{\mathbb{F}}
\newcommand{\bN}{\mathbb{N}}
\newcommand{\bQ}{\mathbb{Q}}
\newcommand{\bR}{\mathbb{R}}
\newcommand{\bS}{\mathbb{S}}
\newcommand{\bZ}{\mathbb{Z}}

%Math caligraphy
\newcommand{\cA}{\mathcal{A}}
\newcommand{\cC}{\mathcal{C}}
\newcommand{\cK}{\mathcal{K}}
\newcommand{\cM}{\mathcal{M}}
\newcommand{\cO}{\mathcal{O}}

%Greek blackboard font:
\newcommand{\bmu}{\mbox{$\raisebox{-0.59ex}
  {$l$}\hspace{-0.18em}\mu\hspace{-0.88em}\raisebox{-0.98ex}{\scalebox{2}
  {$\color{white}.$}}\hspace{-0.416em}\raisebox{+0.88ex}
  {$\color{white}.$}\hspace{0.46em}$}{}}

\lhead{University of California, Berkeley}
\rhead{Math 113, Spring 2022}

\begin{document}
\begin{center}
\Large {Homework Assignment 4}\\
\small {Due Friday, February 18}
\end{center}
\begin{enumerate}
  \item In this exercise we study products of finite cyclic groups.  Recall that we denote by $Z_n$ the cyclic group of order $n$ (written multiplicatively).
  \begin{enumerate}
    \item Prove that $Z_2\times Z_2$ is not a cyclic group.
    \item Prove that $Z_2\times Z_3\cong Z_6$.  Conclude that $Z_2\times Z_3$ is a cyclic group.
  \end{enumerate}
  Those two examples really cover all the bases.  Use the intuition you gained from them to prove the following classification result.
  \begin{enumerate}
    \setcounter{enumii}{2}
    \item Show that $Z_n\times Z_m$ is cyclic if and only if $\gcd(n,m)=1$.  (Hint: recall that up to isomorphism there is only one cyclic group of order $N$ for every positive integer $N$).
  \end{enumerate}

  \item Let $G$ be a group and $H$ a \textit{nonempty} subset of $G$.  Let's introduce a few tricks to speed up testing if something is a subgroup.
  \begin{enumerate}
    \item \textit{(Subgroup Criterion)} Suppose that for all $x,y\in H$, $xy^{-1}\in H$.  Show that $H$ is a subgroup of $G$.
    \item \textit{(Finite Subgroup Criterion)} Show that if $H$ is finite and closed under multiplication, then $H$ is a subgroup of $G$.
  \end{enumerate}
	\item Let $G$ be a group.  Let $H,K\le G$ be two subgroups.
  \begin{enumerate}
    \item Show that the intersection $H\cap K$ is a subgroup of $G$.
    \item Give an example to show that the union $H\cup K$ need not be a subgroup of $G$.
    \item Show that $H\cup K$ is a subgroup of $G$ if and only if $H\subset K$ or $K\subset H$.
    \item Adjust your proof from part (a) to show that the intersection of an arbitrary collection of subgroups is a subgroup.  That is, let $\cA$ be a collection of subgroups of $G$.  Show that
    \[\bigcap_{H\in\cA}H\]
    is a subgroup of $G$.  This completes the proof that the subgroup generated by a subset is in fact a subgroup.
    \begin{hint}
      For part (d), the proof should be very similar to part (a), with only cosmetic modifications.  You won't need to use induction.  In fact, since $\cA$ is could in principle be uncountable, induction won't work without modifications (think about why this is).
    \end{hint}
  \end{enumerate}
    \item Given a homomorphism $\varphi:G\to H$, we obtain 2 important subgroups, one of $G$ and one of $H$.  They are called the \textit{kernel of $\varphi$} and \textit{image of $\varphi$} and are defined by the following rules:
  \begin{eqnarray*}
    \ker\varphi &=& \{g\in G:\varphi(g) = 1_H\},\\
    \operatorname{im}\varphi &=& \{h\in H:h =\varphi(g)\text{ for some }g\in G\}.
  \end{eqnarray*}
  \begin{enumerate}
    \item Show that $\ker\varphi$ is a subgroup of $G$.
    \item Show that $\im\varphi$ is a subgroup of $H$.
    \item{\textit{Important:} Show that $\varphi$ is injective if and only if $\ker\varphi = \{1_G\}$.  (This is an incredibly useful fact!)}
  \end{enumerate}
  \item The kernel has the following important generalization.  For $h\in H$ define the \textit{fiber over $h$} as
  \[\varphi^{-1}(h) = \{g\in G:\varphi(g) = h\}.\]
  This is sometimes also called the \textit{preimage of $h$}.  Observe that by definition, the kernel of $\varphi$ is the fiber over 1.
  \begin{enumerate}
    \item{Show that the fiber over $h$ is a subgroup if and only if $h=1_H$.}
    \item{Show that the \textit{nonempty} fibers of $\varphi$ form a partition of $G$.  (In particular, if $\varphi$ is surjective its fibers partition $G$.)}
    \item{Show that all nonempty fibers have the same cardinality.  (Hint: if $\varphi^{-1}(h)$ is nonempty, build a bijection between it and $\ker\varphi$.)  Observe that this generalizes 2(c).}
  \end{enumerate}
  \item Let $G$ be a group and $A$ a set, and suppose we are given homomorphism $\varphi:G\to S_A$.  Show that the rule:
  \[g\cdot a = \varphi(g)(a)\text{ for all }g\in G\text{ and }a\in A,\]
  describes a group action of $G$ on $A$, and further that the permutation representation of this action is $\varphi$ itself.
  \item Let $G$ be a group acting on a set $A$.  For an element $a\in A$, we define the \textit{stabilizer} of $a$ to be the collection of elements of $G$ that act trivially on $a$, that is:
  \[G_a: = \{g\in G: g\cdot a = a\}.\]
  The \textit{kernel} of the group action is the collection of elements of $G$ that act trivially on \textit{all of} $A$, that is:
  \[G_0:=\{g\in G:g\cdot a=a\text{ for all }a\in A\}.\]
  \begin{enumerate}
  \item Prove that $G_a$ and $G_0$ are subgroups of $G$.
  \item Prove that $G_0$ is equal to the kernel of the \text{permutation representation} associated to the action of $G$ on $A$.  (cf. Problem 4: This justifies the naming convention).
  \end{enumerate}
    \item For $n\ge 2$ let $G = S_n$ be the symmetric group equipped with it's natural action on $\Omega_n = \{1,2,\cdots,n\}$ by permutations.  For $i\in\Omega_n$, let $G_i = \{\sigma\in G|\sigma(i)=i\}$ be the stabilizer of $i$.  Describe an isomorphism between $G_i$ and $S_{n-1}$.

\end{enumerate}
\end{document}
