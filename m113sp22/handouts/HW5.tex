\documentclass[11pt]{article}
\usepackage[top = 1in, bottom = 1in, left =1in, right = 1in]{geometry}
\usepackage{graphicx}
\usepackage{amsmath}
\usepackage{tabu}
\usepackage{amssymb}
\usepackage{amsmath}
\usepackage{etoolbox}
\usepackage{xcolor}
\usepackage{amsthm}
\usepackage{tikz-cd}
\usepackage{tikz}
\usepackage{tkz-graph}
\usepackage{seqsplit}
\usepackage{ulem}
\usepackage{tabularx}
\AtBeginEnvironment{proof}{\color{blue}}
\newtheorem{theorem}{Theorem}
\newtheorem{proposition}{Proposition}
\newtheorem{lemma}{Lemma}
\newtheorem*{facts}{Fact}
\newtheorem*{remark}{Remark}
\newtheorem{corollary}{Corollary}
\newtheorem{definition}[theorem]{Definition}
\newtheorem{question}[theorem]{Question}
\newtheorem*{hint}{Hint}
\usepackage{enumerate}
\usepackage{hyperref}
\usepackage{fancyhdr}\pagestyle{fancy}
\newcommand{\la}{\langle}
\newcommand{\ra}{\rangle}
\newcommand{\tors}{\mathrm{tors}}
\newcommand{\ab}{\mathrm{ab}}
\newcommand{\Aut}{\operatorname{Aut}}
\newcommand{\Inn}{\operatorname{Inn}}
\newcommand{\im}{\operatorname{im}}
\newcommand{\lcm}{\operatorname{lcm}}
\newcommand{\ch}{\operatorname{char}}

%Math blackboard:
\newcommand{\bC}{\mathbb{C}}
\newcommand{\bF}{\mathbb{F}}
\newcommand{\bN}{\mathbb{N}}
\newcommand{\bQ}{\mathbb{Q}}
\newcommand{\bR}{\mathbb{R}}
\newcommand{\bS}{\mathbb{S}}
\newcommand{\bZ}{\mathbb{Z}}

%Math caligraphy
\newcommand{\cA}{\mathcal{A}}
\newcommand{\cC}{\mathcal{C}}
\newcommand{\cK}{\mathcal{K}}
\newcommand{\cM}{\mathcal{M}}
\newcommand{\cO}{\mathcal{O}}

%Greek blackboard font:
\newcommand{\bmu}{\mbox{$\raisebox{-0.59ex}
  {$l$}\hspace{-0.18em}\mu\hspace{-0.88em}\raisebox{-0.98ex}{\scalebox{2}
  {$\color{white}.$}}\hspace{-0.416em}\raisebox{+0.88ex}
  {$\color{white}.$}\hspace{0.46em}$}{}}

\lhead{University of California, Berkeley}
\rhead{Math 113, Spring 2022}

\begin{document}
\begin{center}
\Large {Homework Assignment 5}\\
\small {Due Friday, February 25}
\end{center}
In this assignment we answer the following question:
\begin{question}\label{q1}
Let $G$ be a group, and $H\le G$ a subgroup.  When is $G/H$ a group?
\end{question}
More specifically we are asking when the set of a cosets a group under the mutliplication rule given by $(g_1H)(g_2H) = g_1g_2H$.  We want a way to answer the question \textit{intrinsically} to $G$ and $H$.  For this we recall the following definition from the 2/22 lecture.
\begin{definition}
Let $G$ be a group and $H\le G$ a subgroup.  For $g\in G$ the \textbf{conjugate of $H$ by $g$} is the set:
\[gHg^{-1} = \{ghg^{-1}\text{ }:\text{ }h\in H\}.\]
We say that $H$ is a \textbf{normal subgroup} if for every $g\in G$ we have
$gHg^{-1} = H$.  If $H\le G$ is a normal subgroup, we write $H\unlhd G$.
\end{definition}
An intrinsic answer to Question \ref{q1} is given by the following theorem.
\begin{theorem}\label{t1}
Let $G$ be a group and $H\le G$ a subgroup.  The following are equivalent.
\begin{enumerate}[(i)]
\item $H\unlhd G$
\item $G/H$ is a group under the rule $(g_1H)(g_2H) = g_1g_2H$.
\item $H$ is the kernel of a group homomorphism with domain $G$.
\end{enumerate}
\end{theorem}
\begin{enumerate}
\item{There is only one goal in this assignment: to prove Theorem \ref{t1}.  To achieve this goal, we will prove $(i)\implies(ii)\implies(iii)\implies(i)$.
\begin{enumerate}
\item{Suppose $H\unlhd G$.  Show that $G/H$ is a group under the rule $(g_1H)(g_2H) = g_1g_2H$.  This shows that $(i)\implies(ii)$.}
\item{Suppose that $G/H$ is a group under the rule $(g_1H)(g_2H) = g_1g_2H$.  Produce a group homomorphism from $G$ to some group whose kernel is $H$.  This shows that $(ii)\implies (iii)$.  (Note: we essentially gave this argument in the 2/22 lecture, but do include a full proof here as well).}
\item{Let $\varphi:G\to G'$ be a group homomorphism with kernel $H$.  Show that $H\unlhd G$.  This proves $(iii)\implies(i)$, thereby completing the proof of Theorem \ref{t1}.}
\end{enumerate}
}
\end{enumerate}
\end{document}
