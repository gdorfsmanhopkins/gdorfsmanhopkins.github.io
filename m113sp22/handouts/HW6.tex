\documentclass[11pt]{article}
\usepackage[top = 1in, bottom = 1in, left =1in, right = 1in]{geometry}
\usepackage{graphicx}
\usepackage{amsmath}
\usepackage{tabu}
\usepackage{amssymb}
\usepackage{amsmath}
\usepackage{etoolbox}
\usepackage{xcolor}
\usepackage{amsthm}
\usepackage{tikz-cd}
\usepackage{tikz}
\usepackage{tkz-graph}
\usepackage{seqsplit}
\usepackage{ulem}
\usepackage{tabularx}
\AtBeginEnvironment{proof}{\color{blue}}
\newtheorem{theorem}{Theorem}
\newtheorem{proposition}{Proposition}
\newtheorem{lemma}{Lemma}
\newtheorem*{facts}{Fact}
\newtheorem*{remark}{Remark}
\newtheorem{corollary}{Corollary}
\newtheorem{definition}[theorem]{Definition}
\newtheorem{question}[theorem]{Question}
\newtheorem*{hint}{Hint}
\usepackage{enumerate}
\usepackage{hyperref}
\usepackage{fancyhdr}\pagestyle{fancy}
\newcommand{\la}{\langle}
\newcommand{\ra}{\rangle}
\newcommand{\tors}{\mathrm{tors}}
\newcommand{\ab}{\mathrm{ab}}
\newcommand{\Aut}{\operatorname{Aut}}
\newcommand{\Inn}{\operatorname{Inn}}
\newcommand{\im}{\operatorname{im}}
\newcommand{\lcm}{\operatorname{lcm}}
\newcommand{\ch}{\operatorname{char}}

%Math blackboard:
\newcommand{\bC}{\mathbb{C}}
\newcommand{\bF}{\mathbb{F}}
\newcommand{\bN}{\mathbb{N}}
\newcommand{\bQ}{\mathbb{Q}}
\newcommand{\bR}{\mathbb{R}}
\newcommand{\bS}{\mathbb{S}}
\newcommand{\bZ}{\mathbb{Z}}

%Math caligraphy
\newcommand{\cA}{\mathcal{A}}
\newcommand{\cC}{\mathcal{C}}
\newcommand{\cK}{\mathcal{K}}
\newcommand{\cM}{\mathcal{M}}
\newcommand{\cO}{\mathcal{O}}

%Greek blackboard font:
\newcommand{\bmu}{\mbox{$\raisebox{-0.59ex}
  {$l$}\hspace{-0.18em}\mu\hspace{-0.88em}\raisebox{-0.98ex}{\scalebox{2}
  {$\color{white}.$}}\hspace{-0.416em}\raisebox{+0.88ex}
  {$\color{white}.$}\hspace{0.46em}$}{}}

\lhead{University of California, Berkeley}
\rhead{Math 113, Spring 2022}

\begin{document}
\begin{center}
\Large {Homework Assignment 6}\\
\small {Due Friday, March 4}
\end{center}
\begin{enumerate}
  \item Let $G$ be a group, and let $A$ be a subset of $G$.  Let's establish some facts about centralizers and normalizers.
  \begin{enumerate}
    \item Let $A$ be a subset of $G$.  Prove that $C_G(A)\le G$.
    \item Deduce the following chain of inclusions.
    \[Z(G)\le C_G(A)\le N_G(A)\le G.\]
    (\textit{Note: }In class we only defined the normalizer of a subgroup, but we can define the normalizer of a subset the same way: $N_G(A) = \{g\in G:gAg^{-1} = A\}$,)
    \item Show that $C_G(A) = C_G(\la A\ra)$.
    \item Give an example to show the analog of part (c) for normalizers is not true.  That is, give $A\subseteq G$ where $N_G(A)\not=N_G(\la A\ra)$.
    \item Show that if $H$ is a subgroup of $G$, then $H\le N_G(H)$.
    \item Show that $H\le C_G(H)$ if and only if $H$ is abelian.
  \end{enumerate}
    \item Compute the center of the dihedral group.  Explicitly, let $n$ be an integer $\ge3$.  Compute $Z(D_{2n})$.  (Note: you will need to split into the two cases, where $n$ is even or $n$ is odd).
    \item In this exercise we see that we can learn important facts about groups by studying their quotients.
      \begin{enumerate}
    \item Suppose $H\le Z(G)$.  Show that $H$ is a normal subgroup of $G$.  (In particular, $Z(G)$ is normal).
    \item Show that if $G/Z(G)$ is cyclic, then $G$ is abelian.
    \item Let $p$ and $q$ be prime numbers (not necessarily distinct), and $G$ a group of order $pq$.  Show that if $G$ is not abelian, then $Z(G) = \{1\}$.
  \end{enumerate}
  \item In this exercise we show that if $G$ is a nonabelian group of order $6$.  We will show $G\cong S_3$.
  \begin{enumerate}
    \item Show that there is an element $x\in G$ of order 2.  (Once we have Cauchy's theorem for nonabelian groups this part becomes easy, but since $G$ has 6 elements, one can do this by inspection using Lagrange's theorem).
    \item Let $x\in G$ have order 2, and let $H = \la x\ra$.  Show that $H$ is not normal in $G$.  (\textit{Hint:} Show that if $H$ is normal then $H\le Z(G)$, then  apply 3(c) to find a contradiction.)
    \item Define an action of $G$ on the set $A = G/H$ by \textit{left multiplication}: that is $g\cdot(xH) = gxH$.  Show that this defines a well defined group action.
    \item Consider the action of $G$ on $A = G/H$ by left multiplication.  Show that the associated permutation representation is injective.  Conclude that $G\cong S_3$.
    \end{enumerate}
As we start defining more exotic properties of groups we will need to expand our library of finite groups to exhibit some of these interesting properties.  We finish with two new examples of finite groups.  First up: Quaternions.
\begin{definition}
  The \textit{quaternion group of order 8}, denoted $Q_8$ is the group of the following 8 elements:
  \[Q_8 = \{\pm1,\pm i, \pm j, \pm k\}\]
  subject to the relations:
  \[(-1)^2 = 1\]
  \[i^2 = j^2 = k^2 = -1,\]
  \[(-1)x = -x = x(-1)\text{ for all }x,\]
  \begin{eqnarray*}
    ij = k, & \hspace{20pt} & ji = -k,\\
    jk = i, & \hspace{20pt} & kj = -i,\\
    ki = j, & \hspace{20pt} & ik = -j.
  \end{eqnarray*}
\end{definition}
  \item Let's establish some basic facts about $Q_8$. Much of this is worked out in the book.
  \begin{enumerate}
    \item Write the entire multiplication table for $Q_8$.
    \item Find 2 elements which generate all of $Q_8$.  (\textit{Bonus:} Can you give a presentation of $Q_8$?)
    \item Prove that $Q_8$ is not isomorphic to $D_8$.
    \item Find all the subgroups of $Q_8$, and draw them in a lattice ordered by inclusion.  (\textit{Hint}: there are 6 total subgroups).
    \item Prove that every subgroup of $Q_8$ is normal.  (\textit{Note}: we saw that if a group is abelian, every subgroup is normal.  This shows the converse isn't true!)
    \item Prove that every \textit{proper} subgroup and quotient group of $Q_8$ is abelian (\textit{Hint}: You can appeal to TH1\#4).
    \item Show that $Q_8/Z(Q_8)$ has order 4.  By TH1 it must be isomorphic to $Z_4$ or $V_4$.  Which one is it?  Justify your answer. (\textit{Hint for the second part}: you can do this by hand, but it might be slicker to apply 3(b)).
  \end{enumerate}  
Let's finish by introducing finite matrix groups.  We will need a definition.
\begin{definition}
  A \textit{field} is a set $F$ together with two commutative binary operations, $+$ and $\cdot$ (addition and multiplication), such that $(F,+)$ and $(F\setminus\{0\},\cdot)$ are abelian groups, and such that the distributive law holds.  That is, for all $a,b,c\in F$ we have:
  \[a\cdot(b+c) = a\cdot b + a\cdot c.\]
  For any field we let $F^\times = F\setminus\{0\}$ be its \textit{mutliplicative group}.  A field $F$ is called a finite field if $|F|<\infty$.
\end{definition}
It turns out that vector space theory over $F$ is pretty much identical to vector space theory over $R$.  We can define the first matrix group we hope to study.
\begin{definition}
  Let $F$ be a field.  If $M,N$ are matrices with entries in $F$, we can compute their product $MN$ and the determinant $\det(M)\in F$ using the same formulas as if $F=\bR$.  Then the \textit{general linear group of degree} $n$ over $F$ is,
  \[GL_n(F) = \{A\text{ }|\text{ } A\text{ is an }n\times n\text{ matrix with entries in }F\text{ and }\det(A)\not=0\}.\]
\end{definition}
You may use the following fact without proof (since it is a standard result of linear algebra).
\begin{proposition}
  The set $GL_n(F)$ can be identified with the set of linear bijections $F^n\to F^n$, and matrix multiplication corresponds to composition of functions.  In particular, $GL_n(F)$ is a group under matrix multiplication.
\end{proposition}
  \item It turns out that we have seen examples of finite fields already.
  \begin{enumerate}
    \item Let $p$ be a prime number.  Show that $\bZ/p\bZ$ with the operations $+$ and $\times$ is a field.  This is the \textit{finite field of order} $p$ and will be denoted by $\bF_p$.
    \item Show that if $n$ is not prime, $\bZ/n\bZ$ is not a field.
  \end{enumerate}
\item Now let's study $GL_2(\bF_p)$.
  \begin{enumerate}
    \item Prove that $|GL_2(\bF_2)| = 6$.
    \item Write all the elements of $GL_2(\bF_2)$ and compute the order of each element.
    \item Show that $GL_2(\bF_2)$ is not abelian.  Conclude that it is isomorphic to $S_3$.
    \item Generalizing part (a), show that if $p$ is prime then
    \[|GL_2(\bF_p)| = p^4-p^3-p^2+p.\]
  \end{enumerate}
\end{enumerate}

\end{document}
