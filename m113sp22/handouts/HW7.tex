\documentclass[11pt]{article}
\usepackage[top = 1in, bottom = 1in, left =1in, right = 1in]{geometry}
\usepackage{graphicx}
\usepackage{amsmath}
\usepackage{tabu}
\usepackage{amssymb}
\usepackage{amsmath}
\usepackage{etoolbox}
\usepackage{xcolor}
\usepackage{amsthm}
\usepackage{tikz-cd}
\usepackage{tikz}
\usepackage{tkz-graph}
\usepackage{seqsplit}
\usepackage{ulem}
\usepackage{tabularx}
\AtBeginEnvironment{proof}{\color{blue}}
\newtheorem{theorem}{Theorem}
\newtheorem{proposition}{Proposition}
\newtheorem{lemma}{Lemma}
\newtheorem*{facts}{Fact}
\newtheorem*{remark}{Remark}
\newtheorem{corollary}{Corollary}
\newtheorem{definition}[theorem]{Definition}
\newtheorem{question}[theorem]{Question}
\newtheorem*{hint}{Hint}
\usepackage{enumerate}
\usepackage{hyperref}
\usepackage{fancyhdr}\pagestyle{fancy}
\newcommand{\la}{\langle}
\newcommand{\ra}{\rangle}
\newcommand{\tors}{\mathrm{tors}}
\newcommand{\ab}{\mathrm{ab}}
\newcommand{\Aut}{\operatorname{Aut}}
\newcommand{\Inn}{\operatorname{Inn}}
\newcommand{\im}{\operatorname{im}}
\newcommand{\lcm}{\operatorname{lcm}}
\newcommand{\ch}{\operatorname{char}}

%Math blackboard:
\newcommand{\bC}{\mathbb{C}}
\newcommand{\bF}{\mathbb{F}}
\newcommand{\bN}{\mathbb{N}}
\newcommand{\bQ}{\mathbb{Q}}
\newcommand{\bR}{\mathbb{R}}
\newcommand{\bS}{\mathbb{S}}
\newcommand{\bZ}{\mathbb{Z}}

%Math caligraphy
\newcommand{\cA}{\mathcal{A}}
\newcommand{\cC}{\mathcal{C}}
\newcommand{\cK}{\mathcal{K}}
\newcommand{\cM}{\mathcal{M}}
\newcommand{\cO}{\mathcal{O}}

%Greek blackboard font:
\newcommand{\bmu}{\mbox{$\raisebox{-0.59ex}
  {$l$}\hspace{-0.18em}\mu\hspace{-0.88em}\raisebox{-0.98ex}{\scalebox{2}
  {$\color{white}.$}}\hspace{-0.416em}\raisebox{+0.88ex}
  {$\color{white}.$}\hspace{0.46em}$}{}}

\lhead{University of California, Berkeley}
\rhead{Math 113, Spring 2022}

\begin{document}
\begin{center}
\Large {Homework Assignment 7}\\
\small {Due Friday, March 11}
\end{center}
\begin{enumerate}
  \item{
  Let $G$ be a group and let $H,K\le G$ be subgroups.  Recall that we defined the set:
  \[HK = \{hk: h\in H,k\in K\}\subseteq G.\]
  The second isomorphism theorem relied on the following two facts, which you will now verify.
  \begin{enumerate}
  \item{Show that $HK$ is a subgroup of $G$ if and only if $HK = KH$.}
  \item{Use part (a) to show that that if $H\le N_G(K)$, then $HK$ is a subgroup of $G$.  Explain why this means that if either $H$ or $K$ are normal subgroups, then $HK\le G$.}
  \end{enumerate}
  }
  \item Let $G$ be a group, and $M,N\unlhd G$ normal subgroups such that $MN = G$.  Use the first and second isomorphism theorems to establish the following facts.
  \begin{enumerate}
    \item Show $G/(M\cap N)\cong (G/M)\times (G/N)$
    \item Suppose further that $M\cap N=\{1\}$.  Show that $G\cong M\times N$.
  \end{enumerate}  
  \item{
  We continue by proving the fourth isomorphism theorem.  Let $N\unlhd G$ be a normal subgroup of a group $G$.  Let $\pi:G\to G/N$ be the natural projection.
  \begin{enumerate}
    \item Let $H\le G/N$.  Show that the preimage $\pi^{-1}(H) = \{g\in G:\pi(g)\in H\}$ is a subgroup of $G$ containing $N$.
    \item Let $H\le G$.  Show that its image $\pi(H)$ is a subgroup of $G/N$.
    \item These constructions do not in general give a bijection between subgroups of $G$ and subgroups of $G/N$.  Give an example showing why.
    \item If we restrict our attention to certain subgroups of $G$ we do get a bijection.  Show that the constructions in parts (a) and (b) give a bijection:
    \[\left\{
    \begin{array}{c}
      \text{Subgroups }H\le G\\
      \text{such that }N\le H
    \end{array}\right\}
    \Longleftrightarrow
    \left\{
    \begin{array}{c}
      \text{Subgroups}\\
      \overline{H}\le G/N
    \end{array}
    \right\}
    \]
  \item This bijection satisfies certain properties.  First let's establish some notation. Let $H,K\in G$ be two subgroups containing $N$, and denote the corresponding subgroups of $G/N$ by $\overline H$ and $\overline K$.  Prove the following properties.
    \begin{enumerate}
      \item $H\le K$ if and only if $\overline H\le\overline K$.
      \item $H\unlhd K$ if and only if $\overline H\unlhd\overline K$.
      \item $\overline{H\cap K} = \overline H\cap\overline K$
      \item $\overline{\langle H,K\rangle} = \langle\overline H,\overline K\rangle$.
    \end{enumerate}
    \begin{hint}
      You can do (iii) and (iv) directly, but if you want to be really slick use that the intersection of two subgroups is the largest subgroup contained in both, (and the dual notion for the subgroup generated by two subgroups).  Notice that this means that being the intersection of two subgroups (or generated by two subgroups) is a condition on the lattice of $G$ (or $G/N$).  Then the result should easily follow from part (i).
    \end{hint}
  \end{enumerate}
  }
    \item By Cayley's theorem, the group $Q_8$ from HW6 Problem 5 is isomorphic to a subgroup of $S_8$.  Let's write down such a subgroup explicitly!
  \begin{enumerate}
    \item Label $\{1,-1,i,-i,j,-j,k,-k\}$ as the numbers $\{1,2,3,4,5,6,7,8\}$. Then the action of $Q_8$ on itself by left multiplication gives an injective map $Q_8\to S_8$.  Write the permutation representations for $-1$ and $i$ as elements $\sigma_{-1},\sigma_i\in S_8$, and verify that $\sigma_i^2 = \sigma_{-1}$.  (Using the multiplication table from HW6 Problem 5 may make this easier).
    \item Use the generators from HW6 Problem 5(b) to give two elements of $S_8$ which generate a subgroup $H\le S_8$ isomorphic to $Q_8$.
    \end{enumerate}
\item Let $G$ be a group.  Let $[G,G] = \la x^{-1}y^{-1}xy | x,y\in G\ra$.
  \begin{enumerate}
    \item Show that $[G,G]$ is a normal subgroup of $G$.
    \item Show that $G/[G,G]$ is abelian.
  \end{enumerate}
  $[G,G]$ is called the \textit{commutator subgroup} of $G$, and $G/[G,G]$ is called the \textit{abelianization} of $G$, denoted $G^\ab$.  The rest of this exercise explains why.
  \begin{enumerate}
    \setcounter{enumii}{2}
    \item Let $\varphi:G\to H$ be a homomorhism with $H$ abelian.  Show $[G,G]\subseteq\ker\varphi$.
    \item Conclude that for $H$ an abelian group there is a bijection:
    \[\left\{
    \begin{array}{c}
      \text{Homomorphisms }\varphi:G\to H\\
    \end{array}\right\}
    \Longleftrightarrow
    \left\{
    \begin{array}{c}
      \text{Homomorphisms }\tilde\varphi:G^\ab\to H\\
    \end{array}
    \right\}
    \]
    \begin{hint}
      Recall the technique of passing to the quotient described in the 5/3 lecture.
    \end{hint}
  \end{enumerate}
  \item Let's now compute $D_{2n}^\ab$.  We should begin computing $xyx^{-1}y^{-1}$.  There are 3 cases.
  \begin{enumerate}
    \item Compute $x^{-1}y^{-1}xy$ in each of the following 3 cases. (\textit{Hint:} HW2\#9(e) gives the inverse for a reflection.)
    \begin{enumerate}[(i)]
      \item $x,y$ both reflections.  So $x=sr^i$ and $y=sr^j$.
      \item $x$ a reflection and $y$ not a reflection.  So $x=sr^i$ and $y=r^j$.
      \item Neither $x$ nor $y$ are reflections.  So $x=r^i$ and $y=r^j$.
    \end{enumerate}
    \item Prove that $[D_{2n},D_{2n}] = \la r^2\ra$.  If $n$ is odd one could choose another generator.  What is it?
    \item Now prove that $D_{2n}^\ab$ is either $V_4$ or $Z_2$ depending on whether $n$ is odd or even.  Note that since this is so small we should interpret this as suggesting that $D_{2n}$ is far from abelian.
    \end{enumerate}
    \end{enumerate}
 Let $F$ be a field.  The general linear group $GL_n(F)$ from HW6 Problem 7 has lots of interesting subgroups and quotients, which we study in the following problem.  You may use the following fact without proof, as it is a standard result of linear algebra.
 \begin{proposition}
  If $A,B\in GL_n(F)$, then $\det(AB)=\det(A)\det(B)$.  In particular, $\det:GL_n(F)\to F^\times$ is a group homomorphism.
\end{proposition}
\begin{enumerate}
\setcounter{enumi}{6}
   \item
  \begin{enumerate}
    \item Show that the constant diagonal matrices are a normal subgroup of $GL_n(F)$ isomorphic to $F^\times$.
  \end{enumerate}
  We will often abuse notation and denote this by $F^\times\unlhd GL_n(F)$.  The quotient group $GL_n(F)/F^\times$ is called the \textit{projective general linear group} and denoted $PGL_n(F)$.
  \begin{enumerate}
    \setcounter{enumii}{1}
    \item The \textit{special linear group} $SL_n(F)$ is defined
    \[SL_n(F) = \{A\in GL_n(F)\text{ }|\text{ }\det(A) = 1.\}\]
    Show that $SL_n(F)$ is a normal subgroup of $GL_n(F)$ and prove that
    \[GL_n(F)/SL_n(F)\cong F^\times.\]
    (\textit{Hint:} Use the First Isomorphism Theorem and Proposition 1)
    \item List all the elements of $SL_2(\bF_2)$.
    \item Compute $|SL_2(\bF_p)|$  (\textit{Hint}, between 4(b) and HW6 Problem 7(d) you've already done all the work).
    \item Let $I$ be the identity matrix.  Show that $\{\pm I\}\le SL_n(F)$ if and only if $n$ is even.
    \item Use the second isomorphism theorem to construct an isomorphism:
    \[PGL_2(\bC)\cong SL_2(\bC)/\{\pm I\}.\]
    (As a bonus, think about why this is not true for a general field.  For example, it is false over $\bR$, or over $\bF_p$ for $p\not=2$.)
  \end{enumerate}
\end{enumerate}

\end{document}
