\documentclass[11pt]{article}
\usepackage[top = 1in, bottom = 1in, left =1in, right = 1in]{geometry}
\usepackage{graphicx}
\usepackage{amsmath}
\usepackage{tabu}
\usepackage{amssymb}
\usepackage{amsmath}
\usepackage{etoolbox}
\usepackage{xcolor}
\usepackage{amsthm}
\usepackage{tikz-cd}
\usepackage{tikz}
\usepackage{tkz-graph}
\usepackage{seqsplit}
\usepackage{ulem}
\usepackage{tabularx}
\AtBeginEnvironment{proof}{\color{blue}}
\newtheorem{theorem}{Theorem}
\newtheorem{proposition}{Proposition}
\newtheorem{lemma}{Lemma}
\newtheorem*{facts}{Fact}
\newtheorem*{remark}{Remark}
\newtheorem{corollary}{Corollary}
\newtheorem{definition}[theorem]{Definition}
\newtheorem{question}[theorem]{Question}
\newtheorem*{hint}{Hint}
\usepackage{enumerate}
\usepackage{hyperref}
\usepackage{fancyhdr}\pagestyle{fancy}
\newcommand{\la}{\langle}
\newcommand{\ra}{\rangle}
\newcommand{\tors}{\mathrm{tors}}
\newcommand{\ab}{\mathrm{ab}}
\newcommand{\Aut}{\operatorname{Aut}}
\newcommand{\Inn}{\operatorname{Inn}}
\newcommand{\im}{\operatorname{im}}
\newcommand{\lcm}{\operatorname{lcm}}
\newcommand{\ch}{\operatorname{char}}

%Math blackboard:
\newcommand{\bC}{\mathbb{C}}
\newcommand{\bF}{\mathbb{F}}
\newcommand{\bN}{\mathbb{N}}
\newcommand{\bQ}{\mathbb{Q}}
\newcommand{\bR}{\mathbb{R}}
\newcommand{\bS}{\mathbb{S}}
\newcommand{\bZ}{\mathbb{Z}}

%Math caligraphy
\newcommand{\cA}{\mathcal{A}}
\newcommand{\cC}{\mathcal{C}}
\newcommand{\cK}{\mathcal{K}}
\newcommand{\cM}{\mathcal{M}}
\newcommand{\cO}{\mathcal{O}}

%Greek blackboard font:
\newcommand{\bmu}{\mbox{$\raisebox{-0.59ex}
  {$l$}\hspace{-0.18em}\mu\hspace{-0.88em}\raisebox{-0.98ex}{\scalebox{2}
  {$\color{white}.$}}\hspace{-0.416em}\raisebox{+0.88ex}
  {$\color{white}.$}\hspace{0.46em}$}{}}

\lhead{University of California, Berkeley}
\rhead{Math 113, Spring 2022}

\begin{document}
\begin{center}
\Large {Homework Assignment 8}\\
\small {Due Friday, March 18}
\end{center}
\begin{enumerate}
  \item Cayley's theorem says that if $|G|=n$ then $G$ embeds into $S_n$ (that is, is isomorphic to a subgroup of $S_n$).  One could ask if this $n$ is \textit{sharp}, or if perhaps $G$ can embed in some smaller symmetric group.
  \begin{enumerate}
  \item{Give an example to show that Cayley's theorem isn't always sharp.  That is, give a group of order $n$ which embeds into $S_d$ for some $d<n$.}
  \end{enumerate}
    Nevertheless, we are about to see that for $Q_8$ the symmetric group given by Cayley's theorem is the smallest.  This shows that there can be no strengthening of Cayley's theorem in general.
  \begin{enumerate}
  \setcounter{enumii}{1}
    \item Let $Q_8$ act on a set $A$ with $|A|\le 7$.  Let $a\in A$.  Show that the stabilizer of $a$,  $(Q_8)_a\le Q_8$ must contain the subgroup $\{\pm1\}$.  (\textit{Hint:} It might be helpful to use the orbit stabilizer theorem and the lattice from HW6 Problem 5(d).)
    \item Deduce that the kernel of the action of $Q_8$ on $A$ contains $\{\pm1\}$.
    \item Conclude that $Q_8$ cannot embed into $S_n$ for $n\le7$.  That is, show there is no injective homomorphisms $Q_8\hookrightarrow S_n$ for $n\le7$.
  \end{enumerate}
    \item Find all groups with exactly 2 conjugacy classes.  (\textit{Hint}: Use the class equation.)
    \item Compute all the conjugacy classes for the following groups, and verify that the class equation holds in each case.
    \begin{enumerate}
    \item $S_3$
    \item $Q_8$
    \end{enumerate}
    \end{enumerate}
    For the next problem it may be useful to recall the following fact we proved in class.
    \begin{theorem}[Cauchy's Theorem for Abelian Groups]
    Let $G$ be an abelian group of order $n$.  If $p$ is a prime dividing $n$, then $G$ has a subgroup of order $p$.
    \end{theorem}
    This will turn out to be true for all groups, so so far we only have it in the abelian case.
    \begin{enumerate}
    \setcounter{enumi}{3}
    \item The converse to Lagrange's theorem holds for groups of prime power order.  To prove this we will need to strengthen the fourth isomorphism theorem (HW5\#1).
  \begin{enumerate}
    \item Let $G$ be a group and $N\unlhd G$.  Let $N\le H\le K\le G$, and let $\overline H,\overline K$ be the corresponding subgroups of $G/N$ as in HW5\#1.  Show that $|K:H| = |\overline K:\overline H|$.  (\textit{Hint}: There is an obvious map $K/H\to\overline K/\overline H$.  Prove it is bijective.  Be careful though, we don't know that $K/H$ is a group, just a set of cosets.)
    \item Suppose $|G| = p^d$ for a prime $p$ and $d\ge 1$.  Show that $G$ has a normal subgroup of order $p$.  In particular, we have extended Cauchy's theorem to nonabelian $p$-groups!  (\textit{Hint:} What did the class equation say about the center of a $p$-group?)
    \item Suppose $|G| = p^d$ for a prime $p$ and $d\ge1$.  Show that for every $a = 1,2,\cdots,d$, $G$ has a subgroup of order $p^a$.  (Use parts (a) and (b) to proceed by induction).
  \end{enumerate}
  \item Here we classify all abelian groups of order $pq$ for $p\not=q$ prime.
  \begin{enumerate}
  \item Let $G$ be a group of finite order and suppose that $x,y\in G$ are commuting elements, i.e., that $xy = yx$.  Show that that $|xy|$ divides the least common multiple of $x$ and $y$.
  \item Let $G$ be an abelian group of order $pq$ for primes $p\not=q$.  Show that $G\cong Z_{pq}$.
  \item Classify all groups of order 6 up to isomorphism.
  \end{enumerate}
  \item Let $V$ be an abelian group of order $p^n$ for some prime $p$ and $n>0$.  Suppose that every element of $V$ has order $\le p$.  Show by induction on $n$ that:
  \[V\cong \underbrace{Z_p\times Z_p\times\cdots Z_p}_{n\text{ times}}.\]
  We will call such a $V$ the \textit{elementary abelian group of order $p^n$}.  We will see in the following question that these are the same as finite dimensional $\bF_p$ vector spaces!
  \item Let $V$ be an elementary abelian group of order $p^n$.  And identify it with
  \[V\cong\underbrace{(\bZ/p\bZ)\times\cdots(\bZ/p\bZ)}_{n\text{ times}}.\]
  For $\lambda\in\bF_p$ and $v = (v_1,\cdots,v_n)\in V$, we can let:
  \[\lambda v = (\lambda v_1,\cdots,\lambda v_n).\]
  \begin{enumerate}
  \item Explain why the scalar multiplication giving above makes $V$ into an $\bF_p$-vector space.
  \item Show that a function $\varphi:V\to V$ is a homomorphism if and only if it is a linear map of vector spaces.
  \item Using Proposition 1 from HW6, identify the set of isomorphisms from $V$ to itself with a group we have already seen.
  \end{enumerate}
\end{enumerate}
\end{document}
