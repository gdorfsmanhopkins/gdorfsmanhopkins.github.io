\documentclass[11pt]{article}
\usepackage[top = 1in, bottom = 1in, left =1in, right = 1in]{geometry}
\usepackage{graphicx}
\usepackage{amsmath}
\usepackage{tabu}
\usepackage{amssymb}
\usepackage{amsmath}
\usepackage{etoolbox}
\usepackage{xcolor}
\usepackage{amsthm}
\usepackage{tikz-cd}
\usepackage{tikz}
\usepackage{tkz-graph}
\usepackage{seqsplit}
\usepackage{ulem}
\usepackage{tabularx}
\AtBeginEnvironment{proof}{\color{blue}}
\newtheorem{theorem}{Theorem}
\newtheorem{proposition}{Proposition}
\newtheorem{lemma}{Lemma}
\newtheorem*{facts}{Fact}
\newtheorem*{remark}{Remark}
\newtheorem{corollary}{Corollary}
\newtheorem{definition}[theorem]{Definition}
\newtheorem{question}[theorem]{Question}
\newtheorem*{hint}{Hint}
\usepackage{enumerate}
\usepackage{hyperref}
\usepackage{fancyhdr}\pagestyle{fancy}
\newcommand{\la}{\langle}
\newcommand{\ra}{\rangle}
\newcommand{\tors}{\mathrm{tors}}
\newcommand{\ab}{\mathrm{ab}}
\newcommand{\Aut}{\operatorname{Aut}}
\newcommand{\Inn}{\operatorname{Inn}}
\newcommand{\im}{\operatorname{im}}
\newcommand{\lcm}{\operatorname{lcm}}
\newcommand{\ch}{\operatorname{char}}

%Math blackboard:
\newcommand{\bC}{\mathbb{C}}
\newcommand{\bF}{\mathbb{F}}
\newcommand{\bN}{\mathbb{N}}
\newcommand{\bQ}{\mathbb{Q}}
\newcommand{\bR}{\mathbb{R}}
\newcommand{\bS}{\mathbb{S}}
\newcommand{\bZ}{\mathbb{Z}}

%Math caligraphy
\newcommand{\cA}{\mathcal{A}}
\newcommand{\cC}{\mathcal{C}}
\newcommand{\cK}{\mathcal{K}}
\newcommand{\cM}{\mathcal{M}}
\newcommand{\cO}{\mathcal{O}}

%Greek blackboard font:
\newcommand{\bmu}{\mbox{$\raisebox{-0.59ex}
  {$l$}\hspace{-0.18em}\mu\hspace{-0.88em}\raisebox{-0.98ex}{\scalebox{2}
  {$\color{white}.$}}\hspace{-0.416em}\raisebox{+0.88ex}
  {$\color{white}.$}\hspace{0.46em}$}{}}

\lhead{University of California, Berkeley}
\rhead{Math 113, Spring 2022}

\begin{document}
\begin{center}
\Large {Homework Assignment 9}\\
\small {Due Friday, April 8}
\end{center}
\begin{enumerate}
\item Let $G$ be a finite group, $p$ a prime, and $P\in Syl_p(G)$ a Sylow $p$-subgroup.  Use \textbf{(Sylow 2)} to show that $P\unlhd G$ if and only if $n_p = \#Syl_p(G) = 1$.
\item Use Sylow's theorems to prove that a group of order 200 can never be simple.
  \item{
  Generalizing question 2, let $G$ be a group of order $p^2q$ for primes $p$ and $q$.  We will show that $G$ always has a nontrivial \textit{normal} Sylow subgroup.
  \begin{enumerate}
    \item Suppose $p>q$.  Show that $G$ has a normal subgroup of order $p^2$.
    \item Suppose $q>p$.  Show that either $G$ has a normal subgroup of order $q$, or else $G\cong A_4$.  (You may use the result from the April 5 lecture that if $|G|=12$ and $n_3\not=1$ then $G\cong A_4$).
    \item Explain why a group of order $p^2q$ can never be simple.  (You may need to treat the cases where $G=A_4$ or $p=q$ separately).
  \end{enumerate}
  }
\item Let $G$ be a group of order $99 = 3^2 * 11$.  Let's show that $G$ is abelian.
\begin{enumerate}
\item  Let $P\le G$ be a Sylow 3-subgroup.  Show that $P\unlhd G$.
\item Construct an injective homomorphism $G/C_G(P)\hookrightarrow Aut(P)$.  (Can you use group actions and part (a)?)
\item Deduce from part (b) and Lagrange's theorem that $G=C_G(P)$.  Leverage this fact to prove that $G$ is abelian.
\end{enumerate}
\end{enumerate}
\end{document}
