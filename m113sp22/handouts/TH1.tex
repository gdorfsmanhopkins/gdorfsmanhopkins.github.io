\documentclass[11pt]{article}
\usepackage[top = 1in, bottom = 1in, left =1in, right = 1in]{geometry}
\usepackage{graphicx}
\usepackage{amsmath}
\usepackage{tabu}
\usepackage{amssymb}
\usepackage{amsmath}
\usepackage{etoolbox}
\usepackage{xcolor}
\usepackage{amsthm}
\usepackage{tikz-cd}
\usepackage{tikz}
\usepackage{tkz-graph}
\usepackage{seqsplit}
\usepackage{ulem}
\usepackage{tabularx}
\AtBeginEnvironment{proof}{\color{blue}}
\newtheorem{theorem}{Theorem}
\newtheorem{proposition}{Proposition}
\newtheorem{lemma}{Lemma}
\newtheorem*{facts}{Fact}
\newtheorem*{remark}{Remark}
\newtheorem{corollary}{Corollary}
\newtheorem{definition}{Definition}
\newtheorem*{hint}{Hint}
\usepackage{enumerate}
\usepackage{hyperref}
\usepackage{fancyhdr}\pagestyle{fancy}
\newcommand{\la}{\langle}
\newcommand{\ra}{\rangle}
\newcommand{\tors}{\mathrm{tors}}
\newcommand{\ab}{\mathrm{ab}}
\newcommand{\Aut}{\operatorname{Aut}}
\newcommand{\Inn}{\operatorname{Inn}}
\newcommand{\im}{\operatorname{im}}
\newcommand{\lcm}{\operatorname{lcm}}
\newcommand{\ch}{\operatorname{char}}

%Math blackboard:
\newcommand{\bC}{\mathbb{C}}
\newcommand{\bF}{\mathbb{F}}
\newcommand{\bN}{\mathbb{N}}
\newcommand{\bQ}{\mathbb{Q}}
\newcommand{\bR}{\mathbb{R}}
\newcommand{\bS}{\mathbb{S}}
\newcommand{\bZ}{\mathbb{Z}}

%Math caligraphy
\newcommand{\cA}{\mathcal{A}}
\newcommand{\cC}{\mathcal{C}}
\newcommand{\cK}{\mathcal{K}}
\newcommand{\cM}{\mathcal{M}}
\newcommand{\cO}{\mathcal{O}}

%Greek blackboard font:
\newcommand{\bmu}{\mbox{$\raisebox{-0.59ex}
  {$l$}\hspace{-0.18em}\mu\hspace{-0.88em}\raisebox{-0.98ex}{\scalebox{2}
  {$\color{white}.$}}\hspace{-0.416em}\raisebox{+0.88ex}
  {$\color{white}.$}\hspace{0.46em}$}{}}

\lhead{University of California, Berkeley}
\rhead{Math 113, Spring 2022}

\begin{document}
\begin{center}
\Large {Takehome Assignment 1}\\
\small {Due Tuesday, February 22}
\end{center}
In this assignment, we will prove an important result called \textit{Lagrange's Theorem}.  It goes as follows.
\begin{theorem}[Lagrange's Theorem]~\\
  If $G$ is a finite group and $H$ is a subgroup of $G$.  Then:
  \begin{enumerate}[(i)]
     \item $|H|$ divides $|G|$.
     \item $|G/H| = |G|/|H|$
     \item $|H\backslash G| = |G|/|H|$.
  \end{enumerate}
\end{theorem}
We remind the you that $H\backslash G = \{Hx:x\in G\}$ is the set of \textit{right cosets} of $G$.  With this result in hand, we will be able to deduce a celebrated result of Fermat, which is central to number theory.
\begin{theorem}[Fermat's Little Theorem]~\\
  Let $p$ be a prime number and $a$ an integer.  Then $a^p\equiv a\mod p$.
\end{theorem}
We will also be able to begin our mission of classifying finite groups up to isomorphisms, giving a complete answer for groups of order $\le5$.  To do all this, we will make the following definition.
\begin{definition}~\\
  Let $H$ be a group acting on a set $A$ and fix $a\in A$.  The \textit{orbit} of $a$ under $H$ is the set
  \[H\cdot a = \{b\in A\text{ }|\text{ }b=h\cdot a\text{ for some }h\in H\}.\]
\end{definition}
Lets begin!
\begin{enumerate}
  \item Let $H$ be a group acting on a set $A$.
  \begin{enumerate}
    \item Show that the relation
    \begin{center}
      $a\sim b$ if and only if $a = h\cdot b$ for some $h\in H$
    \end{center}
    is an equivalence relation on the set $A$.
    \item Show that the equivalence classes of this equivalence relation are precisely the orbits of the elements of $A$ under the action of $H$.
    \item Conclude that the orbits of $A$ under the action of $H$ form a partition of $A$.
  \end{enumerate}
  \item Let $H$ be a subgroup of a group $G$, and let $H$ act on $G$ by left mulptilication.
  \begin{eqnarray*}
    H\times G &\to& G\\
    (h,g) &\mapsto& hg
  \end{eqnarray*}
  \begin{enumerate}
    \item Prove this is an action.
    \item Fix $x\in G$, and consider its orbit $H\cdot x$. Show that $H$ and $H\cdot x$ have the same cardinality.  Deduce that all the orbits of $G$ under the action of $H$ have the same cardinality.
    \item Now suppose further that $G$ is a finite group.  Use part (b) and exercise 1 to deduce the parts (i) and (iii) of Lagrange's theorem.
    \item Observe that the argument we gave computed the number of right cosets.  Modify your argument to deduce part (ii) of Lagrange's theorem.
  \end{enumerate}
  \item We can use Lagrange's theorem and what we know about cyclic groups to prove Fermat's little theorem.
  \begin{enumerate}
    \item Let $|G|=n<\infty$.  Fix some $x\in G$.  Use Lagrange's theorem to show that $x^n = 1$.
    \item Let $p$ be a prime number.  Compute the order of $(\bZ/p\bZ)^\times$.  Fully justify your answer.
    \item Combine parts (a) and (b) to prove Fermat's little theorem.
  \end{enumerate}
  \item With Lagrange's theorem in hand, we can classify all finite groups of order $\le5$.
  \begin{enumerate}
    \item We first classify all groups of prime order.  Let $|G| = p$ for a prime number $p$.  Show that $G$ is cyclic.  This take care of groups of order 2,3,5 (and infinitely more cases!).  For today, only order 4 remains.
    \item Suppose every element of $G$ has order $\le2$.  Show that $G$ is abelian.
    \item Show that if $|G|=4$, then $G$ is abelian.
    \item Prove that if $|G|=4$, then $G\cong Z_4$ or $G\cong Z_2\times Z_2$.  (\textit{Remark:} The latter of these two groups is called the \textit{Klein 4-Group}, and is sometimes denoted $V_4$).
    \item Explain why $Z_4\not\cong V_4$, thus showing our classification is not redundant.
  \end{enumerate}
\end{enumerate}
\end{document}
