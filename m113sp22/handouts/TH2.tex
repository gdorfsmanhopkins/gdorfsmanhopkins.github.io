\documentclass[11pt]{article}
\usepackage[top = 1in, bottom = 1in, left =1in, right = 1in]{geometry}
\usepackage{graphicx}
\usepackage{amsmath}
\usepackage{tabu}
\usepackage{amssymb}
\usepackage{amsmath}
\usepackage{etoolbox}
\usepackage{xcolor}
\usepackage{amsthm}
\usepackage{tikz-cd}
\usepackage{tikz}
\usepackage{tkz-graph}
\usepackage{seqsplit}
\usepackage{ulem}
\usepackage{tabularx}
\AtBeginEnvironment{proof}{\color{blue}}
\newtheorem{theorem}{Theorem}
\newtheorem{proposition}{Proposition}
\newtheorem{lemma}{Lemma}
\newtheorem*{facts}{Fact}
\newtheorem*{remark}{Remark}
\newtheorem{corollary}{Corollary}
\newtheorem{definition}[theorem]{Definition}
\newtheorem{question}[theorem]{Question}
\newtheorem*{hint}{Hint}
\usepackage{enumerate}
\usepackage{hyperref}
\usepackage{fancyhdr}\pagestyle{fancy}
\newcommand{\la}{\langle}
\newcommand{\ra}{\rangle}
\newcommand{\tors}{\mathrm{tors}}
\newcommand{\ab}{\mathrm{ab}}
\newcommand{\Aut}{\operatorname{Aut}}
\newcommand{\Inn}{\operatorname{Inn}}
\newcommand{\im}{\operatorname{im}}
\newcommand{\lcm}{\operatorname{lcm}}
\newcommand{\ch}{\operatorname{char}}

%Math blackboard:
\newcommand{\bC}{\mathbb{C}}
\newcommand{\bF}{\mathbb{F}}
\newcommand{\bN}{\mathbb{N}}
\newcommand{\bQ}{\mathbb{Q}}
\newcommand{\bR}{\mathbb{R}}
\newcommand{\bS}{\mathbb{S}}
\newcommand{\bZ}{\mathbb{Z}}

%Math caligraphy
\newcommand{\cA}{\mathcal{A}}
\newcommand{\cC}{\mathcal{C}}
\newcommand{\cK}{\mathcal{K}}
\newcommand{\cM}{\mathcal{M}}
\newcommand{\cO}{\mathcal{O}}

%Greek blackboard font:
\newcommand{\bmu}{\mbox{$\raisebox{-0.59ex}
  {$l$}\hspace{-0.18em}\mu\hspace{-0.88em}\raisebox{-0.98ex}{\scalebox{2}
  {$\color{white}.$}}\hspace{-0.416em}\raisebox{+0.88ex}
  {$\color{white}.$}\hspace{0.46em}$}{}}

\lhead{University of California, Berkeley}
\rhead{Math 113, Spring 2022}

\begin{document}
\begin{center}
\Large {Takehome Assignment 2}\\
\small {Due Monday, April 4 at 5pm}
\end{center}
In this assignment, we complete the proof of Sylow's Theorems.  Let's recall the relevant definitions and statements.
\begin{definition}
Let $p$ be a prime number.  A group $H$ is called a \textbf{$p$-group} if $|H| = p^r$ for some $r$.  If $G$ is a group and $H\le G$ is a subgroup which is a $p$-group, we call it a \textbf{$p$-subgroup} of $G$.
\end{definition}
\begin{definition}\label{maindef}
Let $G$ be a finite group of order $|G| = p^\alpha m$ for $p$ a prime not dividing $m$.  A subgroup $P\la G$ of order $p^\alpha$ is called a \textbf{Sylow $p$-subgroup} of $G$.  The collection of all Sylow $p$-subgroups of $G$ is denoted $Syl_p(G)$ and the number of Sylow $p$-subgroups is often denoted $n_p = \#Syl_p(G)$.
\end{definition}
\begin{theorem}[Sylow's Theorems]
Adopt the notation from Definition \ref{maindef}.
\begin{itemize}
\item{
\textbf{(Sylow 1)} There exists a Sylow $p$-subgroup of $G$.
}
\item{
\textbf{(Sylow 2)} Let $P\in Syl_p(G)$ and let $Q\le G$ any $p$-subgroup of $G$.  Then there exists some $g\in G$ with $gQg^{-1}\le P$.
}
\item{
\textbf{(Sylow 3)} Let $P\in Syl_p(G)$.
\begin{enumerate}[(a)]
\item $n_p\equiv 1\mod p$.
\item $n_p = [G:N_G(P)]$.  In particular $n_p|m$.
\end{enumerate}
}
\end{itemize}
\end{theorem}
We already proved \textbf{(Sylow 1)} in class, \textbf{(Sylow 2)} and \textbf{(Sylow 3)} remain.  As is often the case, group actions will be a useful tool!  To help us along the way, we introduce one more definition.
\begin{definition}
Let $G$ be a group acting on a set $A$.  The fixed points of the action are:
\[A^G = \{a\in A:g\cdot a = a\textit{ for all }g\in G\}.\]
\end{definition}
\begin{enumerate}
\item Let's establish a few facts about the fixed points.
\begin{enumerate}
\item Let $G$ be a group.  Compute the fixed points of the following actions.
\begin{enumerate}
\item $G$ acting on $G$ by left multiplication.
\item $G$ acting on $G$ by conjugation.
\end{enumerate}
\item Let $G$ be a $p$-group acting on a finite set $A$.  Show that $|A^G|\equiv |A|\mod p$.  (\textit{Hint:} One could model this off of the proof of the class equation.  Use the orbit-stabilizer theorem to see what happens when reducing mod $p$).
\item Let $G$ be a $p$-group acting on a nonempty set $A$, and suppose that $p$ does not divide $|A|$.  Show that the action of $G$ on $A$ has at least one fixed point.
\end{enumerate}
\end{enumerate}
\textbf{(Sylow 2)} now follows from a clever application of 1(c).  All we have to do is look at the right group action!
\begin{enumerate}
\setcounter{enumi}{1}
\item Let $G$ be as in Definition \ref{maindef}, and $P$ a Sylow $p$-subgroup of $G$.  Let $Q\le G$ be a $p$-subgroup.
\begin{enumerate}
\item Use 1(c) to deduce that the action of $Q$ on $G/P$ by left multiplication has a fixed point.  (There are 2 cardinality conditions to apply 1(c), explain why they both hold.)
\item Use the fixed point of this action to show that a conjugate of $Q$ is contained in $P$, thereby proving \textbf{(Sylow 2)}.
\item Deduce that all Sylow $p$-subgroups of $G$ are conjugate and isomorphic.
\end{enumerate}
\end{enumerate}
The two parts of \textbf{(Sylow 3)} follow from the orbit-stabilizer theorem and clever application of 1(b), keeping careful track of the numerics!
\begin{enumerate}
\setcounter{enumi}{2}
\item Let $G$ be as in Definition \ref{maindef}, and $P$ a Sylow $p$-subgroup of $G$.
\begin{enumerate}
\item Show that $G$ acts on the set $Syl_p(G)$ by conjugation.  What is the stabilizer of $P$?
\item Use the orbit-stabilizer theorem of the action from part (a) to prove \textbf{(Sylow 3)(b)}.  (You can use 2(c) to compute the orbit $G*P$).
\item Restrict the action from part (a) to an action of $P$ on $Syl_p(G)$.  Show that the action of $P$ on $Syl_p(G)$ has a single fixed point: $P$ itself!
\item Deduce \textbf{(Sylow 3)(a)} from 1(b) and 3(c).
\end{enumerate}
\end{enumerate}
\textbf{Good job!  You did it!  We will explore many consequences of these results in the coming week!}
\end{document}
