\documentclass[11pt]{article}
\usepackage[top = 1in, bottom = 1in, left =1in, right = 1in]{geometry}
\usepackage{graphicx}
\usepackage{amsmath}
\usepackage{tabu}
\usepackage{amssymb}
\usepackage{amsmath}
\usepackage{etoolbox}
\usepackage{xcolor}
\usepackage{amsthm}
\usepackage{tikz-cd}
\usepackage{tikz}
\usepackage{tkz-graph}
\usepackage{seqsplit}
\usepackage{ulem}
\usepackage{tabularx}
\AtBeginEnvironment{proof}{\color{blue}}
\newtheorem{theorem}{Theorem}
\newtheorem{proposition}{Proposition}
\newtheorem{lemma}[theorem]{Lemma}
\newtheorem*{facts}{Fact}
\newtheorem*{remark}{Remark}
\newtheorem{corollary}{Corollary}
\newtheorem{definition}[theorem]{Definition}
\newtheorem{question}[theorem]{Question}
\newtheorem*{hint}{Hint}
\usepackage{enumerate}
\usepackage{hyperref}
\usepackage{fancyhdr}\pagestyle{fancy}
\newcommand{\la}{\langle}
\newcommand{\ra}{\rangle}
\newcommand{\tors}{\mathrm{tors}}
\newcommand{\ab}{\mathrm{ab}}
\newcommand{\Aut}{\operatorname{Aut}}
\newcommand{\Inn}{\operatorname{Inn}}
\newcommand{\im}{\operatorname{im}}
\newcommand{\lcm}{\operatorname{lcm}}
\newcommand{\ch}{\operatorname{char}}

%Math blackboard:
\newcommand{\bC}{\mathbb{C}}
\newcommand{\bF}{\mathbb{F}}
\newcommand{\bN}{\mathbb{N}}
\newcommand{\bQ}{\mathbb{Q}}
\newcommand{\bR}{\mathbb{R}}
\newcommand{\bS}{\mathbb{S}}
\newcommand{\bZ}{\mathbb{Z}}

%Math caligraphy
\newcommand{\cA}{\mathcal{A}}
\newcommand{\cC}{\mathcal{C}}
\newcommand{\cK}{\mathcal{K}}
\newcommand{\cM}{\mathcal{M}}
\newcommand{\cO}{\mathcal{O}}

%Math Frakture
\newcommand{\fN}{\mathfrak{N}}
\newcommand{\fJ}{\mathfrak{J}}
\newcommand{\fp}{\mathfrak{p}}
\newcommand{\fq}{\mathfrak{q}}
\newcommand{\fm}{\mathfrak{m}}

\newcommand{\maps}{\operatorname{Maps}}


%Greek blackboard font:
\newcommand{\bmu}{\mbox{$\raisebox{-0.59ex}
  {$l$}\hspace{-0.18em}\mu\hspace{-0.88em}\raisebox{-0.98ex}{\scalebox{2}
  {$\color{white}.$}}\hspace{-0.416em}\raisebox{+0.88ex}
  {$\color{white}.$}\hspace{0.46em}$}{}}

\lhead{University of California, Berkeley}
\rhead{Math 113, Spring 2022}

\begin{document}
\begin{center}
\Large {Takehome Assignment 3}\\
\small {Due Friday, May 13 \textbf{at 11:59 pm}}
\end{center}
\begin{enumerate}
\item{Let's start with some group theory!  For this first question, let $G$ be a finite group of order $n$.  We'd like to understand the size of the center of $G$, say $|Z(G)|  = z$.
\begin{enumerate}
\item{Show that it is not possible for $z$ to fall in the range $\frac{n}{4}<z<n$.}
\item{Show that these bounds are optimal.  That is, give examples of a group where $z=n$, and one where $z=\frac{n}{4}$.}
\end{enumerate}
}
\end{enumerate}
Now let's think about some special ideals in commutative unital rings.  We remind the reader of the following definition.
\begin{definition}
Let $R$ be a commutative unital ring.  An ideal $\fp\subseteq R$ is called a \textit{prime ideal} if $\fp\not=R$ and for any $a,b\in R$, if $ab\in\fp$ then either $a\in\fp$ or $b\in\fp$.
\end{definition}
\begin{enumerate}
\setcounter{enumi}{1}
\item{Let $R$ be a commutative ring with $1\not=0$.  Recall that in ideal $\fm\subseteq R$ is maximal if and only if $R/\fm$ is a field.  We will see there is a similar characterization of primality.
\begin{enumerate}
\item{Prove that an ideal $\fp\subseteq R$ is prime if and only if the quotient ring $R/\fp$ is an integral domain.}
\item{Prove that a maximal ideal $\fm\subseteq R$ is prime.}
\item{What are all the prime ideals of $\bZ$?}
\item{Prove that the ideal $(x)\subseteq\bZ[x]$ is prime but not maximal.}
\end{enumerate}
}
  \item{
  Let $\varphi:R\to S$ be a homomorphism between commutative unital rings with $\varphi(1_R)=1_S$.
  \begin{enumerate}
    \item{
    Let $\fq\subseteq S$ be a prime ideal.  Show that $\varphi^{-1}(\fq)$ is a prime ideal of $R$.
    }
    \item{
    Suppose $\varphi$ is surjective, and $\fm\subseteq S$ is a maximal ideal.  Show that $\varphi^{-1}(\fm)$ is a maximal ideal of $R$.
    }
    \item{
    Give a counterexample to part (b) if $\varphi$ is not surjective.
    }
  \end{enumerate}
  }
  \item{
  In this exercise we calculate the intersection of all the maximal ideals in a commutative unital ring $R$.  Given a ring $R$, we define the \textit{Jacobson radical} of $R$ to be the ideal:
  \[\fJ(R) = \bigcap_{\fm\subseteq R\text{ maximal}}\fm.\]
  \begin{enumerate}
    \item{
    Show that $\fN(R)\subseteq\fJ(R)$.
    }
    \item{
    Show that an element $r\in R$ is a unit if and only if it is not contained in any maximal ideal.
    }
    \item{
    Suppose $\fm$ is a maximal ideal and $r\in R\setminus\fm$.  Compute the ideal $(\fm,r)$ generated by $\fm$ and $r$.
    }
    \item{
    Prove that $r\in\fJ(R)$ if and only if $1-ry\in R^\times$ for every $y\in R$.  (Parts (b) and (c) might help!)
    }
  \end{enumerate}
  }
    \item{
  Let's finish by exploring unit groups. For parts (a)-(c) we do not assume that $R$ is commutative.  Recall that if $R$ is a (unital) ring, then $R^\times$ is the set of units, endowed with a group structure given by multiplication in $R$.
  \begin{enumerate}
    \item{
    Let $\varphi:R\to S$ be a (unital) homomorphism of rings.  Show that if $r\in R^\times$ then $\varphi(r)\in S^\times$.  Give a counterexample where $\varphi$ is not unital.
    }
    \item{
    Show that the restriction of $\varphi$ to $R^\times$ is a group homomorphism $\varphi^\times: R^\times\to S^\times$, which is injective if $\varphi$ is.
    }
    \item{
    The analogous statement does not hold for $\varphi$ surjective.  Give an example of a surjective (unital) homomorphism $\varphi:R\to S$, but such that the induced map on unit groups $\varphi^\times:R^\times\to S^\times$ is not surjective.
    }
    \item{
    Let $\varphi:R\to S$ be a surjective (unital) homomorphism of \textit{commutative} rings, and suppose that $\ker\varphi\subseteq\fJ(R)$.  Prove that the induced map $\varphi^\times:R^\times\to S^\times$ is surjective.
    }
  \end{enumerate}
}
\end{enumerate}
\textbf{Congratulations!!}  We've covered a ton of material and done a ton of problems this semester.  \textbf{Good work!}

\end{document}
