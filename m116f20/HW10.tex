\documentclass[11pt]{article}
\usepackage[top = 1in, bottom = 1in, left =1in, right = 1in]{geometry}
\usepackage{graphicx}
\usepackage{amsmath}
\usepackage{tabu}
\usepackage{amssymb}
\usepackage{etoolbox}
\usepackage{xcolor}
\usepackage{amsthm}
\usepackage{tikz-cd}
\usepackage{tikz}
\usepackage{seqsplit}
\usepackage{ulem}
\AtBeginEnvironment{proof}{\color{blue}}
\newtheorem{theorem}{Theorem}
\newtheorem{proposition}{Proposition}
\newtheorem{lemma}{Lemma}
\newtheorem*{facts}{Fact}
\newtheorem*{remark}{Remark}
\newtheorem{corollary}{Corollary}
\newtheorem{definition}{Definition}
\usepackage{enumerate}
\usepackage{hyperref}
\usepackage{fancyhdr}\pagestyle{fancy}
\newcommand{\la}{\langle}
\newcommand{\ra}{\rangle}
\newcommand{\tors}{\mathrm{tors}}
\newcommand{\ab}{\mathrm{ab}}
\newcommand{\Aut}{\operatorname{Aut}}
\newcommand{\Inn}{\operatorname{Inn}}
\newcommand{\im}{\operatorname{im}}
\newcommand{\lcm}{\operatorname{lcm}}
\newcommand{\ch}{\operatorname{char}}

%Math blackboard:
\newcommand{\bC}{\mathbb{C}}
\newcommand{\bF}{\mathbb{F}}
\newcommand{\bN}{\mathbb{N}}
\newcommand{\bQ}{\mathbb{Q}}
\newcommand{\bR}{\mathbb{R}}
\newcommand{\bS}{\mathbb{S}}
\newcommand{\bZ}{\mathbb{Z}}

%Math caligraphy
\newcommand{\cC}{\mathcal{C}}
\newcommand{\cK}{\mathcal{K}}
\newcommand{\cM}{\mathcal{M}}
\newcommand{\cO}{\mathcal{O}}

%Greek blackboard font:
\newcommand{\bmu}{\mbox{$\raisebox{-0.59ex}
  {$l$}\hspace{-0.18em}\mu\hspace{-0.88em}\raisebox{-0.98ex}{\scalebox{2}
  {$\color{white}.$}}\hspace{-0.416em}\raisebox{+0.88ex}
  {$\color{white}.$}\hspace{0.46em}$}{}}

\lhead{University of California, Berkeley}
\rhead{Math 116, Fall 2020}

\begin{document}
\begin{center}
\Large {Homework 10}\\
\small {Due Thursday, November 19}
\end{center}
\section*{Implementation Part}
\begin{enumerate}
  \item{
  Implement Pollard's $\rho$ algorithm to solve the discrete log problem $g^x = h\mod p$.  You should define an algorithm \verb|PollardRhoLog(g,h,p)| which takes as input a prime $p$ and $g,h\in\bF_p^*$ with $g$ a primitive root, and outputs the solution to $g^x = h\mod p$.  Your algortithm should use as a mixing function:
  \[f(x)\equiv
  \begin{cases}
    gx \mod p & 0\le x <p/3\\
    x^2\mod p & p/3\le x< 2p/3\\
    xh\mod p & 2p/3\le x<p
  \end{cases},\]
  And should compute $x_i = \underbrace{(f\circ f\circ\cdots\circ f)}_{i\text{ times}}(x)$ and $y_i = x_{2i}$.  Exploit that each $x_i = g^{\alpha_i}h^{\beta_j}$ and similarly for the $y_i$.  You will need to keep track of these exponents as well, but you should not be making a list (we described how to do this in class).  (Hint: A collision will let you compute the discrete log of a power of $h$.  Passing from this to the discrete log of $h$ should look a lot like HW7 4(c)).
  }
  \item{
  Use \verb|PollardRhoLog| to solve the following..
  \begin{enumerate}
    \item{
    $3^t\equiv 5\mod 17$.
    }
    \item{
    $19^t \equiv 24717\mod48611$.  (Note, this is Example 5.52 in the book so you can double check if your algorithm worked).
    }
    \item{
    $29^t \equiv 5953042 \mod15239131$.
    }
    \item{
    $2^t \equiv 2598854876 \mod2810986643$
    }
  \end{enumerate}
  }
  \item{
  This problem goes hand in hand with Problem 4 in the written part of the assignment, implementing Pollard's $\rho$ method to factor large numbers.
  \begin{enumerate}
    \item{
    Program an algorithm \verb|PollardRhoFactor(N,f,x = 2)| which implements the algorith described in Problem 4 to find a nontrivial factor of $N$.  It should take as input a large number $N$, a mixing function $f$, and an initial value for the mixing function $x$ (which will initialize to 2 if not given).  It should also print the number $k$ of steps it took to find the nontrivial factor and the ration $\sqrt N/k$ (for our analysis in Problem 4).
    }
    \item{
    Test out \verb|PollardRhoFactor| with mixing function $f(x) = x^2+1$ to find a nontrivial factor of:
    \begin{enumerate}
      \item{$2201$}
      \item{$9409613$}
      \item{$1782886219$}
    \end{enumerate}
    }
    \item{
    Repeat part (b) with a mixing function of $f(x) = x^2+2$.
    }
    \item{
    Repeat part (b) with a mixing function of $f(x) = x^2$.
    }
    \item{
    Repeat part (b) with a mixing function of $f(x) = x^2-2$.
    }
    \item{
    Test out \verb|PollardRhoFactor| on some prime numbers.
    }
    \item{
    Write a function \verb|PollardRhoFactorize(N,f,x=2)| which repeatedly uses \verb|PollardRhoFactor| to find a complete factorization of $N$.
    }
  \end{enumerate}
  }
\end{enumerate}
\section*{Written Part}
\begin{enumerate}
  \setcounter{enumi}{3}
  \item{
  This problem goes hand in hand with Problem 3 in the impelementation part of this assignment.  We describe how (the abstact version of) Pollard's $\rho$ method can be used to factor large numbers $N$ relatively quickly.  It works best when $N$ has a relatively small prime factor $p$.  We first describe the method.  Suppose you have a mixing function:
  \[f:\bZ/N\bZ\to\bZ/N\bZ.\]
  Let $x_0=y_0\in\bZ_N\bZ$, and compute $x_{i+1} = f(x_i)$ and $y_{i+1} = f(f(y_i))$.  At each step compute:
  \[g_i = \gcd(|x_i-y_i|,N).\]
  \begin{enumerate}
    \item{
    Suppose $f$ is sufficiently random and let $p$ be the smallest prime divisor of $N$.  Show that with high probability we find some $g_k = p$ for $k=\cO(\sqrt{p})$.
    }
    \item{
    Compare what happened in 3(b) and 3(c).  Did one have a faster run time?  Why?
    }
    \item{
    Explain what happened in 3(d) when the mixing function was $f(x) = x^2$.
    }
    \item{
    Explain what happened in 3(e) when the mixing function was $f(x) = x^2-2$.
    }
    \item{
    Explain what happened in 3(f) when $N$ was prime.
    }
  \end{enumerate}
  }
\end{enumerate}
In class we stated and proved the forward direction of the following theorem.
\begin{theorem}\label{secrecy}
  Fix a cryptosystem with $\#\cM = \#\cC = \#\cK$.  The system has perfect secrecy if and only if the following two conditions hold.
  \begin{enumerate}[(1)]
    \item{
    Each key $k\in\cK$ is used with equal probability.
    }
    \item{
    For each plaintext $m\in\cM$ and ciphertext $c\in\cC$ there exists a unique key $k\in\cK$ with $e_k(m) = c$.
    }
  \end{enumerate}
\end{theorem}
\begin{enumerate}
  \setcounter{enumi}{4}
  \item{
  Complete the proof of Theorem \ref{secrecy} by proving the \textit{only if} direction.  That is, assuming conditions (1) and (2) hold, show the system has perfect secrecy.
  }
  \item{
  Prove the following identities for binomial coefficients.  (Parts (c) and (d) generalize computations in HW8 Problems 3(e) and 3(f)).
  \begin{enumerate}
    \item{$\sum_{k=0}^n{n\choose k} = 2^n$}
    \item{$\sum_{k=0}^n(-1)^k{n\choose k} = 0$}
    \item{$\sum_{k\ge0}{n\choose 2k} = 2^{n-1}$}
    \item{$\sum_{k\ge0}{n\choose 2k+1} = 2^{n-1}$}
  \end{enumerate}
  }
  \item{
  Consider the elliptic curve $E$ given by the equation $y^2 = x^3 - 2x + 4$.  Let $P = (0,2)$ and $Q = (3,-5)$.
  \begin{enumerate}
    \item{
    Show $P,Q\in E$.
    }
    \item{
    Compute $P\oplus Q$.
    }
    \item{
    Compute $P\oplus P$.
    }
    \item{
    Compute $P\oplus P\oplus P$.
    }
  \end{enumerate}
  }
\end{enumerate}
\end{document}
