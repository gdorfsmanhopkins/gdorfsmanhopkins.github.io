\documentclass[11pt]{article}
\usepackage[top = 1in, bottom = 1in, left =1in, right = 1in]{geometry}
\usepackage{graphicx}
\usepackage{amsmath}
\usepackage{tabu}
\usepackage{amssymb}
\usepackage{etoolbox}
\usepackage{xcolor}
\usepackage{amsthm}
\usepackage{tikz-cd}
\usepackage{tikz}
\usepackage{seqsplit}
\usepackage{ulem}
\AtBeginEnvironment{proof}{\color{blue}}
\newtheorem{theorem}{Theorem}
\newtheorem{proposition}{Proposition}
\newtheorem{lemma}{Lemma}
\newtheorem*{facts}{Fact}
\newtheorem*{remark}{Remark}
\newtheorem{corollary}{Corollary}
\newtheorem{definition}{Definition}
\usepackage{enumerate}
\usepackage{hyperref}
\usepackage{fancyhdr}\pagestyle{fancy}
\newcommand{\la}{\langle}
\newcommand{\ra}{\rangle}
\newcommand{\tors}{\mathrm{tors}}
\newcommand{\ab}{\mathrm{ab}}
\newcommand{\Aut}{\operatorname{Aut}}
\newcommand{\Inn}{\operatorname{Inn}}
\newcommand{\im}{\operatorname{im}}
\newcommand{\lcm}{\operatorname{lcm}}
\newcommand{\ch}{\operatorname{char}}

%Math blackboard:
\newcommand{\bC}{\mathbb{C}}
\newcommand{\bF}{\mathbb{F}}
\newcommand{\bN}{\mathbb{N}}
\newcommand{\bQ}{\mathbb{Q}}
\newcommand{\bR}{\mathbb{R}}
\newcommand{\bS}{\mathbb{S}}
\newcommand{\bZ}{\mathbb{Z}}

%Math caligraphy
\newcommand{\cC}{\mathcal{C}}
\newcommand{\cK}{\mathcal{K}}
\newcommand{\cM}{\mathcal{M}}
\newcommand{\cO}{\mathcal{O}}

%Greek blackboard font:
\newcommand{\bmu}{\mbox{$\raisebox{-0.59ex}
  {$l$}\hspace{-0.18em}\mu\hspace{-0.88em}\raisebox{-0.98ex}{\scalebox{2}
  {$\color{white}.$}}\hspace{-0.416em}\raisebox{+0.88ex}
  {$\color{white}.$}\hspace{0.46em}$}{}}

\lhead{University of California, Berkeley}
\rhead{Math 116, Fall 2020}

\begin{document}
\begin{center}
\Large {Homework 9}\\
\small {Due Saturday, November 14}
\end{center}
\section*{Implementation Part}
\begin{enumerate}
  \item{
  Implement the discrete log collision algorithm \verb|dLogCollide(g,h,p,n)| which takes as input a prime $p$, a primitive root $g\in\bF_p^*$, an element $h\in\bF_p^*$, and returns the discrete log $\log_g h$.  It should also take an integer $n$ which should default to $\lfloor\sqrt{p-1}\rfloor$ if no $n$ is given, this is the length of your lists in the collision algorithm.  It should do the following:
  \begin{enumerate}[(1)]
    \item{
    Compute a list $\{g^i\}$ for $n$ random integers $i$.
    }
    \item{
    Compute a list $\{hg^j\}$ for $n$ random integers $j$.
    }
    \item{
    Find an overlap $g^i = hg^j$ between the two lists and use this to return the discrete log.
    }
  \end{enumerate}
  In order to optimize the search for an overlap, you can use a \verb|set| or hash table like in HW5 Problem 2.
  }
  \item{
  Test out your algorithm to compute the following discrete logs:
  \begin{enumerate}
    \item{
    $\log_2 390$ in $\mathbb{F}_{659}$
    }
    \item{
    $\log_{10}106$ in $\mathbb{F}_{811}$
    }
  \end{enumerate}
  }
\end{enumerate}
\section*{Written Part}
\begin{enumerate}
  \setcounter{enumi}{2}
  \item{
  Let $X:\Omega\to\bR$ be a random variable, taking values in the set $\{x_1,x_2,\cdots x_r\}$.
  \begin{enumerate}
    \item{
    Show that if $a\not=b$ then the events $(X=a)$ and $(X=b)$ are disjoint.
    }
    \item{
    Show that
    \[\Omega = \bigcup_{i}(X=x_i)\]
    }
    \item{
    Show that
    \[\sum_{i=1}^r f_X(x_i) = 1.\]
    }
    \item{
    Recall that the expected value of $X$ was defined to be:
    \[E(X) = \sum_{i=1}^r x_i f_X(i).\]
    Prove that it this is equal to the folloing value:
    \[\sum_{\omega\in\Omega}X(\omega)Pr(\omega).\]
    }
  \end{enumerate}
  }
  \item{
  In the following cases compute the expected value of the variable $X$
  \begin{enumerate}
    \item{
    $X$ is uniformly distributed on $\{0,1,\cdots,N-1\}$.
    }
    \item{
    $X$ is uniformly distributed on $\{1,2,\cdots,N\}$.
    }
    \item{
    $X$ is uniformly distributed on the first 7 prime numbers.
    }
    \item{
    $X$ is a random variable with a binomial density function.  (Hint: use the binomial theorem and a differentiation to get a closed form for the sum).
    }
  \end{enumerate}
  }
  \item{
  In this problem we will use probability and expected values to study why the \verb|findPrime| algorithm from problem 1 was so successful.
  \begin{enumerate}
    \item{
    Let $L<U$ be a positive integers.  Use the prime number theorem to estimate
    \[\rho = \rho(L,U)=\text{(the probability that a randomly chosen number $n$ with $L<n\le U$ is prime)}\]
    in terms of $L$ and $U$.
    }
    \item{
    Let $\Omega$ the set of outcomes consisting of infinite sequences of numbers between $L$ and $U$:
    \[\Omega = \{n_1,n_2,n_3,\cdots | L< n_i\le U\text{ for all }i\}.\]
    Let $X:\Omega\to\bZ$ be the random variable whose value is number of guesses until the first prime.  That is:
    \[X(n_1n_2n_3...) = i\Longleftrightarrow n_i\text{ is prime and }n_j\text{ is not prime for any }j<i.\]
    Let $a$ be a positive integer.  Compute the probability density $f_X(i)$ in terms of $i$ and the probability $\rho$ from part (a).  (That is, what is the probability that the $i$th number is the first prime?)
    }
    \item{
    Compute the expected value $E(X)$.  Interpret in words what this number means.  (This computation should look a lot like the expected value of the coin flipping example in the 11/5 lecture).
    }
    \item{
    Use part (c) to estimate the following:
    \begin{enumerate}
      \item{If I randomly guess 2 digit numbers how many guesses will it take to find a prime?}
      \item{If I randomly guess 100 digit numbers how many guesses will it take to find a prime?}
      \item{If I randomly guess 500 digit numbers how many guesses will it take to find a prime?}
    \end{enumerate}
    }
    \item{
    Use the evidence you've gathered to explain why \verb|findPrime| from the first project was successful.
    }
  \end{enumerate}
  }
  \item{
  Suppose 23 random people are in a room.  Compute the probability that at least 2 of them share a birthday.  (This is the most well known statement of the \textit{birthday paradox}).
  }
\end{enumerate}
The next problem concerns the following theorem from class, for which we did prove part (i).
\begin{theorem}\label{collision}
  Suppose there is an urn with $N$ balls, of which $n$ are red and $N-n$ are blue.  Suppose further that you randomly choose $m$ balls, replacing after each selection.
  \begin{enumerate}[(i)]
    \item{
    $Pr($at least one red$) = 1-\left(1-\frac{n}{N}\right)^m$
    }
    \item{
    $Pr($at least one red$) \ge 1-e^{\frac{-mn}{N}}$
    }
    \item{
    If $N$ is large and $m,n$ are not much larger than $\sqrt N$ then the estimate from (ii) is quite accurate
    }
  \end{enumerate}
\end{theorem}
\begin{enumerate}
  \setcounter{enumi}{6}
  \newpage
  \item{
  Lets prove parts (ii) and (iii) of Theorem \ref{collision}.  We may use part (i) in our proofs since it was established in class.
  \begin{enumerate}
    \item{
    \[e^{-x} \ge 1-x\text{ for all }x.\]
    (Hint: use calculus to find the global minimum of $e^{-x} - (1-x))$).
    }
    \item{
    Use part (a) and Theorem \ref{collision}(i) to prove Theorem \ref{collision}(ii).
    }
    \item{
    Prove that for all $a>1$ and $0\le x\le 1$ the following inequality holds.
    \[e^{-ax}\le (1-x)^a +\frac{1}{2}ax^2.\]
    }
    \item{
    Use part (c) and Theorem \ref{collision}(i) to prove the following identity:
    \[Pr(\text{at least one red})\le 1-e^{\frac{-mn}{N}} + \frac{mn^2}{2N^2}.\]
    Use this to deduce Theorem \ref{collision}(iii).
    }
  \end{enumerate}
  }
  \item{
  In the Miller-Rabin problem I suggested that we interpret the Prime Number Theorem as saying the probability of a number n being prime is $\frac{\ln n}{n}$, but of course this way off the mark.   The prime number theorem says there are $\frac{n}{\ln n}$ primes less than n.  Thus the probability of one being prime in particular is approximately
  \[\frac{n/\ln n}{n} = \frac{1}{\ln n}.\]
  Notice that $\frac{1}{\ln n}$ is MUCH LARGER DENSITY than $\frac{\ln n}{n}$.
  \begin{enumerate}
    \item{
    To really feel the difference between these two densities, use each to compute the probability that a random number less than $10^{100}$ is prime.  This should illustrate the gravity of the mistake.}
  \end{enumerate}
  The beauty of the prime number theorem is it says primes are rather dense, and the value I gave said they are extremely sparse.  This likely affected the expected correctness of your Miller-Rabin computation.
  \begin{enumerate}
    \setcounter{enumii}{1}
    \item{
    Redo the computations from HW8 Problem 8 with this correct probability, so that we have computed the correct values.  (Don't worry about re-deriving everything, just show the formula and plug in the correct values.)  In particular you should show that:
    \[Pr\left(\text{n is prime }|\text{ Miller-Rabin Fails }N\text{ times}\right)\ge 1-\frac{\ln n}{4^N}.\]
    In particular, how confident are we in are primes when implemented RSA, where $N=20$ and $2^{511}\le n< 2^{512}$?
    }
  \end{enumerate}
  }
\end{enumerate}
\end{document}
