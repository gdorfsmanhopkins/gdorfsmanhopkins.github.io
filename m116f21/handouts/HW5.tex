\documentclass[11pt]{article}
\usepackage[top = 1in, bottom = 1in, left =1in, right = 1in]{geometry}
\usepackage{graphicx}
\usepackage{amsmath}
\usepackage{tabu}
\usepackage{amssymb}
\usepackage{etoolbox}
\usepackage{xcolor}
\usepackage{amsthm}
\usepackage{tikz-cd}
\usepackage{tikz}
\AtBeginEnvironment{proof}{\color{blue}}
\newtheorem{theorem}{Theorem}
\newtheorem{proposition}{Proposition}
\newtheorem{lemma}{Lemma}
\newtheorem*{facts}{Fact}
\newtheorem*{remark}{Remark}
\newtheorem{corollary}{Corollary}
\newtheorem{definition}{Definition}
\usepackage{enumerate}
\usepackage{hyperref}
\usepackage{fancyhdr}\pagestyle{fancy}
\newcommand{\la}{\langle}
\newcommand{\ra}{\rangle}
\newcommand{\tors}{\mathrm{tors}}
\newcommand{\ab}{\mathrm{ab}}
\newcommand{\Aut}{\operatorname{Aut}}
\newcommand{\Inn}{\operatorname{Inn}}
\newcommand{\im}{\operatorname{im}}
\newcommand{\lcm}{\operatorname{lcm}}
\newcommand{\ch}{\operatorname{char}}

%Math blackboard:
\newcommand{\bC}{\mathbb{C}}
\newcommand{\bF}{\mathbb{F}}
\newcommand{\bN}{\mathbb{N}}
\newcommand{\bQ}{\mathbb{Q}}
\newcommand{\bR}{\mathbb{R}}
\newcommand{\bS}{\mathbb{S}}
\newcommand{\bZ}{\mathbb{Z}}

%Math caligraphy
\newcommand{\cC}{\mathcal{C}}
\newcommand{\cK}{\mathcal{K}}
\newcommand{\cM}{\mathcal{M}}
\newcommand{\cO}{\mathcal{O}}

%Greek blackboard font:
\newcommand{\bmu}{\mbox{$\raisebox{-0.59ex}
  {$l$}\hspace{-0.18em}\mu\hspace{-0.88em}\raisebox{-0.98ex}{\scalebox{2}
  {$\color{white}.$}}\hspace{-0.416em}\raisebox{+0.88ex}
  {$\color{white}.$}\hspace{0.46em}$}{}}

\lhead{University of California, Berkeley}
\rhead{Math 116, Fall 2021}

\begin{document}
\begin{center}
\Large {Homework 5}\\
\small {Due Thursday, October 7}
\end{center}
\section*{Implementation Part}
\begin{enumerate}
  \item{
  Implement Sun-Tzu's algorithm for solving concurrent congruences (in the book this is called the Chinese Remainder Theorem).  Specifically, define a function \verb|SunTzu(moduli,residues)| which satisfies the following:
  \begin{center}
    \begin{tabular}{c|c}
      Input & Output\\
      \hline
      A list of moduli $m_1,\cdots,m_t$ (positive integers) & If moduli are pairwise coprime\\
      A list of integers $a_1,\cdots,a_t$ &$x$ satisfying $x\equiv a_i\mod m_i$ for all $i$.\\
      &Otherwise an error message.
    \end{tabular}
  \end{center}
  \textit{Hints}:
  \begin{itemize}
    \item{One way you could do this is to make an auxiliary function \verb|SunTzuPairs(m1,m2,a1,a2)| which solves the problem for 2 congruences, and have \verb|SunTzu| feed recursively into \verb|SunTzuPairs|}
    \item{
    Naively checking the moduli are coprime takes running the Euclidean algorithm $\cO(t^2)$ times, but you should be able to do so only running it $\cO(t)$ time.
    }
  \end{itemize}
  }
  \item{
  Implement the Pohlig-Hellman algorithm to solve the DLP for an element $g\in\bF_p^*$ of order $N = m_1m_2...m_t$ (for coprime $m_i$).  Specifically, define a function \verb|pohligHellman(g,h,p,factors)|
  \begin{center}
    \begin{tabular}{c|c}
      Input & Output\\
      \hline
      A prime $p$ & $\log_g(h)$ if it exists\\
      An element $g\in\bF_p^*$ & \\
      An element $h\in\bF_p^*$ & \\
      The prime power factors $m_1,\cdots m_t$ of $|g|$ & \
    \end{tabular}
  \end{center}
  \textit{Hints}
  \begin{itemize}
    \item{
    The structure should loosely be as follows.  Reduce the problem to solving the DLP for elements of smaller order, let \verb|babyGiant| solve those problems (make sure to tell it the order is smaller, otherwise you aren't saving any time), and then use \verb|SunTzu| to stitch them together.
    }
    \item{
    It is difficult in general to compute $|g|$ (about as difficult as factoring $p-1$), and so checking if the $m_i$ are indeed the prime factors of $|g|$ may be difficult.  Instead, check that $g^{m_1m_2...m_t} = 1$.  In this case your algorithm should still work (see Problem 6).
    }
  \end{itemize}
  }
  \item{
  Use \verb|SunTzu| to solve to following sets of congruences, and check that the solution given works.
  \begin{enumerate}
    \item{
      $x \equiv 9\mod 23$ and $x = 25\mod 41$
    }
    \item{
    \begin{eqnarray*}
      x&\equiv& 1\mod2\\
      x&\equiv& 2\mod3\\
      x&\equiv& 4\mod5\\
      x&\equiv& 6\mod7\\
      x&\equiv& 10\mod 11\\
      x&\equiv& 1\mod 13\\
      x&\equiv& 16\mod 17
    \end{eqnarray*}
    }
  \end{enumerate}
  }
  \item{
  Let's test out Pohlig-Hellman.
  \begin{enumerate}
    \item{
    Let $p = 113$.  Last week we used baby steps-giant steps to compute $\log_3 19$ modulo $p$.  Notice that $112$ factors as $2^4*7$.  Use this information and Pohlig-Hellman to compute $\log_3 19$ and see if your answer matches.
    }
    \item{
    Let $p = 30235367134636331149$.  Last week we tried using baby steps-giant steps to compute the discrete log $\log_6 3295$ modulo $p$.  You might have had trouble getting it to run.  I did.  What if I told you that $p-1$ has the following prime factorization?
    \[p-1 = 2^2 * 3^2 * 13 * 41143 * 335341 * 4682597.\]
    Now use Pohlig-Hellman to speed up your computation.  (It speeds it up considerably!).  Use fast powering to make sure you got the right answer (it is very satisfying!).
    }
  \end{enumerate}
  }
  \end{enumerate}
  \section*{Written Part}
  \begin{enumerate}
  \setcounter{enumi}{4}
  \item{
  For \verb|pohligHellman| instead of checking that the $m_i$ were indeed the prime power factors of $|g|$, we just checked that $g^{m_1m_2\cdots m_t} = 1$.  Prove that if this condition holds (and the $m_i$ are still coprime) that \verb|pohligHellman| returns the correct logarithm.
  }
  \item{
  Show that \verb|SunTzu| runs in $\cO(\log N)$ steps where $N = m_1m_2\cdots m_t$ is the product of the moduli.  (You may assume your basic operations $+,-,\times,\div,\%$ are all $\cO(1)$.
  }
  \item{
  Let's prove the uniqueness part Sun-Tzu's theorem.
  \begin{enumerate}
    \item{
    Let $a,b,c$ be positive integers and suppose that:
    \begin{eqnarray*}
      a|c,&b|c,&\gcd(a,b) = 1.
    \end{eqnarray*}
    Then $ab|c$.
    }
    \item{
    Suppose $m_1,\cdots,m_t$ are pairwise coprime positive integers, and suppose $a_1,\cdots,a_t\in\bZ$.  Show that if $y$ and $z$ are both solutions to the system of congruences
    \begin{eqnarray*}
      x&\equiv& a_1\mod m_1\\
      x&\equiv& a_2\mod m_2\\
      &\vdots&\\
      x&\equiv& a_t\mod m_t,
    \end{eqnarray*}
     then $y\equiv z\mod m_1m_2\dots m_t$
     }
   \end{enumerate}
  }
  \end{enumerate}
  Let's finish by proving the following theorem:
  \begin{theorem}\label{SquareRoots}
  Let $m$ be an odd number and $a$ an integer not divisble by any of the prime factors of $m$.  Then $a$ has a square root mod $m$ if and only if $a^{\frac{p-1}{2}}\equiv 1\mod p$ for every prime factor $p$ of $m$.
  \end{theorem}
  \begin{enumerate}
  \setcounter{enumi}{7}
  \item{
  \begin{enumerate}
    \item{
    Let $a$ be an integer not divisble by an odd prime $p$.  Show that $a$ has a square root mod $p$ if and only if $a^{\frac{p-1}{2}}\equiv 1\mod p$. (\textit{Hint:} Use HW2 Problem 8.)
    }
    \item{
    Let $m = p_1^{\alpha_1}p_2^{\alpha_2}\cdots p_t^{\alpha_t}$ and $a$ an integer.  Show that $a$ has a square root mod $m$ if and only if it has a square root mod $p_i^{\alpha_i}$ for each $i$. (\textit{Hint:} Use Sun-Tzu's Theorem.)
    }
    \item{
    Let $m$ be an odd number and suppose $a$ is an integer not divisible by any prime factor of $m$.  Show $a$ has a square root mod $m$ if and only if it has a square root mod $p$ for every prime $p$ dividing $m$. (\textit{Hint:} Use HW4 Problem 7).
    }
    \item{
    Deduce Theorem \ref{SquareRoots} from parts (a),(b), and (c) above.
    }
    \item{
    Can you relax any of the hypotheses of Theorem \ref{SquareRoots}?  For example, what if $m$ is even?  Or what if some prime factor of $m$ divides $a$?  Compute some examples and informally discuss your thoughts.
    }
    \item{
      Explain why part (a) also solves the bonus question of HW3 Problem 6(f).
    }
  \end{enumerate}
  }
  \end{enumerate}
\end{document}
