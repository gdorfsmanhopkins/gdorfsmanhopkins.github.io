\documentclass[11pt]{article}
\usepackage[top = 1in, bottom = 1in, left =1in, right = 1in]{geometry}
\usepackage{graphicx}
\usepackage{amsmath}
\usepackage{tabu}
\usepackage{amssymb}
\usepackage{etoolbox}
\usepackage{xcolor}
\usepackage{amsthm}
\usepackage{tikz-cd}
\usepackage{tikz}
\usepackage{seqsplit}
\usepackage{ulem}
\AtBeginEnvironment{proof}{\color{blue}}
\newtheorem{theorem}{Theorem}
\newtheorem{proposition}{Proposition}
\newtheorem{lemma}{Lemma}
\newtheorem*{facts}{Fact}
\newtheorem*{remark}{Remark}
\newtheorem{corollary}{Corollary}
\newtheorem{definition}{Definition}
\usepackage{enumerate}
\usepackage{hyperref}
\usepackage{fancyhdr}\pagestyle{fancy}
\newcommand{\la}{\langle}
\newcommand{\ra}{\rangle}
\newcommand{\tors}{\mathrm{tors}}
\newcommand{\ab}{\mathrm{ab}}
\newcommand{\Aut}{\operatorname{Aut}}
\newcommand{\Inn}{\operatorname{Inn}}
\newcommand{\im}{\operatorname{im}}
\newcommand{\lcm}{\operatorname{lcm}}
\newcommand{\ch}{\operatorname{char}}

%Math blackboard:
\newcommand{\bC}{\mathbb{C}}
\newcommand{\bF}{\mathbb{F}}
\newcommand{\bN}{\mathbb{N}}
\newcommand{\bQ}{\mathbb{Q}}
\newcommand{\bR}{\mathbb{R}}
\newcommand{\bS}{\mathbb{S}}
\newcommand{\bZ}{\mathbb{Z}}

%Math caligraphy
\newcommand{\cC}{\mathcal{C}}
\newcommand{\cK}{\mathcal{K}}
\newcommand{\cM}{\mathcal{M}}
\newcommand{\cO}{\mathcal{O}}

%Greek blackboard font:
\newcommand{\bmu}{\mbox{$\raisebox{-0.59ex}
  {$l$}\hspace{-0.18em}\mu\hspace{-0.88em}\raisebox{-0.98ex}{\scalebox{2}
  {$\color{white}.$}}\hspace{-0.416em}\raisebox{+0.88ex}
  {$\color{white}.$}\hspace{0.46em}$}{}}

\lhead{University of California, Berkeley}
\rhead{Math 116, Fall 2021}

\begin{document}
\begin{center}
\Large {Homework 8}\\
\small {Due Thursday, November 4}
\end{center}
\section*{Implementation Part}
\begin{enumerate}
  \item{
  Let's begin by writing functions that efficiently compute Legendre and Jacobi symbols.
  \begin{enumerate}
    \item{
    Write a function \verb|legendreSymbol(a,p)| which takes as input an integer $a$ and an \textit{odd} prime $p$, and returns the Legendre symbol $\left(\frac{a}{p}\right)$ in $\cO(\log(p))$ time.
    }
    \item{
    Write a function \verb|jacobiSymbol(a,b)| which takes as input integers $a$ and $b$ where $b$ is odd and positive and returns the Jacobi symbol $\left(\frac{a}{b}\right)$ \textit{without factoring b}.  We remind you of the following properties of Jacobi symbols which should help with your computation.
    \begin{itemize}
      \item{
      This only depends on the residue of $a$ modulo $b$.
      }
      \item{
      If $a\equiv -1,0,1,2\mod b$ this is easy to compute directly (using quadratic reciprocity for $-1$ and $2$).
      }
      \item{
      If $b$ is prime then this is a Legendre symbol!  (\verb|probablyPrime| will help determine this quickly).
      }
      \item{
      You can use quadratic reciprocity to relate $\left(\frac{a}{b}\right)$ and $\left(\frac{b}{a}\right)$.  Since we can reduce $b$ modulo $a$ this gives us a strictly smaller problem!  (\textbf{Warning:}, if $a$ is even the $\left(\frac{b}{a}\right)$ doesn't make sense!  You will have to factor out the 2's from a use the multiplicativity of the Jacobi function to deal with this case!)
      }
    \end{itemize}
    }
    \item{
    Compute the following Jacobi symbols.  For the first 3 you can check your work by hand. \[\left(\frac{8}{15}\right),\left(\frac{11}{15}\right),\left(\frac{12}{15}\right),\left(\frac{171337608}{536134436237}\right).\]
    }
  \end{enumerate}
  }
\end{enumerate}
\section*{Written Part}
\begin{enumerate}
  \setcounter{enumi}{1}
  \item{
  This problem is part written, part implementation.  We'll walk through a toy example of using the index-calculus to solve a discrete log.  There will be some calculations you'll want to do in Sage.  Turn in these calculations as part of the implementation part, labelling the cells as ``Calculations for Problem 2".

  Let $g = 17$ and $p=19079$.  Let's compute $\log_g 19$.
  \begin{enumerate}
    \item{Verify in Sage that $g^i\mod p$ is 5-smooth for $i=3030,6892,18312$.  Record their factorizations.  (You may use Sage's \verb|factor| function.)}
    \item{Let $x_\ell = \log_g\ell$ for $\ell=2,3,5$.  Use the factorizations from part (a) to right down 3 linear equations modulo $p-1$ that $x_2,x_3,x_5$ satisfy.}
    \item{Notice that $p-1 = 2*q$ where $q = 9539$ is prime.  Therefore you can use Gaussian elimination to solve for $x_2,x_3$, and $x_5$ modulo 2 and modulo $q$.  You are welcome to use Sage to do this.  Now use Sun-Tzu's theorem to compute $x_2,x_3,x_5$.}
    \item{Verify in Sage that $19g^{-12400}$ is 5-smooth.  Record it's factorization.}
    \item{Use the factorization from part (d) to write $\log_g19$ in terms of of $x_2,x_3,x_5$.  Therefore, using part (c), compute $\log_g19$.}
    \item{Verify that your answer is correct using fast powering.}
  \end{enumerate}
  }
  \item{
  \begin{enumerate}
    \item{Let $p$ be prime.  Verify that the Legendre Symbol satisfies the following 2 identities}
    \begin{enumerate}
      \item{
      $\left(\frac{ab}{p}\right) = \left(\frac{a}{p}\right)\left(\frac{b}{p}\right)$.
      }
      \item{
      If $a\equiv b\mod p$ then $\left(\frac{a}{p}\right) = \left(\frac{b}{p}\right)$.
      }
    \end{enumerate}
    \item{Verify that the Jacobi Symbol satisfies the following 3 identities.}
    \begin{enumerate}
      \item{
      $\left(\frac{a_1a_2}{b}\right) = \left(\frac{a_1}{b}\right)\left(\frac{a_2}{b}\right).$
      }
      \item{
      $\left(\frac{a}{b_1b_2}\right) = \left(\frac{a}{b_1}\right)\left(\frac{a}{b_2}\right).$
      }
      \item{
      If $a_1\equiv a_2\mod b$ then $\left(\frac{a_1}{b}\right) = \left(\frac{a_2}{b}\right)$.
      }
    \end{enumerate}
  \end{enumerate}
    }
  \item{
  Compute the Jacobi symbols $\left(\frac{8}{15}\right),\left(\frac{11}{15}\right),\left(\frac{12}{15}\right)$ by hand and confirm your solutions from 1(c) are correct.
  }
  \item{
  Here we give another characterization of the Legendre symbol from a group theoretic perspective.
  \begin{enumerate}
    \item{
    Let $G,H,K$ be groups, and let $\varphi:G\to H$ and $\psi:H\to K$ be homomorphisms.  Show that the composition $\psi\circ\varphi:G\to K$ is a homomorphism.
    }
    \item{
    Show that the set $\{\pm1\}$ is a group under multiplication.
    }
    \item{
    Let $N$ be a positive even integer.  Show that the map $\bZ/N\bZ\to\{\pm1\}$ given by the rule $x\mapsto (-1)^x$ is a well defined homomorphism (where the group law for $\bZ/N\bZ$ is addition).  What goes wrong if $N$ is odd?
    }
    \item{
    Let $p$ be an odd prime, and let $g\in\bF_p$ be a primitive root.  Show that the composition
    \[
    \begin{tikzcd}
      \bF_p^*\ar[r,"\log_g(\cdot)"]&\bZ/(p-1)\bZ\ar[r,"(-1)^x"]&\{\pm1\}
    \end{tikzcd}
    \]
    is equal to the Legendre function $x\mapsto\left(\frac{x}{p}\right)$.  Use this together with part (a)-(c) to give another proof that the Legendre symbol is multiplicative.
    }
  \end{enumerate}
  }
  \item{
  On previous assignments we've extesively studied the notion of squares modulo $p$ (i.e., \textit{quadratic residues mod $p$}), and one thing we noticed is that the situation differed depending on whether $p$ was even or odd (i.e., it depended on the residue of $p$ modulo $2$).  Here we begin our exploration of cube roots modulo $p$, and we will notice that the story depends on the the residue of $p$ modulo 3.  First a definition:
  \begin{definition}
    Let $p$ be a prime number.  An integer $a$ is called a \textit{cubic residue mod $p$} if $p\not|a$ and there exists an integer $c$ satisfying $c^3\equiv a\mod p$.
  \end{definition}
  Let's begin by studying the case where $p\equiv 1\mod 3$.  \textbf{For parts (a)-(d), assume $p\equiv1\mod 3$}.
  \begin{enumerate}
    \item{
    Let $a,b$ be cubic residues modulo $p$.  Show that $ab$ is a cubic residue mod $p$.
    }
    \item{
    Give an example to show that if $a$ and $b$ are cubic nonresidues mod $p$, then $ab$ could also be a nonresidue.  Explain why this is different from the situation of quadratic residues.
    }
    \item{
    Let $g$ be a primitive root for $\bF_p$.  Show that $a$ is a cubic residue modulo $p$ if and only if $\log_g a$ is a multiple of 3.
    }
    \item{
    Show that if $a$ is a cubic residue modulo $p$, then $a$ has precisely 3 cube roots modulo $p$.
    }
    \item{
    Part (c) showed that if $p\equiv 1\mod 3$ then one third of the elements of $\bF_p^*$ have cube roots.  The case where $p\equiv 2\mod 3$ is quite different.  Suppose $p\equiv 2\mod 3$.  Show that every integer has a cube root modulo $p$.  If $p\not|a$, how many cube roots does $a$ have mod $p$?
    }
    \item{
    Like in the case of square roots mod 2, the case of cube roots mod 3 is different still.  Show that every integer has \textit{precisely 1} cube root modulo 3.
    }
    \item{
    In fact, it is a general principle that $p$th roots modulo $p$ are very simple.  Prove that if $p$ is prime every integer has precisely one $p$th root modulo $p$.  (\textit{Hint}: Fermat's little theorem.)
    }
  \end{enumerate}
  }
  \item{
  In class we suggested that the problem of factoring a number $N$ is in some sense equivalent to being able to compute square roots modulo $N$.  In this problem we will make this precise, for $N=pq$ a product of 2 distinct \textbf{odd} primes.  Simultaneously, we will verify that most integers have four square roots mod $pq$, which was an important input in the quadratic sieve.
  \begin{enumerate}
    \item{
    Suppose you know the factorization of $N$ into $pq$.  Describe an algorithm to efficiently compute whether $a$ has a square root modulo $N$, and prove the correctness of your algorithm.
    }
    \item{
    Suppose $\gcd(a,N)=1$.  Show that if $a$ has one square root modulo $N$, then it exactly 4 square roots modulo $N$.  In the case where $\gcd(a,N)\not=1$, how many square roots might $a$ have? Why?
    }
    \item{
    Suppose you know the factorization of $N$ into $pq$.  Describe an algorithm to compute all the square roots of $a$ modulo $N$ if they exist.  Prove the correctness of your algoritm.  (You may assume you have a fast algorithm to compute square roots modulo primes.)
    }
    \item{
    Conversely, suppose you have an oracle that can tell you all the square roots of $a$ modulo $N$ if they exist.  Describe a way to use constultation with this oracle to factor $N$.  Prove your method works.
    }
  \end{enumerate}
}
\end{enumerate}
\end{document}
